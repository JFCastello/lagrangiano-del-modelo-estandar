%use 
% $ latexbiblio2itex file.tex 
% to generate instiki file
%instiki:category: FisicaSubatomica
%instiki:
%instiki:# Bibliograf\'\i a #
%instiki:
%instiki:***
%instiki:
%instiki:[[NotasFS|Tabla de Contenidos]]
%instiki:
%instiki:***

\begin{thebibliography}{999}
\bibitem{kane} 
Modern Elementary Particle Physics, Gordon Kane, Perseus Publishing, 1993.

\bibitem{cottingham}
An Introduction to Standard Model of Particle Physics. W.N Cottingham and D.A. Greenwood, Cambridge University Press, 1988

\bibitem{r}
Quantum Field Theory, L.H Reyder, Cambridge University Press

\bibitem{q}
Quantum Field Theory, F. Mandl, G. Shaw, John Wiley \& Sons, INC. 1993

\bibitem{a} 
A. Pich, The Standard Model of Electroweak Interactions, hep-ph/0502010

\bibitem{aa} 
The Standard Model: Alchemy and Astrology, hep-ph/0609274

\bibitem{ultimo} 
Relativistic Quantum Mechanics and Field Theory, Franz Gross, John Wiley \& Sons, INC. 1993

\bibitem{ActionPhysics} 
\url{http://es.wikipedia.org/wiki/Principio_de_m\%C3\%ADnima_acci\%C3\%B3n}, 
\url{http://en.wikipedia.org/wiki/Action_\%28physics\%29}


\bibitem{JavaAP}\url{http://www.eftaylor.com/software/ActionApplets/LeastAction.html}

\bibitem{0712.1608} 
``Lagrangian Densities and Principle of Least Action in Nonrelativistic Quantum Mechanics'', Donald H. Kobe, arXiv:0712.1608

\bibitem{NewtonSeconLaw} 
\url{http://es.wikipedia.org/wiki/Leyes_de_Newton#Segunda_Ley_de_Newton_o_Ley_de_la_Fuerza}, 
\href{http://en.wikipedia.org/wiki/Newton\%27s_laws_of_motion#Newton.27s_second_law:_law_of_acceleration}{http://en.wikipedia.org/wiki/Newton\%27s\_laws\_of\_motion}

\bibitem{Gross}
Relativistic Quantum Mechanics and Field Theory, Franz Gross, Wiley Interscience, 1993, Chapter 1.

\bibitem{Gauss}
\url{http://es.wikipedia.org/wiki/Teorema_de_la_divergencia}

\bibitem{daelembertiano}
\url{http://en.wikipedia.org/wiki/D%27Alembertian}

\bibitem{Goldhaber:1971mr}
  A.~S.~Goldhaber and M.~M.~Nieto,
  ``Terrestrial and extra-terrestrial limits on the photon mass,''
  Rev.\ Mod.\ Phys.\  {\bf 43} (1971) 277.

\bibitem{Yao:2006px}
  W.~M.~Yao {\it et al.}  [Particle Data Group],
  ``Review of particle physics,''
  J.\ Phys.\ G {\bf 33}, 1 (2006).

\bibitem{Pauli}
\url{http://en.wikipedia.org/wiki/Pauli_matrices}

\bibitem{Englert:1964et}
  F.~Englert and R.~Brout,
  ``Broken Symmetry and the Mass of Gauge Vector Mesons,''
  Phys.\ Rev.\ Lett.\  {\bf 13} (1964) 321.
  %%CITATION = PRLTA,13,321;%%
  %2524 citations counted in INSPIRE as of 25 Jul 2014

\bibitem{Higgs:1964pj}
  P.~W.~Higgs,
  ``Broken Symmetries and the Masses of Gauge Bosons,''
  Phys.\ Rev.\ Lett.\  {\bf 13} (1964) 508.
  doi:10.1103/PhysRevLett.13.508.
  %%CITATION = doi:10.1103/PhysRevLett.13.508;%%
  %3829 citations counted in INSPIRE as of 09 Jun 2016
  P.~W.~Higgs,
  ``Broken symmetries, massless particles and gauge fields,''
  Phys.\ Lett.\  {\bf 12} (1964) 132.
  doi:10.1016/0031-9163(64)91136-9.
  %%CITATION = doi:10.1016/0031-9163(64)91136-9;%%
  %3551 citations counted in INSPIRE as of 09 Jun 2016
  P.~W.~Higgs,
  ``Spontaneous Symmetry Breakdown without Massless Bosons,''
  Phys.\ Rev.\  {\bf 145} (1966) 1156.
  doi:10.1103/PhysRev.145.1156
  %%CITATION = doi:10.1103/PhysRev.145.1156;%%
  %2477 citations counted in INSPIRE as of 09 Jun 2016



\bibitem{Pich:2005mk}
  A.~Pich,
  ``The standard model of electroweak interactions,''
  arXiv:hep-ph/0502010,
  Published in *Sant Feliu de Guixols 2004, European School of High-Energy Physics* 1-48.

\bibitem{NU}
\url{http://en.wikipedia.org/wiki/Natural_units}
\bibitem{PU}
\url{http://en.wikipedia.org/wiki/Plank_units}

\bibitem{Aitchison:2003tq}
  I.~J.~R.~Aitchison and A.~J.~G.~Hey,
  ``Gauge theories in particle physics: A practical introduction. Vol. 1: From
  relativistic quantum mechanics to QED,''
%\href{http://www.slac.stanford.edu/spires/find/hep/www?irn=5562635}{SPIRES entry}
{\it  Bristol, UK: IOP (2003) 406 p}

\bibitem{andim}
\url{http://groups.google.com/group/sci.physics.research/msg/e6cc1b288df8bbb2}

\bibitem{PDG} 
 W.~M.~Yao {\it et al.}  [Particle Data Group],
  ``Review of particle physics,''
  J.\ Phys.\ G {\bf 33} (2006) 1.

\bibitem{Ryder:1985wq}
  L.~H.~Ryder,
  ``Quantum Field Theory,''
{\it  Cambridge, Uk: Univ. Pr. ( 1985) 443p}
%\href{http://www.slac.stanford.edu/spires/find/hep/www?irn=1444956}{SPIRES entry}

\bibitem{GN} 
\url{http://en.wikipedia.org/wiki/Gell-Mann–Nishijima_formula}

\bibitem{spin}
\url{http://en.wikipedia.org/wiki/Spin_(physics)}

\bibitem{LEP}
\url{http://en.wikipedia.org/wiki/LEP}

\bibitem{muon}
\url{http://en.wikipedia.org/wiki/Muon}

\bibitem{uslhcblog}\url{http://blogs.uslhc.us/?p=481}

\bibitem{Veltman} Lie Groups in Physics, M.J.G Veltman (English version by G. 't Hooft)

\bibitem{Brading:2000hc}
  K.~Brading and H.~R.~Brown,
  ``Noether's theorems and gauge symmetries,''
  hep-th/0009058.


\bibitem{Brading:2003nv}
  K.~Brading and E.~Castellani,
  ``Symmetries in physics: Philosophical reflections,''
  Cambridge, UK: Univ. Pr. (2003) 445 p

\bibitem{Sundermeyer:2014kha}
  K.~Sundermeyer,
  ``Symmetries in fundamental physics,''
  Fundam.\ Theor.\ Phys.\  {\bf 176} (2014).
  doi:10.1007/978-94-007-7642-5

\bibitem{Noehter} Emmy~Noether,Invariant variation problems, arXiv:physics/0503066.

\bibitem{Yang:1954ek}
  C.~N.~Yang and R.~L.~Mills,
  ``Conservation of Isotopic Spin and Isotopic Gauge Invariance,''
  Phys.\ Rev.\  {\bf 96} (1954) 191.
  doi:10.1103/PhysRev.96.191
  %%CITATION = doi:10.1103/PhysRev.96.191;%%
  %2101 citations counted in INSPIRE as of 09 Jun 2016

\bibitem{2015arXiv151203827R}
Y.~Rodriguez, ``A New Pedagogical Way of Finding Out the Gauge Field Strength Tensor in Abelian and Non-Abelian Local Gauge Field Theories'', arXiv:1512.03827



\bibitem{Weinberg:1967tq}
  S.~Weinberg,
  ``A Model of Leptons,''
  Phys.\ Rev.\ Lett.\  {\bf 19} (1967) 1264.
  doi:10.1103/PhysRevLett.19.1264
  %%CITATION = doi:10.1103/PhysRevLett.19.1264;%%
  %10239 citations counted in INSPIRE as of 09 Jun 2016

\bibitem{rodgriges} Rodrigues' rotation formula. In Wikipedia. Retrieved August 10, 2018, from \url{https://en.wikipedia.org/wiki/Rodrigues'_rotation_formula}


\bibitem{Zee:2016fuk}
  A.~Zee,
  ``Group Theory in a Nutshell for Physicists,''
  %%CITATION = INSPIRE-1695330;%%

\bibitem{Dreiner:2008tw}
  H.~K.~Dreiner, H.~E.~Haber and S.~P.~Martin,
  ``Two-component spinor techniques and Feynman rules for quantum field theory and supersymmetry,''
  Phys.\ Rept.\  {\bf 494} (2010) 1
  doi:10.1016/j.physrep.2010.05.002
  [arXiv:0812.1594 [hep-ph]].
  %%CITATION = doi:10.1016/j.physrep.2010.05.002;%%
  %315 citations counted in INSPIRE as of 14 Aug 2019


\bibitem{Quevedo:2010ui}
F.~Quevedo, S.~Krippendorf and O.~Schlotterer,
%``Cambridge Lectures on Supersymmetry and Extra Dimensions,''
[arXiv:1011.1491 [hep-th]].
See also: \url{http://www.damtp.cam.ac.uk/user/examples/3P7.pdf}
\href{https://drive.google.com/file/d/1HpTXflJToqHr6cEa2b3IkWQ8E1MTjp_v/view?usp=sharing}{Google Drive}

\bibitem{Greiner:1990tz}
W.~Greiner,
``Relativistic quantum mechanics: Wave equations,''
%24 citations counted in INSPIRE as of 27 Apr 2020

%\cite{Aoyama:2019ryr}
\bibitem{Aoyama:2019ryr}
T.~Aoyama, T.~Kinoshita and M.~Nio,
``Theory of the Anomalous Magnetic Moment of the Electron,''
Atoms \textbf{7} (2019) no.1, 28
doi:10.3390/atoms7010028
%8 citations counted in INSPIRE as of 22 May 2020
\end{thebibliography}




%%% Local Variables: 
%%% mode: latex
%%% TeX-master: "fullnotes"
%%% End: