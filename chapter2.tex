%instiki:category: FisicaSubatomica
%http://localhost:2500/wiki/show/Cap%C3%ADtulo+II
%% Ilustrar como el teorema de Noether es el mismo
%% para el caso local que global
\chapter{Campos vectoriales}
\label{cha:campos-vectoriales} %noinstiki
%instiki:
%instiki:***
%instiki:
%instiki:[[NotasFS|Tabla de Contenidos]]
%instiki:
%instiki:***
%generated with html2itexTOC instiki_source.html
%instiki:
%instiki:* [Unidades Naturales](#NU)
%instiki:
%instiki:* [Notaci\'on relativista](#srn)
%instiki:
%instiki:  * [Ejemplos de cuadrivectores](#ejemplos_de_cuadrivectores)
%instiki:
%instiki:  * [Ecuaciones covariantes](#ecuac-covar)
%instiki:
%instiki:* [Ecuaciones de Maxwell en notaci\'on covariante](#maxeqs)
%instiki:
%instiki:  * [Lagrangiano Electromagn\'etico](#lagr-electr)
%instiki:
%instiki:  * [Energ\'\i a del campo electromagn\'etico](#energa_del_campo_electromagntico)
%instiki:
%instiki:  * [Fijaci\'on del gauge](#fijacion-del-gauge)
%instiki:
%instiki:* [Ecuaciones de Proca](#ecuacion-de-proca)
%instiki:
%instiki:* [Problemas](#problemas2)
%instiki:
%instiki:***
%instiki:

 
\section{Lagrangiano electromagnético}
\begin{frame}[fragile,allowframebreaks]
Las ecuaciones de Maxwell tienen la simetría obtenida en la ec.~\eqref{eq:aphicov}
\begin{align}
  \label{eq:aphicovch2}
   A^\mu\to {A'}^\mu=&A^\mu-\partial^\mu\theta(x)
\end{align}
 que cuando fue descubierta parecía ser una simple curiosidad matemática de dichas ecuaciones.

Comparando con la transformación más general de un campo vemos, que para un campo vectorial con dicha invarianza: $a_i=0$ y $b^{\mu_i}=1$ en la ec.\eqref{eq:infdt}. 

Ya estamos en capacidad de responder la siguiente pregunta: ¿Cual es el Lagrangiano más general posible para el campo de cuatro componentes $A^{\mu}(x)$ compatible con la invarianza de Lorentz y la invarianza bajo transformaciones gauge?


De hecho, si aplicamos el segundo teorema de Noether al campo de radiación $A^\mu(x)$
\begin{align}
  0=&\partial_{\mu} \left( \mathcal{E}_3 b^{\mu}_3 \right) \nonumber\\
   0=&\partial_{\mu} \left( \mathcal{E}_3 \delta^{\mu}_\nu \right) \nonumber\\
   0=&
 \partial_{\mu} \partial_{\rho} \left[ \frac{\partial \mathcal{L}}{\partial \left(  \partial_{\rho} A_{\nu}\right)}  \delta^{\mu}_\nu \right] - \partial_{\mu} \frac{\partial \mathcal{L}}{\partial A_{\nu}}\delta^{\mu}_{\nu} \nonumber\\
   0=&
 \partial_{\mu} \partial_{\rho}\left[ \frac{\partial \mathcal{L}}{\partial \left(  \partial_{\rho} A_{\mu}\right)}  \right] - \partial_{\mu} \frac{\partial \mathcal{L}}{\partial A_{\mu}} \,.
\end{align}

Podemos definir el cuadrivector
\begin{align}
  j^\mu(x)\equiv \frac{\partial \mathcal{L}}{\partial A_{\mu}}\,.
\end{align}
Si asumimos que este corresponde a un cuadricorriente conservada, tenemos
\begin{align}
  \partial_{\mu} \frac{\partial \mathcal{L}}{\partial A_{\mu}}=&
\partial_{\mu} j^{\mu}=0\,,
\end{align}
Por consiguiente, la identidad resultante 
\begin{align}
   \partial_{\mu} \partial_{\rho}\left[ \frac{\partial \mathcal{L}}{\partial \left(  \partial_{\rho} A_{\mu}\right)}  \right]=0\,,
\end{align}
se puede interpretar como la necesidad de introducir el tensor antisimétrico
\begin{align}
  F_{\mu\nu}=\frac{\partial \mathcal{L}}{\partial \left(  \partial_{\nu} A_{\mu}\right)}\,,
\end{align}
tal que
\begin{align}
  \partial_{\mu}\partial_{\nu}F^{\mu\nu}=0\,.
\end{align}
Para obtener una forma para $F_{\mu\nu}$, es conveniente imponer que la densidad Lagrangiana asociada sólo a las nuevas contribuciones de los campos $A_{\nu}$ y sus derivadas $\partial_{\mu}A_{\nu}$, denotada como $\mathcal{L}_{\text{EM}}$,
%en la ec.
sea invariante gauge local bajo la transformación del campo $A_{\mu}$ en \eqref{eq:tfgl}. Esto implíca que $\mathcal{L}_{\text{EM}}$  solo puede depender de las derivadas de las campos, y por consiguiente $F^{\mu\nu}$ debe ser una combinación antisimétrica sus derivadas
\begin{align}
  F_{\mu\nu}\equiv\partial_{\mu}A_{\nu}-\partial_{\nu}A_{\mu}\,.
\end{align}
Con esta definición, bajo la transformación  \eqref{eq:tfgl}
\begin{align}
  F_{\mu\nu}\to F_{\mu\nu}'=F_{\mu\nu}\,.
\end{align}

En efecto, para el campo electromagnético invariante bajo $\operatorname{U}(1)_Q$, podemos considerar como campo fundamental al tensor de campo electromagnético, que bajo el Grupo de Lorentz transforma como
\begin{align}
  F_{\mu\nu} \to F'_{\mu\nu}=&\partial'_{\mu} A'_{\nu} -\partial'_{\nu} A'_{\mu} \nonumber\\
=&g_{\nu\alpha}\partial'_{\mu} A^{\alpha} -\partial'_{\nu} {A'}_{\mu} \nonumber\\
=&g_{\nu\alpha} { \Lambda_{\mu}}^{\beta}\partial_{\beta} {\Lambda^{\alpha}}_{\gamma} {A}^{\gamma} -\partial'_{\nu} {A'}_{\mu} \nonumber\\
=&g_{\nu\alpha} { \Lambda_{\mu}}^{\beta}\partial_{\beta} {\Lambda^{\alpha}}_{\gamma} g^{\gamma\delta} A_{\delta} -\partial'_{\nu} A'_{\mu} \nonumber\\
  =& { \Lambda_{\mu}}^{\beta}\partial_{\beta}   A_{\delta}{\Lambda_{\nu}}^{\delta} -\partial'_{\nu} A'_{\mu} \nonumber\\
  =& { \Lambda_{\mu}}^{\beta}\partial_{\beta}   A_{\delta}{\left( \Lambda^{-1} \right)^{\delta}}_{\nu} -\partial'_{\nu} A'_{\mu} \,.
\end{align}
Aplicando el resultado al segundo término
\begin{align}
  F_{\mu\nu} \to F'_{\mu\nu} =& { \Lambda_{\mu}}^{\beta}\partial_{\beta}   A_{\delta}{\left( \Lambda^{-1} \right)^{\delta}}_{\nu} 
     - { \Lambda_{\nu}}^{\delta}\partial_{\delta} A_{\beta}{\left( \Lambda^{-1} \right)^{\beta}}_{\mu}  \nonumber\\
  =& { \Lambda_{\mu}}^{\beta} \left( \partial_{\beta}   A_{\delta}-      \partial_{\delta} A_{\beta} \right)
{\left( \Lambda^{-1} \right)^{\delta}}_{\nu} \,.
\end{align}
Finalmente
\begin{align}
  F_{\mu\nu}(x) \to F'_{\mu\nu}(x)  =& { \Lambda_{\mu}}^{\alpha}F_{\alpha\beta}\left( \Lambda^{-1}x \right)\,
                   {\left( \Lambda^{-1} \right)^{\beta}}_{\nu} \,.
\end{align}
que tiene la misma forma que la tranformación $\operatorname{SU}(3)$ de la matrix de tensores de campos de gluones en la ec.~\eqref{eq:ugum}.

De esta forma
\begin{align}
\label{eq:invlorff}
  F_{\mu\nu}F^{\mu\nu}\to   g^{\mu\gamma}g^{\nu\delta} F'_{\mu\nu} F'_{\gamma\delta}=&
  g^{\mu\gamma}g^{\nu\delta} { \Lambda_{\mu}}^{\alpha}  {\left( \Lambda^{-1} \right)^{\beta}}_{\nu}
{ \Lambda_{\gamma}}^{\sigma}  {\left( \Lambda^{-1} \right)^{\rho}}_{\delta}
F_{\alpha\beta}F_{\sigma\rho} \nonumber\\
  =&
   { \Lambda_{\mu}}^{\alpha}  {\left( \Lambda^{-1} \right)^{\beta}}_{\nu}
{ \Lambda^{\mu}}_{\xi}g^{\xi\sigma}  {\left( \Lambda^{-1} \right)_{\eta}}^{\nu}g^{\eta\rho}
F_{\alpha\beta}F_{\sigma\rho}  \nonumber\\
  =& {\left( \Lambda^{-1} \right)^{\alpha}}_{\mu} { \Lambda^{\mu}}_{\xi}
    {\left( \Lambda^{-1} \right)^{\beta}}_{\nu}
 {\Lambda^{\nu}}_{\eta} g^{\xi\sigma}  g^{\eta\rho}
F_{\alpha\beta}F_{\sigma\rho} \nonumber\\
  =& \delta^{\alpha}_{\xi}\delta^{\beta}_{\eta}g^{\xi\sigma}  g^{\eta\rho}
F_{\alpha\beta}F_{\sigma\rho} \nonumber\\
  =& g^{\alpha\sigma}  g^{\beta\rho}
F_{\alpha\beta}F_{\sigma\rho} \nonumber\\
  =& F_{\alpha\beta}F^{\alpha\beta}\,.
\end{align}



Por consiguiente, el único término posible que a la vez es invariante de Lorentz e invariante gauge local es
\begin{align}
  \label{eq:lemfree}
  \mathcal{L}_{\text{EM}}=-\frac{1}{4}F_{\mu\nu}F^{\mu\nu}\,.
\end{align}
%%hablar de la ecuaciones de Euler-Lagrange, el gauge de Lorentz y la ecuación de onda.


Podemos coprobar que si
\begin{align}
F^{\rho\sigma}F_{\rho\sigma}=&(\partial^\rho A^\sigma-\partial^\sigma A^\rho)(\partial_\rho A_\sigma-\partial_\sigma A_\rho)\nonumber\\
=&\partial^\rho A^\sigma\partial_\rho A_\sigma-\partial^\rho A^\sigma\partial_\sigma A_\rho-\partial^\sigma A^\rho\partial_\rho A_\sigma+\partial^\sigma A^\rho\partial_\sigma
A_\rho\nonumber\\
=&g^{\rho\alpha}g^{\sigma\beta}(\partial_\alpha A_\beta\partial_\rho A_\sigma-\partial_\alpha A_\beta\partial_\sigma A_\rho-\partial_\beta A_\alpha\partial_\rho A_\sigma+\partial_\beta A_\alpha\partial_\sigma A_\rho).\nonumber
\end{align}
Entonces
\begin{align}
  \frac{\partial}{\partial(\partial_\mu A_\nu)}F^{\rho\sigma}F_{\rho\sigma}=&g^{\rho\alpha}g^{\sigma\beta}(\delta_{\alpha\mu}\delta_{\beta\nu}\partial_\rho
  A_\sigma+\partial_\alpha A_\beta\delta_{\rho\mu} \delta_{\sigma\nu}-\delta_{\alpha\mu}\delta_{\beta\nu}\partial_\sigma A_\rho-\partial_\alpha A_\beta\delta_{\sigma\mu}\delta_{\rho\nu}\nonumber\\
&-\delta_{\beta\mu}\delta_{\alpha\nu}\partial_\rho A_\sigma-\partial_\beta A_\alpha\delta_{\rho\mu}\delta_{\sigma\nu}+\delta_{\beta\mu}\delta_{\alpha\nu}\partial_\sigma A_\rho+\partial_\beta
A_\alpha\delta_{\sigma\mu}\delta_{\rho\nu}).\nonumber\\
=&  g^{\rho\mu}g^{\sigma\nu}\partial_\rho A_\sigma+g^{\mu\alpha}g^{\nu\beta}\partial_\alpha A_\beta-g^{\rho\mu}g^{\sigma\nu}\partial_\sigma A_\rho-g^{\nu\alpha}g^{\mu\beta}\partial_\alpha A_\beta\nonumber\\
 &-g^{\rho\nu}g^{\sigma\mu}\partial_\rho A_\sigma-g^{\mu\alpha}g^{\nu\beta}\partial_\beta A_\alpha+g^{\rho\nu}g^{\sigma\mu}\partial_\sigma A_\rho+g^{\nu\alpha}g^{\mu\beta}\partial_\beta A_\alpha\nonumber\\
=&  \partial^\mu A^\nu+\partial^\mu A^\nu-\partial^\nu A^\mu-\partial^\nu A^\mu-\partial^\nu A^\mu-\partial^\nu A^\mu+\partial^\mu A^\nu+\partial^\mu A^\nu\nonumber\\
=&4(\partial^\mu A^\nu-\partial^\nu A^\mu) \nonumber\\
\label{eq:dddmufmunu2}
\frac{\partial}{\partial(\partial_\mu A_\nu)}F^{\rho\sigma}F_{\rho\sigma}&=4F^{\mu\nu}
\end{align}
y en efecto
\begin{align}
  F_{\mu\nu}=\frac{\partial \mathcal{L}}{\partial \left(\partial  \partial_{\nu} A_{\mu}\right)}\,,
\end{align}
tal que
\begin{align}
  \partial_{\mu}\partial_{\nu}F^{\mu\nu}=0\,.
\end{align}


El Lagrangiano en ec.~\eqref{eq:lemfree} admite un término adicional:
\begin{align}
    \mathcal{L}=-\frac{1}{4}F^{\mu\nu}F_{\mu\nu}-J^\mu A_\mu\,,
\end{align}
siempre y cuando $J^\mu(x)$ sea una corriente vectorial conservada:
\begin{align}
J^\mu(x)\to  {J'}^\mu(x)=&\Lambda^\mu_\nu J^\nu(\Lambda^{-1}x)\,,&  \partial_\mu J^\mu(x)=&0\,.
\end{align}

De hecho,  usando la ec.~\eqref{eq:consvjmu}, tenemos que para la transformación interna del campo $A^\mu$, eq.~\eqref{eq:aphicovch2}
\begin{align}
  \delta\mathcal{L}=\mathcal{L}'-\mathcal{L}=&
-\frac{1}{4}{F'}^{\mu\nu}F'_{\mu\nu}-J^\mu A'_\mu+\frac{1}{4}F^{\mu\nu}F_{\mu\nu}+J^\mu A_\mu\nonumber\\
  =&-J^\mu A_\mu+J^\mu\partial_\mu\theta(x)+J^\mu A_\mu\nonumber\\
  =&J^\mu\partial_\mu\theta(x)\nonumber\\
=&\partial_\mu(J^\mu\chi)-(\partial_\mu J^\mu)\theta(x)
\end{align}
Para la Acción
\begin{align}
  \delta S=&\int d^4x \left[\partial_\mu(J^\mu\chi)-(\partial_\mu J^\mu)\chi(x)\right]\nonumber\\
=&-\int d^4x (\partial_\mu J^\mu)\chi(x)\nonumber\\
=&-\int d^3x\int_{-\infty}^\infty dt (\partial_\mu J^\mu)\chi(x)\,.
\end{align}
Para tener $\delta S=0$ necesitamos asumir de momento que
$\partial_\mu J^\mu=0$. Sin embargo, veremos que esta es una condición
auto consistente.


En resumen, si la corriente electromagnética, $J^\mu$ se conserva, entonces la Acción es invariante bajo la transformación gauge \eqref{eq:aphicovch2}. Note que la densidad Lagrangiana no es localmente invariante gauge. Sin embargo, la Acción (y por consiguiente la teoría) es invariante gauge.
\end{frame}

\begin{frame}[fragile,allowframebreaks]
Por lo tanto, el Lagrangiano
\begin{equation}
  \label{eq:lagAmum}
  \mathcal{L}=-\frac{1}{4}F^{\mu\nu}F_{\mu\nu}-J^\mu A_\mu
\end{equation}
es el más general que da lugar a una Acción invariante de Lorentz e invariante gauge
local. 

\end{frame}

\begin{frame}[fragile,allowframebreaks]

Usando la ec.~\eqref{eq:dddmufmunu2}, tenemos
\begin{align}
\label{eq:177qft}
  \partial_\mu\left[
    \frac{\partial\mathcal{L}}{\partial(\partial_\mu A_\nu)}  
  \right]-\frac{\partial\mathcal{L}}{\partial A_\nu}&=0\nonumber\\
  -\frac{1}{4}\partial_\mu\left[
    \frac{\partial}{\partial(\partial_\mu A_\nu)}(F^{\rho\sigma}F_{\rho\sigma})  
  \right]+J^\rho\frac{\partial A_\rho}{\partial A_\nu}&=0\nonumber\\
  -\partial_\mu F^{\mu\nu}+J^\rho\delta_{\rho\nu}&=0\nonumber\\
  \partial_\mu F^{\mu\nu}&=J^\nu.
\end{align}
Como era de esperarse una Acción invariante de Lorentz e invariante
gauge local, expresada en términos del Lagrangiano \eqref{eq:lagAmum},
da lugar a la Teoría Electromagnética. 
\end{frame}



\section{Ecuaciones de Maxwell en notaci\'on covariante }
\label{sec:maxeqs}
Ecuaciones homog\'eneas:
\begin{align}
  \label{eq:hom_m_eq}
  \boldsymbol{\nabla}\cdot\mathbf{B}&=0,&\boldsymbol{\nabla}\times\mathbf{E}+\frac{\partial\mathbf{B}}{\partial t}&=0
\end{align}
Ecuaciones inhomog\'eneas:
\begin{align}
  \label{eq:inhom_m_eq}
  \boldsymbol{\nabla}\cdot\mathbf{E}&=\rho,&\boldsymbol{\nabla}\times\mathbf{B}-\frac{\partial\mathbf{E}}{\partial t}&=\mathbf{J}.
\end{align}
La primera ecuaci\'on establece la ausencia de cargas magn\'eticas, la segunda corresponde a la Ley de Faraday y la tercera a la Ley de Gauss. La cuarta sin el t\'ermino de desplazamiento el\'ectrico introducido por Maxwell corresponde a la Ley de Amp\`ere
\begin{equation}
   \boldsymbol{\nabla}\times\mathbf{B}=\mathbf{J}.
\end{equation}
Tomando la divergencia en esta expresi\'on tenemos
\begin{equation}
  \boldsymbol{\nabla}\cdot\mathbf{J}=0,
\end{equation}
que corresponde a la ecuaci\'on de continuidad \eqref{eq:conti} para $\rho$ constante
\begin{equation}
  \label{eq:153}
  \frac{\partial \rho}{\partial t}+\boldsymbol{\nabla}\cdot\mathbf{J}=0.
\end{equation}
De este modo la Ley de Amp\`ere da lugar a la conservaci\'on de carga el\'ectrica pero solo a nivel global:  una perdida de carga el\'ectrica en un punto del universo puede ser compensada por la aparici\'on instant\'anea de carga el\'ectrica en otro lugar del universo. La conservaci\'on global podr\'\i a necesitar la propagaci\'on instant\'anea de se\~nales, y esto est\'a en conflicto con la relatividad especial.


Tomando la divergencia de la Ley de Amp\`ere modificada por Maxwell
\begin{equation}
   \boldsymbol{\nabla}\times\mathbf{B}-\frac{\partial\mathbf{E}}{\partial t}=\mathbf{J},
\end{equation}
obtenemos la ecuaci\'on de continuidad \eqref{eq:153}. Dicha ecuaci\'on establece que la raz\'on de decrecimiento de la carga en un volumen arbitrario $V$ es debido precisa y \'unicamente al flujo de la corriente fuera de su superficie; de modo que la carga no puede ser creada ni destruida dentro de $V$.  Ya que $V$ puede ser arbitrariamente peque\~no esto significa que la carga el\'ectrica debe conservarse localmente.   El t\'ermino extra introducido por Maxwell est\'a motivado por un requerimiento de conservaci\'on local. 



A la luz del teorema de Noether la conservaci\'on local de la carga el\'ectrica debe provenir de una transformaci\'on continua y \emph{local} que deje invariante a la ecuaciones de Maxwell. Las invarianza gauge de la ecuaciones de Maxwell juegan este papel. Las cargas conservadas localmente pueden determinarse a partir de la din\'amica del sistema  \cite{Aitchison:2003tq}, adem\'as del uso de cargas conocidas que participen en alguna reacci\'on. Por ejemplo se puede estudiar la forma como responde una part\'\i cula de carga desconocida a campos electromagn\'eticos para determinar su carga. 

El principio gauge local, que pretendemos formular, va m\'as all\'a del teorema de Noether estableciendo una relaci\'on entre las simetr\'\i as, las leyes de conservaci\'on y la din\'amica. Este se constituye en el paradigma para estudiar las interacciones relevantes en f\'\i sica de part\'\i culas.

Para mostrar la invarianza gauge que exhiben las ecuaciones de Maxwell, es conveniente reescribirlas en forma covariante. Para ello es conveniente usar el potencial escalar el\'ectrico $\phi$ y el potencia vectorial magn\'etico $\mathbf{A}$.

% to_en:The following equations are equivalent to the two homogenous Maxwell equations
De hecho las siguientes ecuaciones \eqref{eq:phia}
\begin{align}
  \label{eq:phia}
  \mathbf{E}&=-\boldsymbol{\nabla}\phi-\frac{\partial\mathbf{A}}{\partial t}&
  \mathbf{B}&=\boldsymbol{\nabla}\times\mathbf{A}.
\end{align}
 son equivalentes a las ecuaciones homog\'eneas de Maxwell, ya que como vimos entonces y repetimos aca
\begin{align*}
  \boldsymbol{\nabla}\times\mathbf{E}&=-\boldsymbol{\nabla}\times\boldsymbol{\nabla}\phi-\frac{\partial}{\partial t}\boldsymbol{\nabla}\times\mathbf{A}\\
  &=-\frac{\partial\mathbf{B}}{\partial t},
\end{align*}
y
\begin{align*}
  \boldsymbol{\nabla}\cdot\mathbf{B}&=\boldsymbol{\nabla}\cdot(\boldsymbol{\nabla}\times\mathbf{A})\\
  &=0
\end{align*}





La ec.~(\ref{eq:phia}) es invariante bajo las siguientes transformaciones
\begin{align}
  \label{eq:phia_transf}
  \mathbf{A}&\to\mathbf{A}'=\mathbf{A}+\boldsymbol{\nabla}\chi&
  \phi&\to\phi'=\phi-\frac{\partial\chi}{\partial t} 
\end{align}
%to_en:Since
Ya que
\begin{equation}
  \label{eq:Etrans}
  \mathbf{E}\to\mathbf{E}'= -\boldsymbol{\nabla}\phi+\frac{\partial}{\partial t}\boldsymbol{\nabla}\chi
  -\frac{\partial\mathbf{A}}{\partial t}-\frac{\partial}{\partial t}\boldsymbol{\nabla}\chi=\mathbf{E}
\end{equation}
\begin{equation}
  \label{eq:btransf}
  \mathbf{B}\to\mathbf{B}'= \boldsymbol{\nabla}\times\mathbf{A}+
  \underbrace{\boldsymbol{\nabla}\times\boldsymbol{\nabla}\chi}_{\displaystyle =0}=\mathbf{B}
\end{equation}
%to_en:This imply that different observers in different space points, using different calibrations for their measures, get the same fields. 
Esto implica que diferentes observadores en diferentes puntos del espacio, usando diferentes calibraciones para sus medidas, obtienen los mismos campos. Las  ecs.~\eqref{eq:phia_transf}, corresponden a \emph{transformaciones gauge locales}

%to_en:In covariant notation
En notaci\'on de cuadrivectores, tenemos   \eqref{eq:aphicov}
\begin{align}
  A^\mu\to {A'}^\mu=&A^\mu-\partial^\mu\chi
\end{align}
%to_en: Let $U$ an element of the Transformation Group $U(1)$:
Sea  $U$ un elemento del Grupo de Transformaciones  $U(1)$:
\begin{equation}
  \label{eq:u1ele}
  U=e^{i\theta(x)}\in U(1)
\end{equation}
%to_en: The Group is defined by the infinity set of elements $U_i=e^{i\theta(x_i)}$. Then
El Grupo est\'a definido por el conjunto infinito de elementos $U_i=e^{i\theta(x_i)}$. Entonces
\begin{itemize} %noinstiki
%instiki:
\item Producto de Grupo 
  \begin{equation*}
      U_1\cdot U_2=e^{i[\theta(x_1)+\theta(x_2)]}\equiv e^{i\theta(x_3)}\in U(1)
  \end{equation*}
\item Identidad: 
  \begin{equation*}
  \theta(x)=0\qquad \text{tal que}\qquad U_I=1  
  \end{equation*}
\item Inverso 
  \begin{equation*}
      \theta(-x)=-\theta(x)\qquad \text{tal que}\qquad U^{-1}=e^{-i\theta(x)}
  \end{equation*}
\end{itemize} %noinstiki
Note que si
\begin{equation}
  \label{eq:amutransf}
  A^\mu\to{A'}^\mu=A^\mu-\frac{i}{e}(\partial^\mu U)U^{-1}
\end{equation}
%to_en:(If $\theta(x)=$cte, $ A^\mu={A^\mu}'$, phase invariance). If $\theta$ is sufficiently small
(Si $\theta=$cte, $ A^\mu={A'}^\mu$, invarianza de fase). Si $\theta$ es suficientemente peque\~no
\begin{align}
  \label{eq:Uinf}
  U=e^{i\theta(x)}&\approx1+i\theta(x)+\mathcal{O}(\theta^2)&U^{-1}=e^{-i\theta(x)}&\approx1-i\theta(x)+\mathcal{O}(\theta^2)
\end{align}
Entonces
\begin{align}
  \label{eq:checkinft}
  {A^\mu}'=&-\frac{1}{e}(i\partial^\mu\theta(8x))[1+i\theta(x)+\mathcal{O}(\theta^2)][1-i\theta(x)+\mathcal{O}(\theta^2)]
  +A^\mu[1+i\theta(x)+\mathcal{O}(\theta^2)][1-i\theta(x)+\mathcal{O}(\theta^2)]\nonumber\\
  =&A^\mu+\frac{1}{e}\partial^\mu\theta(x)+\mathcal{O}(\theta^2)
\end{align}
%to_en:Therefore, if $\chi(x)=-(1/e)\theta(x)$, then eq.~(\ref{eq:aphicov}) is the infinitesimal version of the $U(1)$ transformation in eq.~(\ref{eq:amutransf})
Por consiguiente, si $\chi(x)=-(1/e)\theta(x)$, entonces la ec.~(\ref{eq:aphicov}) es la versi\'on infinitesimal de la transformaci\'on $U(1)$  en ec.~(\ref{eq:amutransf}). Del Teorema de Noether debe existir una carga conservada corresponde a la carga el\'ectrica, asociada la corriente $J^\mu$, de la cual a\'un no hemos especificado su origen. 

%to_en:Under this transformation
Bajo esta transformaci\'on
\begin{align}
  \label{eq:fmunutrans}
  F^{\mu\nu}\to{F'}^{\mu\nu}=&(\partial^\mu{A'}^\nu-\partial^\nu{A'}^\mu)\nonumber\\
  =&\partial^\mu A^\nu-\partial^\mu\partial^\nu\chi-\partial^\nu A^\mu+\partial^\nu\partial^\mu\chi\nonumber\\
  =&\partial^\mu A^\nu-\partial^\nu A^\mu-\partial^\mu\partial^\nu\chi+\partial^\mu\partial^\nu\chi\nonumber\\
  =&F^{\mu\nu}
\end{align}
%to_en:In this way the homogeneous part of Maxwell equations corresponds to a Gauge Theory!. This was a curiosity until 1950's. 
De este modo las ecuaciones homog\'eneas de Maxwell \eqref{eq:fmunu} son invariantes gauge. Como la transformaci\'on gauge solo afecta al campo $A^\mu$, las ecuaciones inhomog\'eneas de Maxwell \eqref{eq:nohomME2} tambi\'en son invariantes gauge. 
De esta forma las ecuaciones de Maxwell corresponde a una Teor\'\i a Gauge!. Esto fue una curiosidad hasta los 1950. 




%to_en:Under this transformation
\begin{frame}[fragile,allowframebreaks]
Bajo la transformación gauge \eqref{eq:172qft}
\begin{align}
  \label{eq:fmunutrans}
  F^{\mu\nu}\to{F'}^{\mu\nu}=&(\partial^\mu{A'}^\nu-\partial^\nu{A'}^\mu)\nonumber\\
  =&\partial^\mu A^\nu-\partial^\mu\partial^\nu\chi-\partial^\nu A^\mu+\partial^\nu\partial^\mu\chi\nonumber\\
  =&\partial^\mu A^\nu-\partial^\nu A^\mu-\partial^\mu\partial^\nu\chi+\partial^\mu\partial^\nu\chi\nonumber\\
  =&F^{\mu\nu}
\end{align}

Si queremos que la Acción refleje las simetrías de las ecuaciones de
Maxwell debemos mantener sólo los términos del Lagrangiano para $A^\mu$
en \eqref{eq:lagAmu} que sean invariantes hasta una derivada total. Bajo una transformación gauge, cada
uno de los términos
\begin{equation*}
  -\frac{1}{4}G^{\mu\nu}G_{\mu\nu}+
  \frac{1}{2}m^2A^\mu A_\mu+\lambda_1\partial_\mu A^\mu A_\nu A^\nu+\lambda_2 A^\mu A_\mu A^\nu A_\nu+\lambda_3F^{\mu\nu}A_\mu A_\nu+\lambda_4G^{\mu\nu}A_\mu A_\nu
+K_\nu(x) A^\nu A_\mu A^\mu
\end{equation*}
dan lugar a un $\delta\mathcal{L}\neq\partial_\mu(\text{algo})$ y la Acción no es
invariante bajo la transformación gauge.

\end{frame}




\section{Tensor EM}


Usando el cuadrivector en ec.~(\ref{eq:cv_phia}) y
expandiendo la ec.~(\ref{eq:phia}), tenemos
\begin{align}
  E^i&=-(\frac{\partial A^0}{\partial x^{i}}+\frac{\partial A^{i}}{\partial x^0})\nonumber\\
  &=(\frac{\partial A^0}{\partial x_i}-\frac{\partial A^{i}}{\partial x_0})\nonumber\\
  &=\partial^{i}A^0-\partial^0 A^{i}\nonumber\\
  \label{eq:Efmunu}
  &=\partial^\mu A^\nu-\partial^\nu A^\mu,
  \qquad 
  \mu=i,\quad \nu=0\\
  \label{eq:E_Fi0} %noinstiki
  E^{i}&=F^{i0}
\end{align}
donde hemos definido el Tensor de intensidad electrom\'agnetica como:
\begin{equation}
  \label{eq:fmunu}
    F^{\mu\nu}=\partial^\mu A^\nu-\partial^\nu A^\mu.
\end{equation}
A $A^\mu$ se le denomina \emph{campo vectorial}. Similarmente
\begin{align}
  B^k&=\epsilon_{ijk}\frac{\partial A^j}{\partial x^{i}}\nonumber\\
  \epsilon_{lmk}B^k&=\epsilon_{lmk}\epsilon_{ijk}\frac{\partial A^j}{\partial x^{i}}\nonumber\\
  &=(\delta_{li}\delta_{mj}-\delta_{lj}\delta_{mi})\frac{\partial A^j}{\partial x^{i}}\nonumber\\
  &=\frac{\partial A^m}{\partial x^l}-\frac{\partial A^l}{\partial x^m}\nonumber\\
  &=-\frac{\partial A^m}{\partial x_l}+\frac{\partial A^l}{\partial x_m}\nonumber\\
  &=\partial^m A^l-\partial^l A^m\nonumber\\
  \label{eq:Bfmunu}
  &=\partial^\mu A^\nu-\partial^\nu A^\mu,
  \qquad 
  \mu=m,\quad \nu=l.\\
  \label{eq:BFij}
\epsilon_{lmk}B^k&=F^{ml}
\end{align}
Por consiguiente la ec.~(\ref{eq:fmunu}) es tambi\'en equivalente a las
dos ecuaciones homog\'eneas de Maxwell. En forma matricial,
\begin{align}
  F^{\mu\nu}&=
  \begin{pmatrix}
    0       &\partial^0A^1-\partial^1A^0&\partial^0A^2-\partial^2A^0&\partial^0A^3-\partial^3A^0\\
    \partial^1A^0-\partial^0A^1&0       &\partial^1A^2-\partial^2A^1&\partial^1A^3-\partial^3A^1\\
    \partial^2A^0-\partial^0A^2&\partial^2A^1-\partial^1A^2&0       &\partial^2A^3-\partial^3A^2\\
    \partial^3A^0-\partial^0A^3&\partial^3A^1-\partial^1A^3&\partial^3A^2-\partial^2A^3&0\\
  \end{pmatrix}\nonumber\\
  &=
  \begin{pmatrix}
    0 &-E^1   &-E^2   &-E^3   \\    
    E^1&0     &\epsilon_{213}B^3&\epsilon_{312}B^2\\
    E^2&\epsilon_{123}B^3&0     &\epsilon_{321}B^1\\
    E^3&\epsilon_{132}B^2&\epsilon_{231}B^1&0\\
  \end{pmatrix}\nonumber\\
  &=
\label{eq:matrixfmunu}
  \begin{pmatrix}
    0 &-E^1&-E^2&-E^3   \\    
    E^1&0  &-B^3&B^2\\
    E^2&B^3 &0  &-B^1\\
    E^3&-B^2&B^1 &0\\
  \end{pmatrix}.
\end{align}

La ec.~(\ref{eq:fmunu}) satisface la identidad,
\begin{equation}
  \label{eq:homME3}
  \partial^\lambda F^{\mu\nu}+\partial^\mu F^{\nu\lambda}+\partial^\nu F^{\lambda\mu}=0
\end{equation}
Definiendo el tensor dual como
\begin{equation*}
  \tilde{F}^{\mu\nu}=\frac{1}{2}\epsilon^{\mu\nu\rho\sigma}F_{\rho\sigma},
\end{equation*}
la ec.~(\ref{eq:homME3}) puede escribirse como
\begin{equation}
  \label{eq:homEM4}
  \partial_\mu\tilde{F}^{\mu\nu}=0.
\end{equation}

Para reescribir las ecuaciones de Maxwell inhomog\'enas en forma
covariante usaremos adem\'as el cuadrivector $J^\mu$ de la
ec.~\eqref{eq:cv_jmu}.  Usando la ec.~\eqref{eq:Efmunu}, la primera
ecuaci\'on de Maxwell inhomog\'enea~\eqref{eq:inhom_m_eq} puede escribirse
como
\begin{align}
  \label{nohomME21}
    \frac{\partial E^{i}}{\partial x^{i}}&=J^0\nonumber\\
    \frac{\partial}{\partial x^{i}}F^{i0}&=J^0\nonumber\\
    \partial_iF^{i0}&=J^0\nonumber\\
    \partial_\mu F^{\mu0}&=J^0\,.
\end{align}
Usando las ecs.~\eqref{eq:Efmunu}, \eqref{eq:Bfmunu}, la segunda
ecuaci\'on de Maxwell inhomog\'enea~\eqref{eq:inhom_m_eq} puede escribirse
como
\begin{align}
  \epsilon_{ijk}\frac{\partial B^j}{\partial x^{i}}-\frac{\partial E^k}{\partial t}&=J^k\nonumber\\
  -\frac{\partial (\epsilon_{ikj}B^j)}{\partial x^{i}}-\frac{\partial E^k}{\partial t}&=J^k\nonumber\\
-\partial_iF^{ki}-\partial_0F^{k0}&=J^k\nonumber\\
\partial_iF^{ik}+\partial_0F^{0k}&=J^k\nonumber\\
\label{nohomME22}
\partial_\mu F^{\mu k}&=J^k
\end{align}

Las ecuaciones \eqref{nohomME21},
\eqref{nohomME22} pueden escribirse en forma compacta como
\begin{align}
\label{eq:nohomME2}
  \partial_\mu F^{\mu\nu}&=J^\nu\\
&=
\begin{cases}
  \partial_\mu(\partial^\mu A^0-\partial^0A^{i})=J^0&\text{para $\nu=0$}\\
  \partial_\mu(\partial^\mu A^{i}-\partial^{i}A^\mu)=J^{i}&\text{para $\nu=i$}
\end{cases}\nonumber
\end{align}


Los resultados sobre la notaci\'on covariante de la ecuaciones de Maxwell est\'an resumidos en la Tabla~\ref{tab:eqmax} %noinstiki
\begin{table} %noinstiki
  \begin{center} %noinstiki
  \begin{tabular}{l|l|l|l} %noinstiki
            &$\mathbf{E}$, $\mathbf{B}$&$A^\mu$            &$F^{\mu\nu}$\\\hline %noinstiki
Homog\'eneas  &Ec.~(\ref{eq:hom_m_eq})   &~(\ref{eq:phia})&~(\ref{eq:fmunu}) \'o (\ref{eq:homME3}) \'o (\ref{eq:homEM4})\\\hline %noinstiki
Inhomog\'eneas& (\ref{eq:inhom_m_eq})    &                &~\eqref{eq:nohomME2} \\ %noinstiki
  \end{tabular} %noinstiki
  \end{center} %noinstiki
  \caption{Ecuaciones de Maxwell} %noinstiki
  \label{tab:eqmax} %noinstiki
\end{table} %noinstiki
De la parte izquierda de la ecuaci\'on \eqref{eq:nohomME2}, podemos ver que
\begin{align*}
  \partial_\nu\partial_\mu F^{\mu\nu}&=0
\end{align*}
Por consiguiente, la cuadricorriente $J^\mu$ es conservada:
\begin{equation}
  \label{eq:consvjmu}
  \partial_\mu J^\mu=0
\end{equation}


\subsection{Lagrangiano Electromagn\'etico}
\label{sec:lagr-electr}
%to_en:Eq.~(\ref{eq:phia}) is invariant under the following transformations

Para ilustrar la relaci\'on entre la conservaci\'on local de la carga el\'ectrica y la transformaci\'on gauge, que no es una conexi\'on simple, considere las ecuaciones de Maxwell
\begin{equation}
    \partial_\mu F^{\mu\nu}=J^\nu,
\end{equation}
que autom\'aticamente incluyen la conservaci\'on local de carga, expresada por la ecuaci\'on de continuidad
\begin{equation}
  \partial_\mu J^\mu=0.  
\end{equation}
Adem\'as las ecuaciones de Maxwell permanecen invariantes bajo la transformaci\'on gauge
\begin{equation}
   A^\mu\to {A'}^\mu=A^\mu-\partial^\mu\chi,
\end{equation}
ya que dicha transformaci\'on deja invariante a $F^{\mu\nu}$. De aqu\'\i{} la sugerencia de la invarianza gauge esta relacionada de alguna manera a la conservaci\'on de la carga. De hecho la acci\'on m\'as general posible para el campo $A^\mu$ compatible tanto con la invarianza de Lorentz y la invarianza gauge local corresponde a la acci\'on que da lugar a las ecuaciones de Maxwell.

La  definición of $F^{\mu\nu}$ incluye de entrada las ecuaciones
homogéneas de Maxwell. Para ver esto, note en primer lugar que las
únicas componentes diferentes de  cero son
\begin{align}
  F^{\mu\nu}=  \begin{cases}
    F^{\mu0}=F^{i0} & \text{$\nu=0$}\\
    F^{\mu l}=F^{ml} & \text{$\nu=l$}\\
  \end{cases}
\end{align}
Para $\nu=0$ tenemos
\begin{align}
F^{i0}  &=\partial^{i}A^0-\partial^0 A^{i}\nonumber\\
  &=(\frac{\partial A^0}{\partial x_i}-\frac{\partial A^{i}}{\partial x_0})\nonumber\\
&=-(\frac{\partial A^0}{\partial x^{i}}+\frac{\partial A^{i}}{\partial x^0})\nonumber\\
&=E^i
\end{align}
donde
\begin{align}
\label{eq:173qft}
   \mathbf{E}&=-\boldsymbol{\nabla}\phi-\frac{\partial\mathbf{A}}{\partial t}\,.
\end{align}
mientras que para $\nu=l$ tenemos
\begin{align}
F^{ml}  &=\partial^m A^l-\partial^l A^m\nonumber\\
  &=(\delta_{lj}\delta_{mi}-\delta_{li}\delta_{mj}){\partial^iA^j}\nonumber\\
  &=-(\delta_{lj}\delta_{mi}-\delta_{li}\delta_{mj}){\partial_iA^j}\nonumber\\
  &=(\delta_{li}\delta_{mj}-\delta_{lj}\delta_{mi}){\partial_iA^j}\nonumber\\
  &=(\delta_{li}\delta_{mj}-\delta_{lj}\delta_{mi})\frac{\partial A^j}{\partial x^{i}}\nonumber\\
    &=\epsilon_{lmk}\epsilon_{ijk}\frac{\partial A^j}{\partial x^{i}}\nonumber\\
&=\epsilon_{lmk}\left(\boldsymbol{\nabla}\times  \mathbf{A}\right)^k\nonumber\\
&=\epsilon_{lmk}B^k\,,
\end{align}
donde
\begin{align}
\label{eq:174qft}
  \mathbf{B}&=\boldsymbol{\nabla}\times \mathbf{A}\,.
\end{align}
Entonces tenemos
\begin{align}
  \{F^{\mu\nu}\}  &=
  \begin{pmatrix}
    0 &-E^1   &-E^2   &-E^3   \\    
    E^1&0     &\epsilon_{213}B^3&\epsilon_{312}B^2\\
    E^2&\epsilon_{123}B^3&0     &\epsilon_{321}B^1\\
    E^3&\epsilon_{132}B^2&\epsilon_{231}B^1&0\\
  \end{pmatrix}\nonumber\\
  &=
\label{eq:matrixfmunu}
  \begin{pmatrix}
    0 &-E^1&-E^2&-E^3   \\    
    E^1&0  &-B^3&B^2\\
    E^2&B^3 &0  &-B^1\\
    E^3&-B^2&B^1 &0\\
  \end{pmatrix}.
\end{align}


De las ecuaciones.~\eqref{eq:173qft}, y \eqref{eq:174qft}
\begin{align*}
  \boldsymbol{\nabla}\times \mathbf{E}&=-\boldsymbol{\nabla}\times \boldsymbol{\nabla}\phi-\frac{\partial}{\partial t}\boldsymbol{\nabla}\times \mathbf{A}\\
  &=-\frac{\partial\mathbf{B}}{\partial t},
\end{align*}
y
\begin{align*}
  \boldsymbol{\nabla}\cdot\mathbf{B}&=\boldsymbol{\nabla}\cdot(\boldsymbol{\nabla}\times \mathbf{A})\\
  &=0
\end{align*}
que son justamente las ecuaciones homogéneas de  Maxwell. Por
consiguiente la expresión  
 \begin{equation}
  \label{eq:fmunu}
    F^{\mu\nu}=\partial^\mu A^\nu-\partial^\nu A^\mu.
\end{equation}
con los $\{F^{\mu\nu}\}$ dada en \eqref{eq:matrixfmunu}, no es más que
una forma equivalente de las ecuaciones homogéneas de Maxwell. Las
ecuaciones de Maxwell restante se pueden obtener a partir de las
ecuaciones de Euler-Lagrange para $A^\nu$:



 
\begin{frame}[fragile,allowframebreaks]
Tomando la derivada con respecto a $\nu$ en ambos lados tenemos
\begin{align}
   \partial_\nu\partial_\mu F^{\mu\nu}&=\partial_\nu J^\nu.
\end{align}
De la parte izquierda de ésta ecuación tenemos
\begin{align*}
  \partial_\nu\partial_\mu F^{\mu\nu}&=\tfrac{1}{2}\left(\partial_\nu\partial_\mu F^{\mu\nu}+\partial_\nu\partial_\mu F^{\mu\nu}  \right)\nonumber\\
&=\tfrac{1}{2}\left(\partial_\nu\partial_\mu F^{\mu\nu}+\partial_\mu\partial_\nu F^{\nu\mu}  \right)
&&\text{intercambiando índices mudos}\nonumber\\
&=\tfrac{1}{2}\left(\partial_\nu\partial_\mu F^{\mu\nu}+\partial_\nu\partial_\mu F^{\nu\mu}  \right)
&&\text{conmutando derivadas}\nonumber\\
&=\tfrac{1}{2}\left(\partial_\nu\partial_\mu F^{\mu\nu}-\partial_\nu\partial_\mu F^{\mu\nu}  \right)
&&\text{usando antisimetría de $F^{\mu\nu}$}\nonumber\\
&=0\,,
\end{align*}
Por consiguiente, la cuadricorriente $J^\mu$ es conservada:
\begin{equation}
  \label{eq:consvjmu}
  \partial_\mu J^\mu=0\,.
\end{equation}


De nuevo, para $\nu=0$, tenemos
\begin{align}
  \label{nohomME21}
    \partial_\mu F^{\mu0}&=J^0\nonumber\\
    \partial_iF^{i0}&=J^0\nonumber\\
    \frac{\partial}{\partial x^{i}}F^{i0}&=J^0\nonumber\\
    \frac{\partial E^{i}}{\partial x^{i}}&=J^0\,,
\end{align}
y por consiguiente
\begin{align}
   \boldsymbol{\nabla}\cdot\mathbf{E}&=\rho\,.
\end{align}
mientras que para $\nu=k$ tenemos
\begin{align}
\partial_\mu F^{\mu k}&=J^k\nonumber\\
\partial_iF^{ik}+\partial_0F^{0k}&=J^k\nonumber\\
-\partial_iF^{ki}-\partial_0F^{k0}&=J^k\nonumber\\
  -\frac{\partial (\epsilon_{ikj}B^j)}{\partial x^{i}}-\frac{\partial E^k}{\partial t}&=J^k\nonumber\\
\epsilon_{ijk}\frac{\partial B^j}{\partial x^{i}}-\frac{\partial E^k}{\partial t}&=J^k\nonumber\\
(\boldsymbol{\nabla}\times \mathbf{B})^k-\frac{\partial E^k}{\partial t}&=J^k.\,.
\end{align}
y por consiguiente
\begin{align}
  \boldsymbol{\nabla}\times \mathbf{B}-\frac{\partial\mathbf{E}}{\partial t}&=\mathbf{J}\,.
\end{align}
De esta forma, la expresión
\begin{align}
   \partial_\mu F^{\mu\nu}&=J^\nu& \text{where}\quad    F^{\mu\nu}=\partial^\mu A^\nu-\partial^\nu A^\mu\,,
\end{align}
es completamente equivalente al conjunto completo de ecuaciones de Maxwell:
\begin{align}
  \label{eq:hom_m_eq}
  \boldsymbol{\nabla}\cdot\mathbf{B}&=0,&\boldsymbol{\nabla}\times \mathbf{E}+\frac{\partial\mathbf{B}}{\partial t}&=0\\
  \label{eq:inhom_m_eq}
  \boldsymbol{\nabla}\cdot\mathbf{E}&=\rho,&\boldsymbol{\nabla}\times \mathbf{B}-\frac{\partial\mathbf{E}}{\partial t}&=\mathbf{J}\,.
\end{align}

\end{frame}
%%%%%%%%%%%%%%%%%%%%%%%%%%%




De este modo las ecuaciones de Maxwell se pueden derivar del requerimiento de que la teor\'\i a, adem\'as de ser invariante de Lorentz, pueda expresarse en t\'erminos de potenciales de una forma que sea invariante gauge en esos potenciales. Si un cuadrivector potencial $A^\mu$ es postulado, y se impone que la teor\'\i a involucre este solamente, de una forma que sea insensible a a cambios de la forma \eqref{eq:aphicov}, se es conducido naturalmente a la idea de que los campos f\'\i sicos entran \'unicamente v\'\i a la cantidad $F^{\mu\nu}$, que es invariante bajo la ec. \eqref{eq:aphicov}. De aqu\'\i{} se puede conjeturar la ecuaci\'on de campo en base a la covarianza de Lorentz. 

Esto no corresponde ciertamente a una prueba de las ecuaciones de Maxwell. A pesar de eso, la idea que la din\'amica (en este caso la completa interconexi\'on entre los efectos el\'ectricos y magn\'eticos) pueda estar \'\i ntimamente relacionada a un requerimiento de invarianza local se ha convertido en algo muy fruct\'\i fero. 

En t\'erminos de transformaciones globales, se puede mostrar \cite{Aitchison:2003tq} que el cambio por una constante del potencial escalar ($\chi=at$, $A_0\to A_0'=A_0-a$, $\mathbf{A}\to \mathbf{A}'=\mathbf{A}$ en ec.\eqref{eq:phia_transf}), m\'as la conservaci\'on de energ\'\i a, da lugar a la conservaci\'on local de la carga. La conservaci\'on local en este contexto requiere que el cambio por una funci\'on del potencial escalar en en \eqref{eq:phia_transf} sea compensado por el correspondiente cambio en el vector potencial magn\'etico $A$. En general, cuando una cierta invarianza global es generalizada a una local, se requiere la existencia de un nuevo campo que compensa, interactuando de una manera espec\'\i fica. La teor\'\i as que dan lugar al Modelo Est\'andar y que describen las interacciones fuertes, d\'ebiles y electromagn\'eticas, son ejemplos de teor\'\i as din\'amicas derivadas desde un requerimiento de invarianza local.

Una de las principales razones
de porque la f\'\i sica de part\'\i culas se formula en t\'erminos de
lagrangianos, es que $\mathcal{L}$ debe ser escalar en cada espacio
relevante, e invariante bajo las transformaciones (hasta derivadas
totales), ya que la acci\'on es invariante. Haciendo el Lagrangiano
covariante de Lorentz por ejemplo, garantiza que todas las
predicciones son invariantes de Lorentz.



\subsection{Energ\'\i a del campo electromagn\'etico}
\begin{frame}[fragile,allowframebreaks]


Necesitamos la expresi\'on para $F_{\mu\nu}$,
\begin{equation}
  \label{eq:16}
  F_{\mu\nu}=g_{\mu\rho}g_{\nu\eta}F^{\rho\eta}\Rightarrow
  \begin{cases}
    F_{0i}=F_{0\nu}=g_{00}g_{ij}F^{0j}=-F^{0i} &\text{para $\mu=0$}\\
    F_{ij}=F_{i\nu}=g_{ik}g_{jl}F^{kl}=F^{ij} &\text{para $\mu=i$}
  \end{cases}
\end{equation}
De la ec.~\eqref{eq:tmunu}, se tiene
\begin{align}
  T^\mu_\nu&=\frac{\partial\mathcal{L}}{\partial(\partial_\mu A_\lambda)}(\partial_\nu A_\lambda)
  -\delta^\mu_\nu\mathcal{L}\nonumber\\
  &=-F^{\mu\lambda}(\partial_\nu A_\lambda)-\delta^\mu_\nu\mathcal{L}
\end{align}
La energ\'\i a del campo, corresponde a la componente $T^0_0$:
\begin{align}
\label{eq:17}
  T^0_0&=-F^{0\lambda}(\partial_0A_\lambda)-\mathcal{L}\nonumber\\
  &=-F^{0\lambda}(\partial_0A_\lambda)+\frac{1}{4}F^{\mu\nu}F_{\mu\nu}+J^\mu A_\mu\nonumber
\end{align}
Usando las ecuaciones 
\eqref{eq:E_Fi0}, %noinstiki\eqref{eq:Efmunu},
\eqref{eq:BFij}, \eqref{eq:16}

\begin{align}
T^0_0 &=-F^{0\lambda}(\partial_0A_\lambda)+\frac{1}{4}F^{\mu\nu}F_{\mu\nu}+J^\mu A_\mu\nonumber\\
  &=-F^{0\mu}(\partial_0A_\mu)+\frac{1}{4}\overbrace{F^{\mu0}F_{\mu0}}^{\nu=0}+\frac{1}{4}\overbrace{F^{\mu i}F_{\mu i}}^{\nu=i}+J^\mu A_\mu\nonumber\\
  &=-F^{0\mu}\partial_\mu A_0-F^{\mu0}F_{\mu0}+\frac{1}{4}F^{\mu0}F_{\mu0}+\frac{1}{4}F^{\mu i}F_{\mu i}+J^\mu A_\mu\,.
\end{align}
Tenemos dos partes
\begin{align}
  -F^{\mu0}F_{\mu0}+\frac{1}{4}F^{\mu0}F_{\mu0}+\frac{1}{4}F^{\mu i}F_{\mu i}
  &=-F^{i0}F_{i0}+\frac{1}{4}F^{i0}F_{i0}+\frac{1}{4}\overbrace{F^{0i}F_{0i}}^{\mu=0}+\frac{1}{4}\overbrace{F^{ji}F_{ji}}^{\mu=j}\nonumber\\
  &=-F^{i0}F_{i0}+\frac{1}{4}F^{i0}F_{i0}+\frac{1}{4}{F^{i0}F_{i0}}+\frac{1}{4}{F^{ji}F_{ji}}\nonumber\\
  &=-\frac{1}{2}{F^{i0}F_{i0}}+\frac{1}{4}{F^{ji}F_{ji}}\,.
\end{align}
Adem\'as
\begin{align}
  -F^{0\mu}\partial_\mu A_0+J^\mu A_\mu=&-\partial_\mu(A_0 F^{0\mu})+A_0\partial_\mu F^{0\mu}+J^\mu A_\mu\nonumber\\
  =&-\partial_\mu(A_0 F^{0\mu})-A_0\partial_\mu F^{\mu0}+J^\mu A_\mu\nonumber\\
  =&-\partial_\mu(A_0 F^{0\mu})-A_0 J^0+J^\mu A_\mu\nonumber\\
  =&-\partial_i(A_0 F^{0i})-\mathbf{J}\cdot\mathbf{A}\,.
\end{align}
Entonces
\begin{align}
  T^0_0&=-\partial_i(A_0F^{0i})-\frac{1}{2}F^{i0}F_{i0}+\frac{1}{4}F^{ji}F_{ji}-\mathbf{J}\cdot\mathbf{A}\nonumber\\
 &=-\partial_i(A_0F^{0i})+\frac{1}{2}F^{i0}F^{i0}+\frac{1}{4}F^{ji}F^{ji}-\mathbf{J}\cdot\mathbf{A},\qquad\text{suma tambi\'en sobre $i,j$}\nonumber\\
  &=\frac{1}{2}E^{i}E^{i}+\frac{1}{4}\epsilon_{ijk}B^k\epsilon_{ijl}B^l+\partial_i(A_0E^{i})-\mathbf{J}\cdot\mathbf{A},\qquad\text{suma tambi\'en sobre $i,j$}\nonumber\\
  &=\frac{1}{2}\mathbf{E}^2+\frac{1}{2}\delta_{kl}B^k B^l+\boldsymbol{\nabla}\cdot(A^0\mathbf{E})-\mathbf{J}\cdot\mathbf{A}\nonumber\\
  &=\frac{1}{2}\mathbf{E}^2+\frac{1}{2}\mathbf{B}^2+\boldsymbol{\nabla}\cdot(A^0\mathbf{E})-\mathbf{J}\cdot\mathbf{A}
\end{align}

Entonces, en ausencia de corrientes
\begin{align}
  \mathcal{H}=&\frac{1}{2}\mathbf{E}^2+\frac{1}{2}\mathbf{B}^2+\boldsymbol{\nabla}\cdot(A^0\mathbf{E})\,.
\end{align}
Similarmente la densidad Lagrangiano puede escribirse como
\begin{align}
   \mathcal{L}=-\frac{1}{4}F^{\mu\nu}F_{\mu\nu}=\frac{1}{2}\left(\mathbf{E}^2-\mathbf{B}^2\right)
\end{align}
En vista a la ec.~\eqref{eq:17}, ya que la densidad Lagrangiana est\'a definida hasta una derivada total, como $\boldsymbol{\nabla}\cdot(A^0\mathbf{E})=\partial_\mu(A_0F^{\mu0})$, la densidad Hamiltoniana tambi\'en estar\'a definida hasta una derivada total. De hecho,
el Hamiltoniano es 
\begin{align}
  H&=\frac{1}{2}\int_Vd^3x\,(\mathbf{E}^2+\mathbf{B}^2)+ \int_Vd^3x\,\boldsymbol{\nabla}\cdot(A^0\mathbf{E})\nonumber\\
  \label{eq:18}
  &=\frac{1}{2}\int_Vd^3x\,(\mathbf{E}^2+\mathbf{B}^2),
\end{align}
y corresponde a la expresi\'on conocida para la energ\'\i a del campo electromagn\'etico. Hemos usado el hecho que en ausencia de corrientes todo lo que entra a un volum\'en debe salir y por consiguiente las integrales sobre el volumen de la divergencia de cualquier vector es cero.

Similarmente el momentum total del
campo, en ausencia de corrientes, corresponde al vector de Pointing:
%examen
\begin{align}
  T^0_i=&\frac{\partial\mathcal{L}}{\partial(\partial_0 A_\nu)}\partial_i A_\nu\nonumber\\
  =&-F^{0\nu}\partial_i A_\nu\nonumber\\
  =&-F^{0j}(\partial_i A_j-\partial_j A_i)-F^{0j}\partial_j A_i\nonumber\\
  =&-F^{0j}F_{ij}-F^{0j}\partial_j A_i\nonumber\\
  =&-F^{0j}F^{ij}-\partial_j (F^{0j}A_i)+(\partial_jF^{0j}) A_i\nonumber\\
  =&E^{j}\epsilon_{jik}B^k+\partial_j (E^jA_i)+(J^0) A_i\nonumber\\
  =&-(\mathbf{E}\times\mathbf{B})^i-\boldsymbol{\nabla}\cdot(A^i\mathbf{E})-\rho A^i\,
\end{align}
En ausencia de cargas y corrientes

\begin{align}
 P^i=-\int_Vd^3x\,T_{i}^0&=\int_Vd^3x\,(\mathbf{E}\times\mathbf{B})^i+\int_Vd^3x\,\boldsymbol{\nabla}\cdot(A^i\mathbf{E})\nonumber\\
 \label{eq:19}
 \mathbf{P}&=\int_Vd^3x\,(\mathbf{E}\times\mathbf{B})\,.
\end{align}
\end{frame}
\subsection{Fijaci\'on del gauge}
\label{sec:fijacion-del-gauge}
%ver secci\'on 4.4 de Cottingan
\begin{frame}[fragile,allowframebreaks]
Para obtener una soluci\'on definitiva a las ecuaciones del campo
electromagn\'etico, se debe remover la arbitrariedad asociada con la
libertad gauge de la ec.~\eqref{eq:aphicov}. De este modo los campos
quedan especificados un\'\i vocamente en todas partes. De hecho, de las
cuatro componentes del campo $A^\mu$, solo dos son independientes y
corresponden a los estados de polarizaci\'on de las ondas
electromagn\'eticas~\cite{Gross} (Cap\'\i tulo 2). A \'este proceso se le
denomina fijar el gauge, y consiste en imponer restricciones sobre los
campos que fijan la funci\'on  $\chi$ y remueven la libertad gauge. 

Nosotros usaremos el Gauge de Lorentz, definido por la condici\'on
\begin{equation}
  \label{eq:20}
  \partial_\mu A^\mu=0
\end{equation}
Si inicialmente $\partial_\mu A^\mu\neq0$, se realiza una transformaci\'on gauge tal
que $\partial_\mu A'^\mu=0$. De acuerdo a la ec.~\eqref{eq:aphicov}, esto da lugar
a la ecuaci\'on de onda inhomog\'enea
\begin{equation*}
\Box\chi=\partial_\mu A^\mu   
\end{equation*}
que puede solucionarse mediante las t\'ecnicas usuales. 

Es importante resaltar que la f\'\i sica queda inafectada por la escogencia
del gauge. El resultado final para cualquier observable f\'\i sico debe
ser independiente del gauge usado para calcularlo.

Las ecuaciones de Maxwell \eqref{eq:nohomME2} pueden escribirse como
\begin{align}
   \partial_\mu F^{\mu\nu}&=J^\nu\nonumber\\
  \partial_\mu(\partial^\mu A^\nu-\partial^\nu A^\mu)&=J^\nu\nonumber\\
  \partial_\mu\partial^\mu A^\nu-\partial_\mu\partial^\nu A^\mu&=J^\nu\nonumber\\
\label{eq:21}
  \partial_\mu\partial^\mu A^\nu-\partial^\nu(\partial_\mu A^\mu)&=J^\nu.
 \end{align}

Apliquemos ahora el gauge de Lorentz, ec.~(\ref{eq:20}) a las
ecuaciones inhomog\'eneas de Maxwell \eqref{eq:21}
\begin{equation}
  \label{eq:22} 
  \Box A^\nu=\partial_\mu\partial^\mu A^\nu=J^\nu.
\end{equation}
De este modo, cada componente del campo $A^\mu$ satisface la ecuaci\'on de
onda (\ref{eq:waveec}), o la ecuaci\'on de Klein-Gordon (\ref{eq:kg})
para masa cero. En ausencia de corrientes el campo $A^\mu$ puede ser
expandido en ondas planas con dos grados independientes de
polarizaci\'on~\cite{Gross}, de forma similar a como se hizo en la
secci\'on~\ref{sec:aplicacion-la-cuerda} para el campo $\phi$. Una vez
cuantizada la teor\'\i a, $A^\mu$ corresponde al fot\'on, y solo queda con dos
grados de libertad independientes que corresponden a los modos
transversales de la onda electromagn\'etica~\cite{Gross} (cap\'\i tulo 2).

La ec.~(\ref{eq:lagAmum}), with $J^\mu=0$, en el Gauge de Lorentz puede escribirse como
\begin{align}
  \label{eq:laglorgauge}
  \mathcal{L}=&-\frac{1}{4}F^{\mu\nu}F_{\mu\nu}\nonumber\\
  =&-\frac{1}{4}(\partial^\mu A^\nu-\partial^\nu A^\mu)(\partial_\mu A_\nu-\partial_\nu A_\mu)\nonumber\\
  =&-\frac{1}{4}(\partial^\mu A^\nu\partial_\mu A_\nu-\partial^\mu A^\nu\partial_\nu A_\mu-\partial^\nu A^\mu\partial_\mu A_\nu+\partial^\nu A^\mu\partial_\nu A_\mu)\nonumber\\
  =&-\frac{1}{4}[\partial^\mu A^\nu\partial_\mu A_\nu-\partial^\mu(A^\nu\partial_\nu A_\mu)+A^\nu\partial_\nu(\partial^\mu A_\mu)-\partial^\nu(A^\mu\partial_\mu A_\nu)+A^\mu\partial_\mu(\partial^\nu A_\nu)+\underbrace{\partial^\nu A^\mu\partial_\nu A_\mu}_{\mu\leftrightarrow\nu}]\nonumber\\
  =&-\frac{1}{4}[2\partial^\mu A^\nu\partial_\mu A_\nu-\partial^\mu(2A^\nu\partial_\nu A_\mu)]\nonumber\\
  =&-\frac{1}{2}\partial^\mu A^\nu\partial_\mu A_\nu
\end{align}
Incluyendo el t\'ermino con corrientes, y usando el hecho de que un signo global no afecta las ecuaciones de movimiento, tenemos
%to_en:Including the term with currents, and using the fact that a global sign will not affect the motion equations, we have
\begin{align}
  \label{eq:laglorgauugefin}
  \mathcal{L}=&\frac{1}{2}\partial^\mu A^\nu\partial_\mu A_\nu+J_\mu A^\mu
\end{align}
\end{frame}
\section{Ecuaciones de Proca}
\label{sec:ecuacion-de-proca}
\begin{frame}[fragile,allowframebreaks]


Consideraremos ahora el efecto de adicionar un t\'ermino de masa a la teor\'\i a de
Maxwell. Los campos vectoriales masivos juegan un papel importante en
f\'\i sica. Campos como $W^\mu$, $Z^\mu$ que median las interacciones d\'ebiles
son ejemplos de campos de este tipo. Las implicaciones de una masa
finita para el fot\'on pueden inferirse de un conjunto de postulados que
hacen de las ecuaciones de Proca la \'unica generalizaci\'on posible de
las ecuaciones de Maxwell \cite{Goldhaber:1971mr}. 

Teniendo en cuenta s\'olo el t\'ermino de masa en la ec.~(\ref{eq:lagAmum})
\begin{equation}
  \label{eq:23}
  \mathcal{L}=-\frac{1}{4}F^{\mu\nu}F_{\mu\nu}+\frac{1}{2}m^2A^\mu A_\mu-J^\mu A_\mu.
\end{equation}
Usando las ecuaciones de Euler-Lagrange, tenemos
\begin{align}
-\frac{1}{4}\partial_\mu
  \left[
\frac{\partial}{\partial(\partial_\mu A_\nu)}F^{\rho\eta}F_{\rho\eta}
  \right]-
\frac{\partial}{\partial A_\nu}
\left(
\frac{1}{2}m^2A^\rho A_\rho-J^\rho A_\rho
\right)&=0\nonumber\\
\label{eq:24}
\partial_\mu F^{\mu\nu}+m^2A^\nu&=J^\nu.
\end{align}
Tomando la cuadridivergencia a ambos lados de la ecuaci\'on y usando la
ec.~(\ref{eq:21}), tenemos
\begin{align}
 &\partial_\nu\partial_\mu\partial^\mu A^\nu-\partial_\nu\partial^\nu\partial_\mu A^\mu+m^2\partial_\nu A^\nu=\partial_\nu J^\nu\nonumber\\
 &\partial_\nu\partial_\mu\partial^\mu A^\nu-\partial_\mu\partial^\mu\partial_\nu A^\nu+m^2\partial_\nu A^\nu=\partial_\nu J^\nu\nonumber\\
\label{eq:25}
 &m^2\partial_\nu A^\nu=\partial_\nu J^\nu
\end{align}
De este modo, en ausencia de corrientes, la ecuaciones de Proca dan
lugar a la condici\'on de Lorentz. De otro lado, si asumimos que la
corriente se conserva, la condici\'on de Lorentz tambi\'en aparece. Por
consiguiente, si la masa de campo vectorial es diferente de cero, la
condici\'on de Lorentz, ec.~(\ref{eq:20}), emerge como una restricci\'on
adicional que debe ser siempre tomada en cuenta. De este modo la
libertad gauge de las ecuaciones de Maxwell se pierde completamente en
la ecuaciones de Proca, que sin perdida de generalidad se pueden
reescribir, usando la condición de Proca
\begin{align}
\label{eq:procacon}
\partial_\mu A^\mu=0\,,
\end{align}
 y las ecs.~(\ref{eq:21}),~(\ref{eq:24}),  como:
\begin{equation}
  \label{eq:26}
   (\Box+m^2)A^\nu=J^\nu
\end{equation}
En ausencia de corrientes, cada una de las componentes del campo
vectorial satisface la ecuaci\'on de Klein-Gordon~(\ref{eq:kg}). Por
consiguiente $m$ corresponde a la masa del campo vectorial
$A^\mu$. 

Aplicando la condici\'on de Lorentz a la ec.~(\ref{eq:23}), obtenemos el
Lagrangiano de la Ecuaci\'on de Proca (\ref{eq:26})
\begin{equation}
  \label{eq:27}
  \mathcal{L}=-\frac{1}{2}\partial_\mu A^\nu\partial^\mu A_\nu+\frac{1}{2} m^2A^\nu A_\nu-J^\nu A_\nu,
\end{equation}
donde hemos reabsorbido un signo global que no afecta las ecuaciones
de movimiento. El primer t\'ermino que incluye s\'olo derivadas de los
campos es llamado \emph{t\'ermino cin\'etico} y dependen s\'olo del esp\'\i n de
las part\'\i culas. El t\'ermino cuadr\'atico en
los campos corresponde al \emph{t\'ermino de masa}, y el \'ultimo
corresponde a la interacci\'on del campo con una corriente. Cuando un
Lagrangiano contiene s\'olo t\'erminos cin\'eticos y de masa diremos que el
campo que da lugar al Lagrangiano es libre de interacciones, o
simplemente que es un \emph{campo libre}. Las otras partes del
Lagrangiano ser\'an llamadas \emph{Lagrangiano de Interacci\'on}. De este
modo podemos reescribir el Lagrangiano \eqref{eq:27} como
\begin{equation*}
\mathcal{L}=\mathcal{L}_{\text{free}}+\mathcal{L}_{\text{int}},  
\end{equation*}
donde,
\begin{align}
\mathcal{L}_{\text{free}}&=-\frac{1}{2}\partial_\mu A^\nu\partial^\mu A_\nu+\frac{1}{2} m^2A^\nu A_\nu\nonumber\\
\label{eq:28}
\mathcal{L}_{\text{int}}&=-J^\nu A_\nu.
\end{align}

Debido a que la teor\'\i a masiva ya no es invariante gauge, la condici\'on
de Lorentz aparece autom\'aticamente como la \'unica restricci\'on apropiada
sobre el campo vectorial.

Una vez se toma en cuenta la condici\'on de Lorentz el campo masivo
libre puede expandirse en ondas planas con tres grados de libertad
independientes de polarizaci\'on. Dos de estos corresponden a los dos
estados transversos que aparecen en las ondas electromagn\'eticas
($A^1$, $A^2$), y el tercero ($A^3$) corresponde a un estado
longitudinal en la direcci\'on del momento de la part\'\i cula \cite{Gross}.

Aunque hemos hecho el an\'alisis de la ecuaci\'on de Proca permitiendo un
t\'ermino de masa para el fot\'on, las implicaciones experimentales de una
teor\'\i a de este tipo dan lugar a restricciones muy fuertes sobre la
masa del fot\'on\cite{Goldhaber:1971mr}. El l\'\i mite actual sobre la masa
del fot\'on es $m\lt 6\times10^{-17}\,$eV ($1.1\times10^{-52}\,$Kg)
\cite{Yao:2006px}. Debido al principio gauge local, desde el punto
te\'orico se espera que la masa del fot\'on sea exactamente cero. En
general, los campos vectoriales puede ser generados a partir de otras
cargas no electromagn\'eticas y pueden ser masivos. El reto durante
varias d\'ecadas fue entender como las masa de los campos vectoriales de
la interacci\'on d\'ebil podr\'\i a hacerse compatible con el principio gauge
local.  

Además, el Lagrangiano libre se puede reescribir como la suma de cuatro lagrangianos independientes para cada campo $\phi=A^0$, $A^i$:
\begin{align}
  \mathcal{L}_{\text{free}}&=-\frac{1}{2}\partial_\mu A^\nu\partial^\mu A_\nu+\frac{1}{2} m^2A^\nu A_\nu\nonumber\\
  &=-\frac{1}{2}\partial_\mu \phi\partial^\mu \phi+\frac{1}{2} m^2\phi^2
   +\frac{1}{2}\partial_\mu \boldsymbol{A}\cdot \partial^\mu \boldsymbol{A}-\frac{1}{2} m^2\boldsymbol{A}\cdot \boldsymbol{A}\,.
\end{align}

\end{frame}










\begin{subappendices}


\section{Lagrangiano para el campo vectorial}
\begin{frame}[fragile,allowframebreaks]
Ya estamos en capacidad de responder la siguiente pregunta: ¿Cual es el Lagrangiano más general posible para el campo de cuatro componentes $A^{\mu}(x)$ compatible con la invarianza de Lorentz 

Las correspondientes transformaciones son

\begin{align}
  A^\mu(x)\to {A'}^\mu(x')={\Lambda^\mu}_\nu A^\nu(\Lambda^{-1}x)\,.
\end{align}

Además, tenemos la ec.\eqref{eq:aphicov}:
\begin{align}
\label{eq:172qft}
  A^\mu\to{A'}^\mu=A^\mu-\partial^\mu\chi(x)\;.
\end{align}




Definiendo
\begin{align*}
  F^{\mu\nu}&=\partial^\mu A^\nu-\partial^\nu A^\mu\\
  G^{\mu\nu}&=\partial^\mu A^\nu+\partial^\nu A^\mu\\
\end{align*}
El Lagrangiano que da lugar a una Acción invariante de Lorentz para el cuadrivector $A^\mu$
es, hasta derivadas totales y potencias en los campos de hasta dimensión 4:
\begin{align}
  \mathcal{L}=&-\frac{1}{4}F^{\mu\nu}F_{\mu\nu}-\frac{1}{4}G^{\mu\nu}G_{\mu\nu}-\frac{1}{2}F^{\mu\nu}G_{\mu\nu}\nonumber\\
&-J^\mu A_\mu+
 \frac{1}{2}m^2A^\mu A_\mu +\lambda_1\partial_\nu A^\nu(x) A_\mu(x) A^\mu(x)+\lambda_2 A^\mu A_\mu A^\nu A_\nu\nonumber\\
&+\lambda_3 F^{\mu\nu}(x)A_\mu(x) A_\nu(x)+\lambda_4G^{\mu\nu}(x)A_\mu(x) A_\nu+\cdots
  \label{eq:lagAmu}
\end{align}
\end{frame}

\begin{itemize}
\item \textbf{Ejercicio:} Show that terms like $\partial^\mu A^\nu(x)\partial_\mu A_\nu(x)$, and hence $F^{\mu\nu}F_{\mu\nu}$, transforms as
  \begin{align}
    \partial^\mu A^\nu\left(\Lambda^{-1}x\right)\partial_\mu A_\nu\left(\Lambda^{-1}x\right)
  \end{align}
Hint: use the Lorentz transformation properties of $\partial_\mu$ in eq.~\eqref{dmulrtran}.
\end{itemize}
In the case of $J^\mu A_\mu$:
\begin{align}
  J^\mu(x)A_\mu(x)\to g_{\mu\nu}{J'}^\mu(x){A'}^\nu(x)=& g_{\mu\nu}{\Lambda^\mu}_\rho J^\rho\left(\Lambda^{-1}x\right){\Lambda^\nu}_\sigma A^\sigma\left(\Lambda^{-1}x\right)\nonumber\\
=& {\Lambda^\mu}_\rho g_{\mu\nu}{\Lambda^\nu}_\sigma J^\rho\left(\Lambda^{-1}x\right)A^\sigma\left(\Lambda^{-1}x\right)\nonumber\\
=& g_{\rho\sigma}J^\rho\left(\Lambda^{-1}x\right)A^\sigma\left(\Lambda^{-1}x\right)\,,
\end{align}
in the case $\partial_\nu A^\nu(x) A_\mu(x) A^\mu(x)$:
\begin{align}
   \partial_\nu A^\nu(x) A_\mu(x) A^\mu(x)\to {\partial'}_\nu{A'}^\nu(x') {A'}_\mu(x') {A'}^\mu(x')=& {\left(\Lambda^{-1}\right)^\sigma}_\nu{\Lambda^\nu}_\rho\partial_\sigma A^\rho\left(\Lambda^{-1}x\right) A_\mu\left(\Lambda^{-1}x\right) A^\mu\left(\Lambda^{-1}x\right)\nonumber\\
=& \delta^\sigma_\rho\partial_\sigma A^\rho\left(\Lambda^{-1}x\right) A_\mu\left(\Lambda^{-1}x\right) A^\mu\left(\Lambda^{-1}x\right)\nonumber\\
=& \partial_\rho A^\rho\left(\Lambda^{-1}x\right) A_\mu\left(\Lambda^{-1}x\right) A^\mu\left(\Lambda^{-1}x\right)\,,\nonumber\\
\end{align}
and similarly for the other terms. Under a Lorentz transformation the full Lagrangian transform as
\begin{align}
  \mathcal{L}(x)\to\mathcal{L}'(x)=\mathcal{L}(\Lambda^{-1}x) 
\end{align}
Since the Action involves the integration over all the points, it is invariant under the Lorentz transformation. The $J^\mu(x)$ does not involves the introduction a new vector field, because it will be identified later as the 4--current.


Terms like
\begin{align}
  K_\nu A^\nu(x) A_\mu(x) A^\mu(x)\,,
\end{align}
(for $K_\nu$ constant) are not Lorentz invariant:
\begin{align}
  K_\nu A^\nu(x) A_\mu(x) A^\mu(x)\to K_\nu{A'}^\nu(x) {A'}_\mu(x) {A'}^\mu(x)=& K_\nu{\Lambda^\nu}_\rho A^\rho\left(\Lambda^{-1}x\right) A_\mu\left(\Lambda^{-1}x\right) A^\mu\left(\Lambda^{-1}x\right)\,.
\end{align}
$K_\nu(x)A^\nu(x)A_\mu(x)A^\mu(x)$ is Lorentz invariant %but not gauge-invariant (see below).




%\begin{frame}[fragile,allowframebreaks]
% The field $A^\mu(x)$ transforms simultaneously as field and as vector under Lorentz transformation
% \begin{align}
%   A^\mu(x)\to {A'}^\mu(x')={\Lambda^\mu}_\nu A^\nu(\Lambda^{-1}x)\,.
% \end{align}
% \end{frame}





\end{subappendices}



\section{Problemas}
\label{sec:problemas2}
\renewcommand{\labelenumi}{\thechapter.\theenumi} %noinstiki

\begin{enumerate}
\item Muestre que el Lagrangiano electromagn\'etico en ausencia de corrientes
  \begin{equation}
    \mathcal{L}=-\frac{1}{4}F^{\mu\nu}F_{\mu\nu}=\frac{1}{2}\left(\mathbf{E}^2-\mathbf{B}^2\right)
  \end{equation}
\label{item:pch2.2} %noinstiki

\item Calcule el rango de la interacci\'on d\'ebil mediada por la part\'\i cula $W^{-}$ de masa
  \begin{equation}
    m_W\approx80\;GeV
  \end{equation}
\label{item:pch2.3} %noinstiki
\end{enumerate} %noinstiki
\renewcommand{\labelenumi}{\theenumi} %noinstiki



%\begin{frame}[fragile,allowframebreaks]
%\end{frame}


%%% Local Variables: 
%%% mode: latex
%%% TeX-master: "fullnotes"
%%% ispell-local-dictionary: "castellano8"
%%% End: