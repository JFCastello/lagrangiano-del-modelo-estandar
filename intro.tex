%instiki:category: FisicaSubatomica
%instiki:
%instiki:***
%instiki:
%instiki:[[NotasFS|Tabla de Contenidos]]
%instiki:
%instiki:***
%instiki:## Introducción ##

\chapter*{Introducción\markboth{Introducción}{Introducción}} %noinstiki
\addcontentsline{toc}{chapter}{Introducción} %noinstiki

Nuestro entendimiento de cómo funciona el mundo físico ha llegado a una culminación exitosa  en a\~nos recientes con el desarrollo y comprobación experimental del Modelo Estándar (ME) de las interacciones fundamentales. A medida que el ME ha llegado a ser mejor entendido y comprobado experimentalmente, mucha más gente con algunos conocimientos en física  desea entender cuantitativamente éste nuevo éxito de las ciencia moderna. Debido a esto, parece esencial tener una presentación del ME que pueda ser usada a nivel de pregrado. En el presente trabajo pretendemos desarrollar un núcleo básico de temas que permitan construir el Lagrangiano del Modelo Estándar y mostrar que dentro de este marco se pueden ilustrar los aspectos más importantes del ME. Este material de apoyo está dirigido a personas que con conocimientos de mecánica y electromagnetismo a nivel de pregrado quieran obtener un conocimiento básico de los principios subyacentes y las principales predicciones del Modelo Estándar de las partículas elementales que permita apreciar una de las fronteras más importantes de la ciencia moderna. 

El Modelo Estándar da una descripción amplia de las partículas básicas y las fuerzas de la naturaleza y de como pueden ser descritos todos los fenómenos físicos que vemos. Este contiene los principios subyacentes de todo el comportamiento de protones, núcleos, átomos, moléculas, materia condensada, estrellas, y más. El Modelo Estándar ha explicado mucho de lo que no fue entendido antes; ha hecho cientos de predicciones exitosas, incluyendo muchas que han sido dramáticas; y a excepción de los resultados sobre oscilaciones de neutrinos, no hay ningún fenómeno en su dominio que no haya sido explicado.

La madurez de la formulación del Modelo Estándar hace que cada vez sea más necesario que cualquier persona formada en física deba tener un conocimiento básico de sus principios subyacentes y sus principales predicciones. Sin embargo, una apreciación completa del éxito y el significado del Modelo Estándar requiere de un conocimiento profundo de la Teoría Cuántica de Campos que va más allá de lo que es  ense\~nado usualmente en los cursos de pregrado en física. La Teoría Cuántica de Campos combina de forma coherente la mecánica cuántica y la relatividad especial. No fue sino hasta que la Teoría Cuántica de Campos quedó completamente formulada a principio de los a\~nos setenta (con la prueba de que las Teorías Gauge no abelianas con ruptura espontánea de simetría eran renormalizables), que el Modelo Estándar consiguió emerger como la teoría que explica el comportamiento conocido de las partículas elementales y sus interacciones. 

A medida que el Modelo Estándar ha llegado a ser mejor entendido y comprobado experimentalmente, mucha más gente con algunos conocimientos en física pero que no esta interesada en trabajar en física de partículas, desea entender cuantitativamente éste nuevo éxito de las ciencias moderna. Debido a esto, parece esencial tener una presentación del Modelo Estándar que pueda ser usada a nivel pregrado. Además, parece aún más importante tener libros que cualquier estudiante o físico que tenga los conocimientos necesarios pueda leer para aprender sobre los desarrollos en física de partículas. Aquellos desarrollos deberían ser parte de la educación de cualquiera interesado en lo que la humanidad ha aprendido acerca de los constituyentes básicos de la materia y las fuerzas de la naturaleza. 

Pero hasta hace poco tiempo no había un sitio donde las personas con los conocimientos suficientes pudieran ir a aprender esos desarrollos. Para llenar este vacío aparecieron libros como el de Kane~\cite{kane} (1993) y Cottingham~\cite{cottingham} (1998) donde los prerrequsitos mínimos son un curso introductorio en mecánica cuántica (Física Moderna) y los cursos normales de pregrado en mecánica y electromagnetismo. Con éstas herramientas es posible obtener un buen entendimiento a nivel cuantitativo de la física de partículas moderna. En estos libros se ha hecho un esfuerzo por extraer los conceptos y técnicas básicas usadas en el Modelo Estándar de modo que un mayor número de físicos no especializados en el área puedan tener una visión del logro intelectual representado por el Modelo, y compartir el excitamiento por su éxito. En estos libros el Modelo Estándar es ense\~nado escribiendo la forma básica de la teoría y extrayendo sus consecuencias. 

Todos los tratamientos anteriores eran a nivel de posgrado  para físicos que querían especializarse en el área, o descripciones populares demasiado superficiales para realmente entender los desarrollos, o descripciones históricas carentes de la lógica profunda del Modelo Estándar. 

Con nuestra experiencia dictando los cursos de introducción a la física de partículas en la carrera de Física basados en estos textos, ha quedado claro que con una presentación del Modelo Estándar de una forma deductiva en lugar de la aproximación histórica usual de los libros más avanzados, los estudiantes pueden llegar a entender la estructura básica de la física de partículas moderna. Además resulta ser una peque\~na extensión adicionar el marco apropiado para entender porque algunas direcciones de la investigación de frontera se enfatizan más, y en que direcciones se espera que aparezcan nuevos progresos.

Con el presente trabajo queremos ir más allá del objetivo de los anteriores libros y mostrar que con una reorganización e inclusión de tópicos adicionales no enfatizado en esos libros, con sólo los conocimientos de los cursos normales de pregrado en mecánica y electromagnetismo, se puede llegar más rápidamente a un entendimiento cuantitativo de los aspectos más importantes del Modelo Estándar.  

En el presente trabajo pretendemos desarrollar un núcleo básico de temas que permitan construir el Langrangiano del Modelo Estándar y mostrar que dentro de este marco se pueden ilustrar los aspectos más importantes del Modelo Estándar.

La Teoría Cuántica de Campos (TCC) es el marco teórico utilizado en la descripción cuántica de campos relativistas. En este teoría se estudian sistemas en los cuales las partículas pueden ser creadas y destruidas. Esta teoría resulta de combinar la mecánica cuántica con la relatividad especial en un marco consistente. Las teorías cuánticas de campos se describen más convenientemente en el formalismo Lagrangiano. De hecho, los Lagrangianos que describen las partículas elementales y sus interacciones puede ser construidos a partir de principios de simetría.  Aspectos básicos de la construcción de la TCC son:
\begin{itemize} %noinstiki
\item Lagrangianos para campos de una sola partícula
\item Tratamiento relativístico de un campo
\item Segunda cuantización de campos de una sola partícula, bien sea mediante la cuantización canónica o a través de integrales de camino
\item Tratamiento de una sola partícula de forma covariante que permita describir la creación y aniquilación de partículas.
\item Invarianza gauge local
\end{itemize}
Ejemplos concretos de teorías cuánticas de campos son:
\begin{itemize}
\item Electrodinámica Cuántica. Resulta de imponer el principio gauge local basado en una simetría abeliana a la ecuación de Dirac para el electrón. Como resultado se obtienen las ecuaciones de Maxwell con un término de corriente asociada a la interacción electromagnética entre el electrón y el fotón.
\item Cromodinámica Cuántica. Resulta de imponer el principio gauge local basado en una simetría no abeliana a la ecuación de Dirac para los quarks. Entre los resultados se explica la libertad asintótica observada en las interacciones fuertes y la autointeracción de los bosones gauge.
\item Teoría Electrodébil. Resulta de imponer el principio gauge local simultáneamente para una simetría abeliana y para una simetría no abeliana a la ecuación de Dirac para los fermiones conocidos. La simetría no abeliana en este caso prohibe términos de masa para los fermiones, mientras que la invarianza gauge local, como en los casos anteriores, prohibe los términos de masa para los bosones gauge. De este modo, cuando las simetrías de la teoría electrodébil son exactas, todas las partículas aparecen sin masa. Para ser consistente con el espectro de fermiones y bosones conocidos, se introducen 4 campos escalares sin masa organizado en lo que se conoce como un doblete de Higgs. Uno de ellos desarrolla un valor esperado de vacío y rompe espontáneamente la simetría, generando masa para todos los fermiones. Mediante el mismo mecanismo los otros 3 campos escalares ayudan a explicar las masas para tres de los bosones gauge del modelo. El fotón, junto con los neutrinos permanecen sin masa. 
\end{itemize} %noinstiki

El Modelo Estándar es la TCC que combina de forma consistente la Cromodinámica Cuántica y la Teoría Electrodébil. Después del rompimiento espontáneo de la simetría la parte Electrodébil se reduce a la Electrodinámica Cuántica. El estudio de los fundamentos y las consecuencias fenomenológicas del Modelo Estándar requiere del desarrollo completo de la TCC basado en conceptos y técnicas avanzadas de mecánica cuántica y relatividad especial a un nivel normalmente de posgrado. 

En las TCC los campos electromagnéticos se convierten en operadores que dependen del espacio y el tiempo. Los valores esperados de estos operadores en el ambiente descrito por los estados cuánticos dan lugar a los campos clásicos. Los otros campos bosónicos, y los campos de Dirac para los fermiones del Modelo Estándar también se convierten en operadores. 

Sin embargo se puede  escribir la forma básica del Modelo Estándar y extraer muchas de sus consecuencias tratando los campos bosónicos y de Dirac como simples funciones. 

En éste caso las excitaciones del campo se pueden pensar como funciones de ondas de una partícula. En este contexto el Lagrangiano del Modelo Estándar puede construirse usando solo conocimientos a nivel de pregrado de mecánica y electromagnetismo y  algunas referencias a aspectos básicos de la mecánica cuántica. 

Con estas herramientas se pueden escribir las ecuaciones de Maxwell en forma covariante y mostrar como éstas se pueden obtener, usando el principio de mínima acción, a partir del Lagrangiano para el vector de campo electromagnético. Se puede introducir a este nivel la importante idea de las transformaciones gauge y relacionarla con la conservación de la carga eléctrica. Se puede generalizar el Lagrangiano  para describir campos vectoriales masivos, lo que da lugar a la ecuación de Proca. A este punto, se puede obtener el Lagrangiano para una partícula escalar real, a partir de la componente escalar del campo vectorial y mostrar que dicho Lagrangiano de lugar a la ecuación de Klein-Gordon. Tomando la componente escalar como un campo independiente se puede usar el principio gauge local para estudiar sus interacciones con los campos gauge. Usando conceptos de mecánica cuántica se puede generalizar el potencial del Lagrangiano para el campo escalar de modo que se pueda  interpretar su masa como las oscilaciones del campo alrededor de estado de energía fundamental, el vació. Se puede estudiar a partir de allí, la ruptura espontánea de simetría. Con estos ingredientes se puede construir finalmente el Lagrangiano bosónico del Modelo Estándar. 

De forma más general se puede definir los campos fermiónicos y bosónicos a partir de sus propiedades de transformación bajo el grupo de Lorentz y usuarlo para construir Lagrangianos suficientemente simples imponiendo simetrías internas adicionales.


Siguiendo esta línea, primero se introducen los conceptos necesarios de teoría de Grupos y de relativid especial haciendo énfasis en los productos escalares bajo diferentes grupos de transformación como los bloques fundamentales para describir la Acción asociada a los campos. Para enfatizar estos puntos se establecen la leyes de Kirchhoff generalizadas para corrientes más allá de la corriente eléctrica. Los campos fundamentales: escalares, vectoriales y espinoriales que pueden formar multipletes bajo simsetría internas, aparecen simplemente como objetos de esos espacios externos e internos que sufren las transformaciones de esos grupos internos y externos.

Para conectar con los aspectos más físicos relacionados con los campos, se muestra que el principio de mínima Acción que usa como coordenadas generalizadas las oscilaciones de los campos y sus correspondientes derivadas, es un método conveniente para analizar cualquier teoría clásica de campos. Todas ellas están conectadas a través de los dos teoremas de Noether asociados con las transformaciones globales y locales respectivamente.  Mostramos entonces que para los campos fermiónicos las transformaciones globales dan lugar a los diferentes postulados de la mecánica cuántica cuando se interpretan como sistemas de una sola partícula descritos por funciones de onda. 
De otro lado, las simetrías locales internas permiten clasificar a los campos en dos tipos: los campos de radiación y los campos de materia. Mostramos entonces como la transformación gauge de un campo de radiación Abeliano da lugar a las ecuaciones de Maxwell.  De la misma forma, una transformación gauge de un campo de materia con carga eléctrica da lugar a la Acción de la electrodinámica cuántica. Después de generalizar los campos de materia y radiación a simetrías no Abelianas, pasamos a estudiar el problema de la generación de las masas para los campos fundamentales. 

A continuación se describe el contenido de algunos capítulos en un orden que no necesariamente coincide con el de la versión actual del texto.

En el capítulo~\ref{chap:tcc} se ha abordado la formulación Lagrangiana de la Teoría de Campos Clásica.
Al comienzo del capítulo se intrucen algunas nociones preliminares de Teoría de Grupos y de relatividad especial. En particular se hace énfasis en que cuando la velocidad de propagacón de los fenomenos ondulatorios no es independiente bajo transformaciones de Lorentz, la notación asociada a la relatividad especial sigue siendo útil. De esta manera, nuestra formulación de la teoría clásica de campos es aplicable a cualquier sistema ondulatorio. En el primer capítulo se hace énfasis en usar solo la notación relativista sin necesidad de imponer la invarianza de Lorentz en ninguno de los resultados importantes, de modo que se puedan aplicar por ejemplo a la ecuación de Scrödinger. 

En la sección \ref{sec:la} (ver Anexo 2), se formula el Principio de Mínima Acción para sistemas de partículas y se establece que la cantidad a minimizar corresponde al Lagrangiano del sistema. Este Lagrangiano depende de coordenadas generalizadas de desplazamiento y su derivada temporal. 

En la sección \ref{sec:la-cuerda-clasica} se construye el sistema continuo más simple correspondiente a las oscilaciones de una cuerda unidimensional. Esta se construye suponiendo un sistema discreto de partículas unidas por resortes que oscilan alrededor de su punto de equilibrio. Usando las técnicas de la sección~\ref{sec:la} se construye el Lagrangiano interpretando como desplazamiento el campo que describe las oscilaciones de cada partícula. Tomando el límite de infinitas partículas y separación cero, se reescribe la Acción en términos de la densidad Lagrangiana y se establece que para sistemas continuos es dicha densidad Lagrangiana la que hay que minimizar para establecer el principio de mínima acción. Tenemos entonces de un lado la Lagrangiana como punto de partida para describir las coordenadas de sistemas discretos, y la densidad Lagrangiana para describir los campos de desplazamiento de sistemas continuos. 

En la sección \ref{sec:principio-de-minima-call} se usan métodos variacionales para calcular la variación de la acción debida a transformaciones de los campos (transformaciones internas) y de las coordenadas (transformaciones externas). Con este resultado se derivan las ecuaciones de Euler-Lagrange en términos de la densidad Lagrangiana y se demuestra el  Teorema de Noether. Para simetrías internas se encuentra la expresión para la corriente conservada $J^\mu$ en términos de la densidad Lagrangiana. Para simetrías externas se encuentra la expresión para el tensor $T^{\mu\nu}$, que en sistemas relativistas se interpreta como el tensor de momento-energía $T^{\mu\nu}$. 

De este modo dada una densidad Lagrangiana, las cantidades a calcular corresponden a la ecuación de movimiento resultante de la ecuaciones de Euler Lagrange, la corriente conservada $J^\mu$ si la densidad de Lagrangiana posee alguna simetría continua, y el tensor $T^{\mu\nu}$. 

Para la densidad Lagrangiana que da lugar a la ecuación de Scr\"odinger en la sección \ref{sec:aplic-mecan-cuant}, la simetría interna de invarianza de fase da lugar a la conservación de la probabilidad, $\int_V T^0_0 d^3x$ da lugar a la energía del sistema, y la integral de $T^i_0$ da lugar al correspondiente número de onda de la solución de onda plana para la ecuación de Scr\"odinger.

La densidad Lagrangiana para la cuerda unidimensional no posee simetrías internas continuas. Con respecto a las simetrías externas, en la sección \ref{sec:aplicacion-la-cuerda} se muestra que para interpretar correctamente el tensor $T^\mu_\nu$ en este caso, es preciso cuantizar el campo que describe las oscilaciones de la cuerda. Al cuantizarlo se encuentra que éste describe una partícula que transporta energía, y en el límite en el cual la velocidad de la onda es la velocidad de la luz, la partícula correspondiente también transporta momentum. De hecho, para que una onda sea solución a las ecuaciones que describen las oscilaciones de una cuerda unidimensional relativista se requiere que su frecuencia y número de onda satisfagan la ecuación de energía-momento relativista con masa cero.
%La adición de un término de masa al Lagrangiano correspondiente, da lugar a la ecuación de Klein-Gordon para un campo real. 

En el capítulo~\ref{cha:campos-vectoriales} enfátiza la importancia de las transformaciones de Lorentz para la construcción de la densidades Lagrangianas compatibles con la relatividad especial, las cuales describen tres de las cuatro interacciones fundamenteles de la naturaleza.
Se estudia la versión covariante de la ecuaciones de Maxwell. Después de introducir las unidades naturales en la sección \ref{sec:NU} y la notación relativista en la sección \ref{sec:srn}, la forma covariante de las ecuaciones de Maxwell se desarrolla en la sección~\ref{sec:maxeqs}. All\'\i{} se explota la invarianza gauge para encontrar el Lagrangiano electromagnético, el cual se reduce a las ecuaciones de Klein-Gordon con masa igual a cero cuando se escoge el Gauge de Lorentz. En la sección~\ref{sec:ecuacion-de-proca}, se modifican las ecuaciones de Maxwell para permitir un término de masa. Cómo dicho término rompe la invarianza gauge de la teoría, el gauge de Lorentz pasa a ser una condición ineludible que da lugar a la ecuación de Klein-Gordon para un campo vectorial.

% En el capítulo se introduce la ecuación de Klein--Gordon como un caso particular de la ecuación de Proca. Se muestra que dicha ecuación exhibe una simetría discreta de paridad para el caso de un campo real. En las secciones subsiguientes se aumenta el Lagrangiano adicionando más grados de libertad de la misma masa y se muestra como va aumentando la simetría de una simetría discreta a simetrías continuas abelianas y no abelianas. A la luz de teorema de Noether estas simetrías dan lugar a cargas conservadas globalmente. En la sección \ref{sec:invar-gauge-local} se establece el principio gauge local para imponer que la carga se conserve localmente. Esto da lugar a la aparición del campo gauge asociado a las interacciones electromagnéticas y a la dinámica asociada a la interacción del campo escalar complejo en presencia del campo electromagnético. En la sección \ref{sec:invar-gauge-local-2} se generaliza el principio gauge local al caso no Abeliano, y en la sección \ref{sec:invar-gauge-local-1} se combinan el caso Abeliano con el no Abeliano. Las nuevas características de las ecuaciones del campo gauge no Abeliano se ilustran usando la representación adjunta del Grupo no Abeliano en la sección \ref{sec:phi-como-un}.

En el cuarto capítulo~\ref{rupt-espont-de} se %repite el mismo procedimiento del anterior capítulo pero reemplazando
reemplaza el potencial escalar de $\frac{1}{2}m^2\phi^2$ a $\frac{1}{2}\mu^2\phi^2-\frac{1}{4}\lambda\phi^4$ con $\mu^2\lt 0$ y $\lambda\gt 0$. De modo que $\mu$ no puede interpretarse como un parámetro de masa. Este potencial contiene un conjunto degenerado de vacíos que no respetan la invarianza del Lagrangiano. El análisis del espectro partículas alrededor de uno de estos mínimos da lugar a bosones de Higgs masivos, sección \ref{sec:masa-para-el}, bosones de Goldstone, sección \ref{sec:boson-de-goldstone}, y campos gauge masivos, sección \ref{sec:masa-para-el-1} a través del mecanismo de ruptura espontánea de la simetría. La sección \ref{sec:mecanismo-de-higgs} es un aplicación de todos los conceptos y técnicas desarrollados al caso del sector bosónico del Modelo Estándar de las partículas elementales. 

% En el quinto capítulo, basándonos en principios de simetría, se encuentra el Hamiltoniano de la Ecuación de Schr\"odinger dependiente sólo de primera derivadas y que es compatible con la ecuación de Klein--Gordon. Este corresponde la ecuación de Dirac que describe los fermiones del Modelo Estándar. El Hamiltoniano resulta ser en este caso una combinación lineal de matrices que satisfacen el álgebra de Dirac. El Lagrangiano de Dirac, resulta tener una invarianza abeliana global. En la sección \ref{sec:electr-cuant} se usa el principio gauge local para hacer que la carga asociada a la transformación abeliana sea local. Como antes  esto da lugar a la aparición del campo gauge asociado a las interacciones electromagnéticas y a la dinámica asociada a la interacción del campo fermiónico  en presencia del campo electromagnético. En la sección \ref{sec:solucion-de-parti} se definen la helicidad izquierda y derecha de los fermiones.

En el capítulo~\ref{cha:modelo-estandar}, se construye el Lagrangiano leptónico gauge local para el Grupo $SU(2)_L\times U(1)_Y$ que describe correctamente las interacciones débiles y electromagnéticas. Como 3 de los 4 bosones gauge son masivos debe introducirse un potencial escalar que rompa espontáneamente la simetría. En la sección \ref{sec:inter-fuert} se introduce el grupo gauge local no Abeliano responsable de las interacciones fuertes entre quarks y que completa la estructura de grupo del Modelo Estándar de las partículas elementales. 

La página web del curso está en:

\url{http://fisica.udea.edu.co/cursos}%\url{http://gfif.udea.edu.co/tikiwiki/showpage?htcc} %noinstiki[http://gfif.udea.edu.co/tikiwiki/showpage?htcc](http://gfif.udea.edu.co/tikiwiki/showpage?htcc)





%%% Local Variables: 
%%% mode: latex
%%% TeX-master: "fullnotes"
%%% End: