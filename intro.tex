%instiki:category: FisicaSubatomica
%instiki:
%instiki:***
%instiki:
%instiki:[[NotasFS|Tabla de Contenidos]]
%instiki:
%instiki:***
%instiki:## Introducción ##

\chapter*{Introducci\'on\markboth{Introducci\'on}{Introducci\'on}} %noinstiki
\addcontentsline{toc}{chapter}{Introducci\'on} %noinstiki

Nuestro entendimiento de c\'omo funciona el mundo f\'\i sico ha llegado a una culminaci\'on exitosa  en a\~nos recientes con el desarrollo y comprobaci\'on experimental del Modelo Est\'andar (ME) de las interacciones fundamentales. A medida que el ME ha llegado a ser mejor entendido y comprobado experimentalmente, mucha m\'as gente con algunos conocimientos en f\'\i sica  desea entender cuantitativamente \'este nuevo \'exito de las ciencias moderna. Debido a esto, parece esencial tener una presentaci\'on del ME que pueda ser usada a nivel pregrado. En el presente trabajo pretendemos desarrollar un n\'ucleo b\'asico de temas que permitan construir el Lagrangiano del Modelo Est\'andar y mostrar que dentro de este marco se pueden ilustrar los aspectos m\'as importantes del ME. Este material de apoyo est\'a dirigido a personas que con conocimientos de mec\'anica y electromagnetismo a nivel de pregrado quieran obtener un conocimiento b\'asico de los principios subyacentes y las principales predicciones del Modelo Est\'andar de las part\'\i culas elementales que permita apreciar una de las fronteras m\'as importantes de la ciencia moderna. 

El Modelo Est\'andar da una descripci\'on amplia de las part\'\i culas b\'asicas y las fuerzas de la naturaleza y de como pueden ser descritos todos los fen\'omenos f\'\i sicos que vemos. Este contiene los principios subyacentes de todo el comportamiento de protones, n\'ucleos, \'atomos, mol\'eculas, materia condensada, estrellas, y m\'as. El Modelo Est\'andar ha explicado mucho de lo que no fue entendido antes; ha hecho cientos de predicciones exitosas, incluyendo muchas que han sido dram\'aticas; y a excepci\'on de los resultados sobre oscilaciones de neutrinos, no hay ning\'un fen\'omeno en su dominio que no haya sido explicado.

La madurez de la formulaci\'on del Modelo Est\'andar hace que cada vez sea m\'as necesario que cualquier persona formada en f\'\i sica deba tener un conocimiento b\'asico de sus principios subyacentes y sus principales predicciones. Sin embargo, una apreciaci\'on completa del \'exito y el significado del Modelo Est\'andar requiere de un conocimiento profundo de la Teor\'\i a Cu\'antica de Campos que va m\'as all\'a de lo que es  ense\~nado usualmente en los cursos de pregrado en f\'\i sica. La Teor\'\i a Cu\'antica de Campos combina de forma coherente la mec\'anica cu\'antica y la relatividad especial. No fue sino hasta que la Teor\'\i a Cu\'antica de Campos qued\'o completamente formulada a principio de los a\~nos setenta (con la prueba de que las Teor\'\i as Gauge no abelianas con ruptura espont\'anea de simetr\'\i a eran renormalizables), que el Modelo Est\'andar consigui\'o emerger como la teor\'\i a que explica el comportamiento conocido de las part\'\i culas elementales y sus interacciones. 

A medida que el Modelo Est\'andar ha llegado a ser mejor entendido y comprobado experimentalmente, mucha m\'as gente con algunos conocimientos en f\'\i sica pero que no esta interesada en trabajar en f\'\i sica de part\'\i culas, desea entender cuantitativamente \'este nuevo \'exito de las ciencias moderna. Debido a esto, parece esencial tener una presentaci\'on del Modelo Est\'andar que pueda ser usada a nivel pregrado. Adem\'as, parece a\'un m\'as importante tener libros que cualquier estudiante o f\'\i sico que tenga los conocimientos necesarios pueda leer para aprender sobre los desarrollos en f\'\i sica de part\'\i culas. Aquellos desarrollos deber\'\i an ser parte de la educaci\'on de cualquiera interesado en lo que la humanidad ha aprendido acerca de los constituyentes b\'asicos de la materia y las fuerzas de la naturaleza. 

Pero hasta hace poco tiempo no hab\'\i a un sitio donde las personas con los conocimientos suficientes pudieran ir a aprender esos desarrollos. Para llenar este vac\'\i o aparecieron libros como el de Kane~\cite{kane} (1993) y Cottingham~\cite{cottingham} (1998) donde los prerrequsitos m\'\i nimos son un curso introductorio en mec\'anica cu\'antica (F\'\i sica Moderna) y los cursos normales de pregrado en mec\'anica y electromagnetismo. Con \'estas herramientas es posible obtener un buen entendimiento a nivel cuantitativo de la f\'\i sica de part\'\i culas moderna. En estos libros se ha hecho un esfuerzo por extraer los conceptos y t\'ecnicas b\'asicas usadas en el Modelo Est\'andar de modo que un mayor n\'umero de f\'\i sicos no especializados en el \'area puedan tener una visi\'on del logro intelectual representado por el Modelo, y compartir el excitamiento por su \'exito. En estos libros el Modelo Est\'andar es ense\~nado escribiendo la forma b\'asica de la teor\'\i a y extrayendo sus consecuencias. 

Todos los tratamientos anteriores eran a nivel de posgrado  para f\'\i sicos que quer\'\i an especializarse en el \'area, o descripciones populares demasiado superficiales para realmente entender los desarrollos, o descripciones hist\'oricas carentes de la l\'ogica profunda del Modelo Est\'andar. 

Con nuestra experiencia dictando los cursos de introducci\'on a la f\'\i sica de part\'\i culas en la carrera de F\'\i sica basados en estos textos, ha quedado claro que con una presentaci\'on del Modelo Est\'andar de una forma deductiva en lugar de la aproximaci\'on hist\'orica usual de los libros m\'as avanzados, los estudiantes pueden llegar a entender la estructura b\'asica de la f\'\i sica de part\'\i culas moderna. Adem\'as resulta ser una peque\~na extensi\'on adicionar el marco apropiado para entender porque algunas direcciones de la investigaci\'on de frontera se enfatizan m\'as, y en que direcciones se espera que aparezcan nuevos progresos.

Con el presente trabajo queremos ir m\'as all\'a del objetivo de los anteriores libros y mostrar que con una reorganizaci\'on e inclusi\'on de t\'opicos adicionales no enfatizado en esos libros, con s\'olo los conocimientos de los cursos normales de pregrado en mec\'anica y electromagnetismo, se puede llegar m\'as r\'apidamente a un entendimiento cuantitativo de los aspectos m\'as importantes del Modelo Est\'andar.  

En el presente trabajo pretendemos desarrollar un n\'ucleo b\'asico de temas que permitan construir el Langrangiano del Modelo Est\'andar y mostrar que dentro de este marco se pueden ilustrar los aspectos m\'as importantes del Modelo Est\'andar.

La Teor\'\i a Cu\'antica de Campos (TCC) es el marco te\'orico utilizado en la descripci\'on cu\'antica de campos relativistas. En este teor\'\i a se estudian sistemas en los cuales las part\'\i culas pueden ser creadas y destruidas. Esta teor\'\i a resulta de combinar la mec\'anica cu\'antica con la relatividad especial en un marco consistente. Las teor\'\i as cu\'anticas de campos se describen m\'as convenientemente en el formalismo Lagrangiano. De hecho, los Lagrangianos que describen las part\'\i culas elementales y sus interacciones puede ser construidos a partir de principios de simetr\'\i a.  Aspectos b\'asicos de la construcci\'on de la TCC son:
\begin{itemize} %noinstiki
\item Lagrangianos para campos de una sola part\'\i cula
\item Tratamiento relativ\'\i stico de un campo
\item Segunda cuantizaci\'on de campos de una sola part\'\i cula, bien sea mediante la cuantizaci\'on can\'onica o a trav\'es de integrales de camino
\item Tratamiento de una sola part\'\i cula de forma covariante que permita describir la creaci\'on y aniquilaci\'on de part\'\i culas.
\item Invarianza gauge local
\end{itemize}
Ejemplos concretos de teor\'\i as cu\'anticas de campos son:
\begin{itemize}
\item Electrodin\'amica Cu\'antica. Resulta de imponer el principio gauge local basado en una simetr\'\i a abeliana a la ecuaci\'on de Dirac para el electr\'on. Como resultado se obtienen las ecuaciones de Maxwell con un t\'ermino de corriente asociada a la interacci\'on electromagn\'etica entre el electr\'on y el fot\'on.
\item Cromodin\'amica Cu\'antica. Resulta de imponer el principio gauge local basado en una simetr\'\i a no abeliana a la ecuaci\'on de Dirac para los quarks. Entre los resultados se explica la libertad asint\'otica observada en las interacciones fuertes y la autointeracci\'on de los bosones gauge.
\item Teor\'\i a Electrod\'ebil. Resulta de imponer el principio gauge local simult\'aneamente para una simetr\'\i a abeliana y para una simetr\'\i a no abeliana a la ecuaci\'on de Dirac para los fermiones conocidos. La simetr\'\i a no abeliana en este caso prohibe t\'erminos de masa para los fermiones, mientras que la invarianza gauge local, como en los casos anteriores, prohibe los t\'erminos de masa para los bosones gauge. De este modo, cuando las simetr\'\i as de la teor\'\i a electrod\'ebil son exactas, todas las part\'\i culas aparecen sin masa. Para ser consistente con el espectro de fermiones y bosones conocidos, se introducen 4 campos escalares sin masa organizado en lo que se conoce como un doblete de Higgs. Uno de ellos desarrolla un valor esperado de vac\'\i o y rompe espont\'aneamente la simetr\'\i a, generando masa para todos los fermiones. Mediante el mismo mecanismo los otros 3 campos escalares ayudan a explicar las masas para tres de los bosones gauge del modelo. El fot\'on, junto con los neutrinos permanecen sin masa. 
\end{itemize} %noinstiki
El Modelo Est\'andar es la TCC que combina de forma consistente la Cromodin\'amica Cu\'antica y la Teor\'\i a Electrod\'ebil. Despu\'es del rompimiento espont\'aneo de la simetr\'\i a la parte Electrod\'ebil se reduce a la Electrodin\'amica Cu\'antica. El estudio de los fundamentos y las consecuencias fenomenol\'ogicas del Modelo Est\'andar requiere del desarrollo completo de la TCC basado en conceptos y t\'ecnicas avanzadas de mec\'anica cu\'antica y relatividad especial a un nivel normalmente de posgrado. 

En las TCC los campos electromagn\'eticos se convierten en operadores que dependen del espacio y el tiempo. Los valores esperados de estos operadores en el ambiente descrito por los estados cu\'anticos dan lugar a los campos cl\'asicos. Los otros campos bos\'onicos, y los campos de Dirac para los fermiones del Modelo Est\'andar tambi\'en se convierten en operadores. 

Sin embargo se puede  escribir la forma b\'asica del Modelo Est\'andar y extraer muchas de sus consecuencias tratando los campos bos\'onicos y de Dirac como simples funciones. 

En \'este caso los campos se pueden pensar como funciones de ondas de una part\'\i cula. En este contexto el Lagrangiano del Modelo Est\'andar puede construirse usando solo conocimientos a nivel de pregrado de mec\'anica y electromagnetismo y  algunas referencias a aspectos b\'asicos de la mec\'anica cu\'antica. 

Con estas herramientas se pueden escribir las ecuaciones de Maxwell en forma covariante y mostrar como \'estas se pueden obtener, usando el principio de m\'\i nima acci\'on, a partir del Lagrangiano para el vector de campo electromagn\'etico. Se puede introducir a este nivel la importante idea de las transformaciones gauge y relacionarla con la conservaci\'on de la carga el\'ectrica. Se puede generalizar el Lagrangiano  para describir campos vectoriales masivos, lo que da lugar a la ecuaci\'on de Proca. A este punto, se puede obtener el Lagrangiano para una part\'\i cula escalar real, a partir de la componente escalar del campo vectorial y mostrar que dicho Lagrangiano de lugar a la ecuaci\'on de Klein-Gordon. Tomando la componente escalar como un campo independiente se puede usar el principio gauge local para estudiar sus interacciones con los campos gauge. Usando conceptos de mec\'anica cu\'antica se puede generalizar el potencial del Lagrangiano para el campo escalar de modo que se pueda  interpretar su masa como las oscilaciones del campo alrededor de estado de energ\'\i a fundamental, el vaci\'o. Se puede estudiar a partir de all\'\i, la ruptura espont\'anea de simetr\'\i a. Con estos ingredientes se puede construir finalmente el Lagrangiano bos\'onico del Modelo Est\'andar. 

Despu\'es de esto, se puede introducir el campo de Dirac como la soluci\'on al la ecuaci\'on equivalente a la ecuaci\'on Scr\"odinger pero compatible con la relatividad especial. Luego de definir el esp\'\i n y la helicidad se impone la invarianza gauge local del Modelo Est\'andar al Lagrangiano de Dirac y se obtiene de esta forma el Lagrangiano completo del Modelo Est\'andar.


En el primer cap\'\i tulo se ha abordado la formulaci\'on Lagrangiana de la Teor\'\i a de Campos Cl\'asica. En la secci\'on \ref{sec:la} (ver Anexo 2), se formula el Principio de M\'\i nima Acci\'on para sistemas de part\'\i culas y se establece que la cantidad a minimizar corresponde al Lagrangiano del sistema. Este Lagrangiano depende de coordenadas generalizadas de desplazamiento y su derivada temporal. 

En la secci\'on \ref{sec:la-cuerda-clasica} se construye el sistema continuo m\'as simple correspondiente a las oscilaciones de una cuerda unidimensional. Esta se construye suponiendo un sistema discreto de part\'\i culas unidas por resortes que oscilan alrededor de su punto de equilibrio. Usando las t\'ecnicas de la secci\'on~\ref{sec:la} se construye el Lagrangiano interpretando como desplazamiento el campo que describe las oscilaciones de cada part\'\i cula. Tomando el l\'\i mite de infinitas part\'\i culas y separaci\'on cero, se reescribe la Acci\'on en t\'erminos de la densidad Lagrangiana y se establece que para sistemas continuos es dicha densidad Lagrangiana la que hay que minimizar para establecer el principio de m\'\i nima acci\'on. Tenemos entonces de un lado la Lagrangiana como punto de partida para describir las coordenadas de sistemas discretos, y la densidad Lagrangiana para describir los campos de desplazamiento de sistemas continuos. 

En la secci\'on \ref{sec:principio-de-minima-call} se usan m\'etodos variacionales para calcular la variaci\'on de la acci\'on debida a transformaciones de los campos (transformaciones internas) y de las coordenadas (transformaciones externas). Con este resultado se derivan las ecuaciones de Euler-Lagrange en t\'erminos de la densidad Lagrangiana y se demuestra el  Teorema de Noether. Para simetr\'\i as internas se encuentra la expresi\'on para la corriente conservada $J^\mu$ en t\'erminos de la densidad Lagrangiana. Para simetr\'\i as externas se encuentra la expresi\'on para el tensor $T^{\mu\nu}$, que en sistemas relativistas se interpreta como el tensor de momento-energ\'\i a $T^{\mu\nu}$. 

De este modo dada una densidad Lagrangiana, las cantidades a calcular corresponden a la ecuaci\'on de movimiento resultante de la ecuaciones de Euler Lagrange, la corriente conservada $J^\mu$ si la densidad de Lagrangiana posee alguna simetr\'\i a continua, y el tensor $T^{\mu\nu}$. 

Para la densidad Lagrangiana que da lugar a la ecuaci\'on de Scr\"odinger en la secci\'on \ref{sec:aplic-mecan-cuant}, la simetr\'\i a interna de invarianza de fase da lugar a la conservaci\'on de la probabilidad, $\int_V T^0_0 d^3x$ da lugar a la energ\'\i a del sistema, y la integral de $T^i_0$ da lugar al correspondiente n\'umero de onda de la soluci\'on de onda plana para la ecuaci\'on de Scr\"odinger.

La densidad Lagrangiana para la cuerda unidimensional no posee simetr\'\i as internas continuas. Con respecto a las simetr\'\i as externas, en la secci\'on \ref{sec:aplicacion-la-cuerda} se muestra que para interpretar correctamente el tensor $T^\mu_\nu$ en este caso, es preciso cuantizar el campo que describe las oscilaciones de la cuerda. Al cuantizarlo se encuentra que \'este describe una part\'\i cula que transporta energ\'\i a, y en el l\'\i mite en el cual la velocidad de la onda es la velocidad de la luz, la part\'\i cula correspondiente tambi\'en transporta momentum. De hecho, para que una onda sea soluci\'on a las ecuaciones que describen las oscilaciones de una cuerda unidimensional relativista se requiere que su frecuencia y n\'umero de onda satisfagan la ecuaci\'on de energ\'\i a-momento relativista con masa cero. La adici\'on de un t\'ermino de masa al Lagrangiano correspondiente, da lugar a la ecuaci\'on de Klein-Gordon para un campo real. 

En el segundo cap\'\i tulo se estudia la versi\'on covariante de la ecuaciones de Maxwell. Despu\'es de introducir las unidades naturales en la secci\'on \ref{sec:NU} y la notaci\'on relativista en la secci\'on \ref{sec:srn}, la forma covariante de las ecuaciones de Maxwell se desarrolla en la secci\'on~\ref{sec:maxeqs}. All\'\i{} se explota la invarianza gauge para encontrar el Lagrangiano electromagn\'etico, el cual se reduce a las ecuaciones de Klein-Gordon con masa igual a cero cuando se escoge el Gauge de Lorentz. En la secci\'on~\ref{sec:ecuacion-de-proca}, se modifican las ecuaciones de Maxwell para permitir un t\'ermino de masa. C\'omo dicho t\'ermino rompe la invarianza gauge de la teor\'\i a, el gauge de Lorentz pasa a ser una condici\'on ineludible que da lugar a la ecuaci\'on de Klein-Gordon para un campo vectorial.

En el tercer cap\'\i tulo se introduce la ecuaci\'on de Klein--Gordon como un caso particular de la ecuaci\'on de Proca. Se muestra que dicha ecuaci\'on exhibe una simetr\'\i a discreta de paridad para el caso de un campo real. En las secciones subsiguientes se aumenta el Lagrangiano adicionando m\'as grados de libertad de la misma masa y se muestra como va aumentando la simetr\'\i a de una simetr\'\i a discreta a simetr\'\i as continuas abelianas y no abelianas. A la luz de teorema de Noether estas simetr\'\i as dan lugar a cargas conservadas globalmente. En la secci\'on \ref{sec:invar-gauge-local} se establece el principio gauge local para imponer que la carga se conserve localmente. Esto da lugar a la aparici\'on del campo gauge asociado a las interacciones electromagn\'eticas y a la din\'amica asociada a la interacci\'on del campo escalar complejo en presencia del campo electromagn\'etico. En la secci\'on \ref{sec:invar-gauge-local-2} se generaliza el principio gauge local al caso no Abeliano, y en la secci\'on \ref{sec:invar-gauge-local-1} se combinan el caso Abeliano con el no Abeliano. Las nuevas caracter\'\i sticas de las ecuaciones del campo gauge no Abeliano se ilustran usando la representaci\'on adjunta del Grupo no Abeliano en la secci\'on \ref{sec:phi-como-un}.

En el cuarto cap\'\i tulo se repite el mismo procedimiento del anterior cap\'\i tulo pero reemplazando el potencial escalar de $\frac{1}{2}m^2\phi^2$ a $\frac{1}{2}\mu^2\phi^2-\frac{1}{4}\lambda\phi^4$ con $\mu^2\lt 0$ y $\lambda\gt 0$. De modo que $\mu$ no puede interpretarse como un par\'ametro de masa. Este potencial contiene un conjunto degenerado de vac\'\i os que no respetan la invarianza del Lagrangiano. El an\'alisis del espectro part\'\i culas alrededor de uno de estos m\'\i nimos da lugar a bosones de Higgs masivos, secci\'on \ref{sec:masa-para-el}, bosones de Goldstone, secci\'on \ref{sec:boson-de-goldstone}, y campos gauge masivos, secci\'on \ref{sec:masa-para-el-1} a trav\'es del mecanismo de ruptura espont\'anea de la simetr\'\i a. La secci\'on \ref{sec:mecanismo-de-higgs} es un aplicaci\'on de todos los conceptos y t\'ecnicas desarrollados al caso del sector bos\'onico del Modelo Est\'andar de las part\'\i culas elementales. 

En el quinto cap\'\i tulo, bas\'andonos en principios de simetr\'\i a, se encuentra el Hamiltoniano de la Ecuaci\'on de Schr\"odinger dependiente s\'olo de primera derivadas y que es compatible con la ecuaci\'on de Klein--Gordon. Este corresponde la ecuaci\'on de Dirac que describe los fermiones del Modelo Est\'andar. El Hamiltoniano resulta ser en este caso una combinaci\'on lineal de matrices que satisfacen el \'algebra de Dirac. El Lagrangiano de Dirac, resulta tener una invarianza abeliana global. En la secci\'on \ref{sec:electr-cuant} se usa el principio gauge local para hacer que la carga asociada a la transformaci\'on abeliana sea local. Como antes  esto da lugar a la aparici\'on del campo gauge asociado a las interacciones electromagn\'eticas y a la din\'amica asociada a la interacci\'on del campo fermi\'onico  en presencia del campo electromagn\'etico. En la secci\'on \ref{sec:solucion-de-parti} se definen la helicidad izquierda y derecha de los fermiones.

En el cap\'\i tulo seis, se construye el Lagrangiano lept\'onico gauge local para el Grupo $SU(2)_L\times U(1)_Y$ que describe correctamente las interacciones d\'ebiles y electromagn\'eticas. Como 3 de los 4 bosones gauge son masivos debe introducirse un potencial escalar que rompa espont\'aneamente la simetr\'\i a. En la secci\'on \ref{sec:inter-fuert} se introduce el grupo gauge local no Abeliano responsable de las interacciones fuertes entre quarks y que completa la estructura de grupo del Modelo Est\'andar de las part\'\i culas elementales. 

La p\'agina web del curso est\'a en:

\url{http://fisica.udea.edu.co/cursos}%\url{http://gfif.udea.edu.co/tikiwiki/showpage?htcc} %noinstiki[http://gfif.udea.edu.co/tikiwiki/showpage?htcc](http://gfif.udea.edu.co/tikiwiki/showpage?htcc)





%%% Local Variables: 
%%% mode: latex
%%% TeX-master: "fullnotes"
%%% End: