%instiki:category: FisicaSubatomica
\chapter{Fermiones}
\label{cha:princ-gauge-local} %noinstiki
%instiki:
%instiki:***
%instiki:
%instiki:[[NotasFS|Tabla de Contenidos]]
%instiki:
%instiki:***
%instiki:
%instiki:* [Ecuaci\'on de klein-Gordon](#ecuacion-de-klein)
%instiki:
%instiki:* [Campos escalares complejos](#camp-escal-compl)
%instiki:
%instiki:* [Invarianza gauge local abeliana](#invar-gauge-local)
%instiki:
%instiki:* [Aplicaci\'on a la mec\'anica cu\'antica](#aplic-la-mecan)
%instiki:
%instiki:* [Invarianza gauge local no abeliana](#invar-gauge-local-2)
%instiki:
%instiki:* [Invarianza gauge local para un grupo semisimple](#invar-gauge-local-1)
%instiki:
%instiki:* [$\Phi$ como un triplete de $SU(2)$](#phi-como-un)
%instiki:
%instiki:* [Problemas](#problemas-3)
%instiki:
%instiki:***
%instiki:



\section{Grupo de Lorentz }


Para estudiar otros posibles tipos de campos además de los escalares y vectoriales, debemos explorar las representaciones del Grupo de Lorentz en $n$ dimensiones. 

Seguiremos el mismo método de encontrar representaciones matriciales a partir del algebra  de los generadores del Grupo (los cuales deben satisfacer la relaciones de conmutación apropiadas) para luego exponenciar estas representaciones infinitesimales.

\begin{frame}[fragile,allowframebreaks]
Para el presente problema, necesitamos conocer las relaciones de conmutación de los generadores del grupo de transformaciones de Lorentz. Hemos mostrado en la ec.~ \eqref{eq:rotgr}  que, a partir de la relación (haciendo expícito el caracter de operadores)
\begin{align}
\label{eq:rxp}
  \widehat{\mathbf{J}}=&\widehat{\mathbf{r}}\times \widehat{\mathbf{p}}=
\widehat{\mathbf{r}}\times (-i\boldsymbol{\nabla})
\end{align}
la parte correspondiente al grupo de rotaciones es
\begin{align*}
  \left[\widehat{J}^i,\widehat{J}^j\right]=i\epsilon_{ijk}\widehat{J}^k\,.
\end{align*}

La ecuación \eqref{eq:rxp} en términos de componentes esta dada en~\eqref{eq:rxpi} y corresponde a
\begin{align}
  \widehat{J}^k=i\epsilon_{ijk}x^i\partial^j
\end{align}
Definimos una representación matricial de los operadores de momento angular como
\begin{align}
  \widehat{J}^{l m}\equiv\epsilon_{lmk}\widehat{J}^k=&i\epsilon_{lmk}\epsilon_{ijk}x^i\partial^j\nonumber\\
=&i(\delta_{li}\delta_{mj}-\delta_{lj}\delta_{mi})x^i\partial^j\nonumber\\
=&i(x^l\partial^m-x^m\partial^l)\,.
\end{align}
\begin{spanish}
  Que involucran tres generadores. La generalización a cuatro dimensiones da lugar a generadores adicionales $\widehat{J}^{0i}$:
\end{spanish}
\begin{english}
  Involving three generators. The generalization to four-dimensions give to arise three further generators $\widehat{J}^{0i}$:
\end{english}
\begin{align}
  \widehat{J}^{\mu\nu}=i(x^\mu\partial^\nu-x^\nu\partial^\mu)\,.
\end{align}
\begin{spanish}
Los seis generadores satisfacen el álgebra
\end{spanish}
\begin{english}
  The six generators $\widehat{J}^{\mu\nu}$ satisfy the algebra
\end{english}
\begin{align}
\label{eq:lrtalg}
  \left[\widehat{J}^{\mu\nu},\widehat{J}^{\rho\sigma}\right]=&
i(g^{\nu\rho}\widehat{J}^{\mu\sigma}-g^{\mu\rho}\widehat{J}^{\nu\sigma}-g^{\nu\sigma}\widehat{J}^{\mu\rho}+g^{\mu\sigma}\widehat{J}^{\nu\rho})\,.
\end{align}

%From \cite{Peskin}:
Cualquier representación matricial de estos operadores que vaya a representar esta álgebra debe obedecer las mismas reglas de conmutación.

\begin{spanish}
  La exponenciación de los generadores da lugar al grupo de elementos 
\end{spanish}
\begin{english}
  The exponentiation of the generators give to arise to group elements
\end{english}
\begin{align}
  \widehat{\Lambda}=\exp\left(-i\omega_{\mu\nu}\frac{\widehat{J}^{\mu\nu}}{2}\right)
\end{align}
\begin{spanish}
  Para encontrar una representación matricial de los boosts y las rotaciones usuales, 
  consideremos un boost
\end{spanish}
\begin{english}
  To find a representation of the usual boosts and rotations, 
consider a boost
\end{english}
\begin{equation}
  \left\{x^\mu\right\}=\begin{pmatrix}
    t\\
    x\\
    y\\
    z
  \end{pmatrix}\to
  \begin{pmatrix}
    t'\\
    x'\\
    y'\\
    z'
  \end{pmatrix}=
  \begin{pmatrix}
    \frac{t+vx}{\sqrt{1-v^2}}\\
    \frac{x+vt}{\sqrt{1-v^2}}\\
    y\\
    z
  \end{pmatrix}=
  \begin{pmatrix}
    \cosh\xi&\sinh\xi&0&0\\
    \sinh\xi&\cosh\xi&0&0\\
    0     &  0  &1&0\\
    0     &  0  &0&1
  \end{pmatrix}
  \begin{pmatrix}
    t\\
    x\\
    y\\
    z
  \end{pmatrix}=\left\{{\Lambda^\mu}_{\nu}\right\}\left\{x^\nu\right\},
\end{equation}
\begin{spanish}
  Ya que
\end{spanish}
\begin{english}
  Since
\end{english}
\begin{align}
  \cosh\xi=&\sum_{n=0}^{\infty}\frac{\xi^{2n}}{2n!}\approx 1+\mathcal{O}(\xi^2)\nonumber\\
  \sinh\xi=&\sum_{n=0}^{\infty}\frac{\xi^{2n+1}}{(2n+1)!}\approx \xi+\mathcal{O}(\xi^2)\,,
\end{align}
\begin{spanish}
  Un boost infinitesimal a lo largo de $x$ es
\end{spanish}
\begin{english}
  one infinitesimal boost along $x$ is
\end{english}
\begin{align}
  \left\{{\Lambda^\mu}_{\nu}\right\}_{x-\text{boost }}\approx
  \begin{pmatrix}
    1&\xi&0&0\\
    \xi&1&0&0\\
    0&0&1&0\\
    0&0&0&1
  \end{pmatrix}=\exp \left( i\xi  K \right),
\end{align}
donde el generador de boost es
\begin{align}
 K= \begin{pmatrix}
    0 & -i & 0 & 0\\
   -i & 0  & 0 & 0\\
   0 & 0 &  0 & 0\\
    0 & 0 &  0 & 0\\ 
  \end{pmatrix}.
\end{align}

\begin{spanish}
  Similarmente una rotación por un ángulo infinitesimal $\theta=\theta_3$ alrededor del plano $xy$ (o sobre el eje $z$)
\end{spanish}
\begin{english}
  Similarly a rotation by an infinitesimal angle $\theta=\theta_3$ along $xy$--plane (or about the $z$--axis)
\end{english}
\begin{align}
  \left\{{\Lambda^\mu}_{\nu}\right\}_{xy-\text{rotation }}\approx
  \begin{pmatrix}
    1&0&0&0\\
    0&1&\theta&0\\
    0&-\theta&1&0\\
    0&0&0&1
  \end{pmatrix}.
\end{align}
Que como hemos visto, puede obtenerse a partir de los generadores del Grupo de rotaciones $SO(3)$, generalizados a matrices $4\times4$
\begin{align}
  \{L^{i}\}\equiv
  \begin{pmatrix}
    1 & 0 & 0& 0\\
    0 &   &  &  \\
    0 &   & L^i_{3\times3}  &  \\
    0 &   &  &  \\
  \end{pmatrix}
\end{align}

\begin{align*}
  L^1_{3\times3}=&
  \begin{pmatrix}
   0 & 0 & 0\\
   0 & 0 & -i\\
   0 & i & 0 \\
  \end{pmatrix}&
 L^2_{3\times3}=&
 \begin{pmatrix}
  0 & 0  & i \\ 
  0 & 0  & 0 \\
 -i & 0  & 0 \\
 \end{pmatrix}&
 L^3_{3\times3}=&
 \begin{pmatrix}
   0 & -i & 0\\
   i & 0  & 0\\
   0 & 0 & 0\\
 \end{pmatrix}
\end{align*}.

In general we define the six independent Lorentz--Group parameters:
\begin{align}
  \omega_{i0}=-\omega_{0i}\equiv&\xi_{i} \nonumber\\
  \omega_{12}=-\omega_{21}\equiv&2\theta^3 &   \omega_{32}=-\omega_{23}\equiv&-2\theta^2 &   \omega_{13}=-\omega_{31}\equiv&2\theta^1\,.
\end{align}
Por lo tanto
\begin{align}
\xi^i=&\omega^{i0}=-\omega^{0i}&\theta^i=&\frac{1}{2}\epsilon^{ijk}\omega_{jk}\,.  
\end{align}

The $4\times 4$ matrices
\begin{align}
   \left(J^{\mu\nu}\right)_{\alpha\beta}=&i\epsilon^{\mu\nu\rho\sigma}\epsilon_{\rho\sigma\alpha\beta}\nonumber\\
   \left(J^{\mu\nu}\right)^{\alpha}_{\beta}=g^{\gamma\alpha}\left(J^{\mu\nu}\right)_{\gamma\beta}= &i\epsilon^{\mu\nu\rho\sigma}\epsilon_{\rho\sigma\gamma\beta}g^{\gamma\alpha}\nonumber\\
\end{align}
\begin{align}
  \left(J^{\mu\nu}\right)_{\alpha\beta}  =&i\left({\delta^\mu}_\alpha{\delta^\nu}_\beta-{\delta^\mu}_\beta{\delta^\nu}_\alpha\right)\nonumber\\
 {\left(J^{\mu\nu}\right)^{\alpha}}_{\beta}=&ig^{\gamma\alpha}\left({\delta^{\mu}}_{\gamma}{\delta^\nu}_\beta-{\delta^\mu}_\beta{\delta^{\nu}}_\gamma\right) \nonumber\\
 =&i\left(g^{\mu\alpha}{\delta^\nu}_\beta-{\delta^\mu}_\beta g^{\nu\alpha}\right)\nonumber\\
{\left(J^{\mu\nu}\right)^{\alpha\beta}}=&i \left( g^{\mu\alpha}g^{\nu\beta}-g^{\mu\beta}g^{\nu\alpha} \right)
\end{align}
where $\mu$ and $\nu$ label which of the six matrices we want, while $\alpha$ and $\beta$ label components of the matrices. These matrices satisfy the commutations relations \eqref{eq:lrtalg}, and generate the three boosts and three rotations of the ordinary Lorentz 4-vectors:
\begin{align}
  {\Lambda^\alpha}_\beta\approx{\delta^\alpha}_\beta-\frac{i}{2}\omega_{\mu\nu}{\left(J^{\mu\nu}\right)^\alpha}_\beta\,.
\end{align}
%ver programa mathematica
En particular
\begin{align*}
  (J^{ij})_{lm}=&i\epsilon^{ij\rho\sigma}\epsilon_{\rho\sigma lm}\\
             =&i\epsilon^{ij\rho 0}\epsilon_{\rho 0 lm}\\
             =&-i\epsilon^{ijk}\epsilon_{klm}\\
             =&\epsilon^{ijk}(L_{k})_{lm}
\end{align*}
o, en términos matriciales
\begin{align}
  L^{i}=\tfrac{1}{2}\epsilon^{ijk}J_{jk}
\end{align}
De modo que
\begin{align}
\label{eq:klkl}
  i\sum_i \theta^{i}L^{i}=-&i\theta^iL_i\nonumber\\
=&- \frac{i}{2}\epsilon^{ikl}\omega_{kl}\frac{1}{2}\epsilon_{imn}J^{mn}\nonumber\\
=&- \frac{i}{4}(\delta^k_m\delta^l_n-\delta^k_n\delta^l_m)\omega_{kl}J^{mn}\nonumber\\
=&- \frac{i}{4}(\omega_{kl}J^{kl}-\omega_{kl}J^{lk})\nonumber\\
=&- \frac{i}{4}(\omega_{kl}J^{kl}+\omega_{kl}J^{kl})\nonumber\\
=&- \frac{i}{2}\omega_{kl}J^{kl}\,.
\end{align}

usando la notación de \cite{0812.1594}, definimos también
\begin{align}
  K^i\equiv J^{0i}=-J^{i0}\,,
\end{align}
Entonces
\begin{align}
-i\omega_{\mu\nu}\frac{\widehat{J}^{\mu\nu}}{2} =&-i\omega_{0\nu}\frac{\widehat{J}^{0\nu}}{2}
-i\omega_{i\nu}\frac{\widehat{J}^{i\nu}}{2}\nonumber\\
=&-i\omega_{0i}\frac{\widehat{J}^{0i}}{2}
-i\omega_{i0}\frac{\widehat{J}^{i0}}{2}
-i\omega_{ij}\frac{\widehat{J}^{ij}}{2}\nonumber\\
=&  -i\omega_{i0}\widehat{J}^{i0}
-i\omega_{ij}\frac{\widehat{J}^{ij}}{2}\nonumber\\
=&  i\omega_{i0}\widehat{J}^{0i}
-i\omega_{ij}\frac{\widehat{J}^{ij}}{2}\,,
\end{align}
y usando \eqref{eq:klkl}
\begin{align}
=&  -i\xi_{i}K^{i}-i\omega_{ij}\frac{\widehat{J}^{ij}}{2}\nonumber\\
=&\sum_i \left(i\xi^i K^i+i\theta^i L^i  \right)\,.
\end{align}
Entonces
\begin{align}
  \{\Lambda\}=\exp\left(-i\omega_{\mu\nu}\frac{\widehat{J}^{\mu\nu}}{2}\right)=
\exp\left( i\boldsymbol{\xi}\cdot\mathbf{K}+i\boldsymbol{\theta}\cdot\mathbf{L} \right)\,.
\end{align}


%ver programa mathematica

\end{frame}


\section{2 representaciones $2\times 2$ del Grupo de Lorentz}




\subsection{Lorentz transformation}

\begin{frame}[fragile,allowframebreaks]
Here we focus on the simplest non-trivial irreducible representations of the Lorentz algebra. These are the two-dimensional (inequivalent) representations: $(\frac{1}{2},0)$ and $(0,\frac{1}{2})$. 

\begin{align}
  S(\Lambda)_{\left( \frac{1}{2},0 \right)}=&\exp\left(-i \omega_{\mu\nu}\frac{\sigma^{\mu\nu}}{2}\right)\nonumber
\end{align}
\begin{align}
  \sigma^{\mu\nu}=\frac{i}{4}\left[\sigma^\mu,\overline{\sigma}^\nu\right]\,.
\end{align}
where
\begin{align}
  \sigma^0=&\mathbf{1}_{2\times2},& \sigma^{i}\to \boldsymbol{\sigma}=&(\sigma^1,\sigma^2,\sigma^3)\nonumber\\
  \overline{\sigma}^0=&\mathbf{1}_{2\times2},& \overline{\sigma}^{i}\to \overline{\boldsymbol{\sigma}}=&(-\sigma^1,-\sigma^2,-\sigma^3)
\end{align}
include the Pauli matrices \eqref{eq:paulimatr}.

\end{frame}
\begin{frame}[fragile,allowframebreaks]
Eq. \eqref{eq:4lt}, which yields
\begin{align}
\label{eq:SLet}
  S(\Lambda)_{\left( \frac{1}{2},0 \right)}\equiv S(\Lambda)=
\exp\left( \boldsymbol{\xi}\cdot \frac{\boldsymbol{\sigma}}{2}+i\boldsymbol{\theta}\cdot \frac{\boldsymbol{\sigma}}{2} \right)\,,
\end{align}

La otra representación independiente es
\begin{align}
\label{eq:SLet}
\left[S(\Lambda)\right]^{\dagger} = S(\Lambda)_{\left(0,\frac{1}{2} \right)}\equiv S(\Lambda^{*})=&
\exp\left( \boldsymbol{\xi}\cdot \frac{\boldsymbol{\sigma}}{2}+i\boldsymbol{\theta}\cdot \frac{\boldsymbol{\sigma}}{2} \right)^{\dagger}\nonumber\\
=&\exp\left( \boldsymbol{\xi}\cdot \frac{\boldsymbol{\sigma}^{*}}{2}-i\boldsymbol{\theta}\cdot \frac{\boldsymbol{\sigma}^{*}}{2} \right)\,.
\end{align}
 
The components of $S(\Lambda)$ will be denoted as ${\left[ S(\Lambda) \right]_{\alpha}}^{\beta}$. In such a case, $S^{*}$ is another independent $2\times2$ representation of the Lorentz Group. It is denoted by $\left( 0,\frac{1}{2}\right)$, and, in order to emphasize the difference, it is convenient to denote their components with dotted indices $\dot{\alpha},\dot{\beta},\cdots$. We can get $S^{*}$ from $S^{\dagger}$.







In summary we have the following Lorentz's transformation properties for the fields
\begin{align}
   \phi(x)\to \phi'(x')=&\phi(x) && \text{Scalar field,}\nonumber\\
   A^\mu(x)\to {A'}^\mu=&{\Lambda^\mu}_\nu A^\nu(\Lambda^{-1}x)&&\text{Vector field,}\nonumber\\
  \psi_\alpha(x)\to\psi'_\alpha(x)=&{\left[ S(\Lambda) \right]_\alpha}^\beta\psi_\beta(\Lambda^{-1}x)
&& \text{Left-handed spinor field,}\nonumber\\
 {\psi'}_{\dot{\alpha}}^{\dagger}\to  {\psi'}_{\dot{\alpha}}^{\dagger}=&{\left[ S^*(\Lambda)\right]_{\dot{\alpha}}}^{\dot{\beta}} \psi_{\dot{\beta}}^{\dagger}&& \text{Rigth-handed anti-spinor field,}\,,
 \end{align}

\end{frame}

we could need a new representation acting in and internal space upon a
two-component field $\psi_a$ ($a=1,2$). An Action with a Lagrangian term linear in the derivatives, could be Lorentz invariant if, taking into account the convention in eq.~(\ref{eq:conven}) and the dotted-undotted structure of the tensor in this internal space, we have that if $a^{\mu}$ is to be a 2th rank tensor of the internal space, it must have components, e.g, $\left( a^{\mu} \right)^{\dot{\alpha}\beta}$. Therefore, a posible Lorentz invariant with a single derivative could be 
 \begin{align}
   {\psi}^{\dagger}a^\mu\partial_\mu\psi\to  {\psi'}^{\dagger}(x)a^\mu\partial'_\mu\psi'
&={\psi'}^{\dagger}_{\dot{\alpha}}a^{\mu\dot{\alpha}\gamma}{\left(\Lambda^{-1}\right)^\rho}_\mu\partial_\rho \psi'_{\gamma}\,,
\end{align}
with the first letters of the Greek alphabet are used to denote the indices of the internal Lorentz space, and the others the external one. 
\begin{align}
{\psi}^{\dagger}a^\mu\partial_\mu\psi\to  {\psi'}^{\dagger}(x)a^\mu\partial_\mu\psi'
&=
 {S^*_{\dot{\alpha}}}^{\dot{\beta}}{\psi}^{\dagger}_{\dot{\beta}}a^{\mu\dot{\alpha\gamma}}{\left(\Lambda^{-1}\right)^\rho}_\mu\partial_\rho \left( {S_{\gamma}}^{\delta}\psi_\delta \right)\,.
 \end{align} 
As the coordinates $\eta_i$ and $\theta_i$ in eq.~\eqref{eq:SLet} are in the internal Lorentz space, the corresponding Lorentz transformation is constant in the external Lorentz space and
\begin{align}
  {\psi}^{\dagger}a^\mu\partial_\mu\psi\to  {\psi'}^{\dagger}(x)a^\mu\partial_\mu\psi'&=
{\psi}^{\dagger}_{\dot{\beta}}{\left(\Lambda^{-1}\right)^\rho}_\mu {S^{\dagger\dot{\beta}}}_{\dot{\alpha}}a^{\mu\dot{\alpha\gamma}}{S_{\gamma}}^{\delta} \partial_\rho  \psi_\delta \nonumber\\
&=\psi^\dagger {\left(\Lambda^{-1}\right)^\rho}_\mu \left(S^\dagger a^\mu S\right)\partial_\rho\psi\nonumber\\
&=\psi^{\dagger}a^\rho\partial_\rho\psi\,,
\end{align}
if the following condition is satisfied:
\begin{align}
 {\left(\Lambda^{-1}\right)^\rho}_\mu  S^{\dagger}a^\mu S=a^\rho\,,
\end{align}
or
\begin{align}
\label{eq:ltrincond}
{\left(\Lambda\right)^\nu}_\rho{\left(\Lambda^{-1}\right)^\rho}_\mu   S^{\dagger}a^\mu S=&
{\left(\Lambda\right)^\nu}_\rho a^\rho\,,\nonumber\\
\delta^{\nu}_{\mu}   S^{\dagger}a^\mu S=&
{\left(\Lambda\right)^\nu}_\rho a^\rho \nonumber\\
S^{\dagger}a^\nu S=&{\left(\Lambda\right)^\nu}_\rho a^\rho\,,
\end{align}
At this point we could show that $2\times2$ matrices $a^{\mu}$ are in fact the Pauli matrices plus the identity by using the internal Lorentz transformation. However we will postpone this demonstration and, as a self consistency check, we will obtain the structure of $a^{\mu}$ from the consequences of the Noether's theorem. 

The other possibility to have a Lorentz  vector  is the combination ${\psi}b^\mu\partial_\mu\psi^{\dagger}$, but it can be shown to be equivalent to the original $\psi^{\dagger}a^\mu\partial_\mu\psi$ with some specific relation between $a^{\mu}$ and $b^{\mu}$. 

Other possible combinations like $\psi^{\dagger}\psi$ or $\psi a^\mu\partial_\mu\psi$ or $\psi^{\dagger} a^\mu\partial_\mu\psi^{\dagger}$ do not have the proper internal Lorentz index structure. 

Therefore  the most general Lagrangian for two-component spinors is
\begin{align*}
  \mathcal{L}=\frac{i}{2}{\psi}^{\dagger}a^\mu\partial_\mu\psi+m\psi\psi+
\left(\frac{i}{2} {\psi}^{\dagger}a^\mu\partial_\mu\psi \right)^{\dagger}+m \left(\psi\psi  \right)^{\dagger}\,,
\end{align*}
where the last two terms guarantee that $\mathcal{L}^{\dagger}=\mathcal{L}$ so that the Action be real, and the coefficients have been choosing before hand to give the proper equations of motion. Then,
\begin{align*}
  \mathcal{L}=&\frac{i}{2}{\psi}^{\dagger}a^\mu\partial_\mu\psi-\frac{i}{2} \partial_\mu\psi^{\dagger}{a^\mu}^{\dagger}\psi+m \left( \psi\psi+\psi^{\dagger}\psi^{\dagger} \right)
\end{align*}
If
\begin{align}
{a^{\mu}}^{\dagger}=a^{\mu}  
\end{align}
as expected to the going to be Pauli matrices, then
\begin{align}
\mathcal{L}=&\frac{i}{2}{\psi}^{\dagger}a^\mu\partial_\mu\psi-\frac{i}{2} \partial_\mu \left(  \psi^{\dagger} a^\mu\psi\right)
+\frac{i}{2}{\psi}^{\dagger}a^\mu\partial_\mu\psi
+m \left( \psi\psi+\psi^{\dagger}\psi^{\dagger} \right)\,,
\end{align}
and dropping out the total derivative, we have finally the most general Action for two-component spinors:
\begin{align}
  \mathcal{L}=&i{\psi}^{\dagger}a^\mu\partial_\mu\psi+
m \left( \psi\psi+\psi^{\dagger}\psi^{\dagger} \right)\,.
\end{align}
If $\psi$ have a continuos charge such that
\begin{align}
  \psi\to\psi'=e^{i\alpha}\psi
\end{align}
we can impose that the Lagrangian be invariant under changes of phase of $\psi$. In such a case the mass of the field must be zero and
\begin{align}
   \mathcal{L}=&i{\psi}^{\dagger}a^\mu\partial_\mu\psi.
\end{align}
The previos Lagrangian which is invariant under
\begin{align}
  \psi\to \psi'=e^{i\alpha}\psi\,,
\end{align}
is the most general one if $\psi$ have any conserved charge, and will be the one the will use in the subsequent discussions.



\subsection{Corriente conservada y Lagrangiano de Weyl}
\label{sec:corriente-conservada}
De la ec.~\eqref{eq:jmuphi}
\begin{align}
  J^0&=\left[\frac{\partial\mathcal{L}}{\partial\left(\partial_0\psi\right)}\right]\delta\psi+\delta{\psi}^{\dagger}\left[\frac{\partial\mathcal{L}}{\partial\left(\partial_0{\psi}^\dagger\right)}\right]
\end{align}
El Lagrangiano es invariante bajo transformaciones de fase globales, $U(1)$
\begin{equation}
  \psi\to\psi'=e^{-i\alpha}\psi\approx\psi-i\alpha\psi,
\end{equation}
de modo que
\begin{equation}
  \delta\psi=-i\alpha\psi.
\end{equation}
Por consiguiente
\begin{align}
  J^0=&i\psi^{\dagger}a^0 (-i\alpha\psi)+0 \nonumber\\
     =&\alpha{\psi}^{\dagger}a^0\psi 
\end{align}
La  densidad de corriente es
\begin{align}
  J^0&\propto \psi^\dagger a^0\psi\,.
\end{align}
Que podemos interpretar como una densidad de probabilidad $\psi^{\dagger}\psi$ si
\begin{align}
  a^0=\mathbf{1}_{2\times2}\,.
\end{align}

En general
\begin{align}
   J^\mu&\propto\left[\frac{\partial\mathcal{L}}{\partial\left(\partial_\mu\psi\right)}\right]\delta\psi+\delta\psi^\dagger\left[\frac{\partial\mathcal{L}}{\partial\left(\partial_\mu\psi^\dagger\right)}\right]\nonumber\\
   &\propto i\psi^\dagger a^\mu(-i\alpha\psi)\nonumber\\
   &\propto i\psi^\dagger a^\mu(-i\alpha\psi)\nonumber\\
   &=\psi^\dagger a^\mu\psi
\end{align}
y
\begin{equation}
     J^\mu=\psi^\dagger  a^\mu\psi\,.
\end{equation}
donde $a^{\mu}$ es el cuadrivector de matrices
\begin{align}
  a^{\mu}=\left(\mathbf{1},a^{1},a^{2},a^{3}\right)\,,
\end{align}
y $a^i$ están aún por determinar
\subsection{Tensor momento-energía}
\label{sec:tens-momento-energi}
Usando $a^{0}=\mathbf{1}$,
\begin{align}
  T^0_0&=\frac{\partial\mathcal{L}}{\partial\left(\partial_0\psi\right)}\partial_0\psi+\partial_0\psi^\dagger\frac{\partial\mathcal{L}}{\partial\left(\partial_0\psi^\dagger\right)}-\mathcal{L}\nonumber\\
  &=i\psi^\dagger\partial_0\psi-\mathcal{L}\nonumber\\
  &=-i\psi^\dagger a^i\partial_i\psi\nonumber\\
  &=\psi^\dagger a^i \left( -i \partial_i\right)\psi\nonumber\\
  &=\psi^\dagger(\mathbf{a}\cdot\mathbf{p})\psi,\nonumber\\
  \label{eq:118qft}
  &=\psi^\dagger\hat{H} \psi,
\end{align}
donde
\begin{equation}
  \label{eq:denshal}
  \hat{H}= \mathbf{a}\cdot\mathbf{p}
\end{equation}
la ecuación de Scröndinger de validez general es entonces:
\begin{equation}
  i\frac{\partial}{\partial t}\psi=\hat{H} \psi
\end{equation}
y, como en mecánica clásica usual
\begin{equation}
  \label{eq:99qft}
  \langle\hat{H}\rangle=\int \psi^\dagger\hat{H} \psi\,d^3x.
\end{equation}


Además
\begin{align}
    T^0_i&=\frac{\partial\mathcal{L}}{\partial\left(\partial_0\psi\right)}\partial_i\psi+\partial_i\psi^\dagger\frac{\partial\mathcal{L}}{\partial\left(\partial_0\psi^\dagger\right)}\nonumber\\
    &=i\psi^\dagger\partial_i\psi\nonumber\\
    &=-\psi^\dagger(-i\partial_i)\psi
\end{align}
de modo que
\begin{equation}
  \langle\hat{\mathbf{p}}\rangle=\int\psi^\dagger\hat{\mathbf{p}}\psi\,d^3 x
\end{equation}
\subsection{Ecuaciones de Euler-Lagrange}
\label{sec:ecuaciones-de-euler}
Queremos que el Lagrangiano de lugar a la ecuación de Scröndinger de validez general
\begin{equation}
  \label{eq:grlsch}
  i\frac{\partial}{\partial t}\psi=\hat{H} \psi
\end{equation}
con el Hamiltoniano dado en la ec.~(\ref{eq:99qft}), que corresponde a un Lagrangiano de sólo derivadas de primer orden y covariante, en lugar del Hamiltoniano para el caso no relativista. 

De hecho, aplicando las ecuaciones de Euler-Lagrange para el campo $\psi^\dagger$ al Lagrangiano en ec.~(\ref{eq:100qft}) ,tenemos
\begin{align}
  \partial_\mu\left[\frac{\partial\mathcal{L}}{\partial\left(\partial_\mu\psi^\dagger\right)}\right]-\frac{\partial\mathcal{L}}{\partial\psi^\dagger}&=0\nonumber\\
  \frac{\partial\mathcal{L}}{\partial\psi^\dagger}&=0\nonumber\\
  \label{eq:114qftm}
  i a^\mu\partial_\mu\psi&=0.
\end{align}
Expandiendo
\begin{align*}
  i a^0\partial_0\psi+i a^i\partial_i\psi&=0\\
  i a^0\partial_0\psi-\boldsymbol{a}\cdot(-i\boldsymbol{\nabla})\psi&=0,\\
  i a^0\partial_0\psi&=(\boldsymbol{a}\cdot\hat{\mathbf{p}})\psi,
\end{align*}
Como $a^{0}=\mathbf{1}$,
\begin{equation}
    i\frac{\partial}{\partial t}\psi=\boldsymbol{a}\cdot\mathbf{p}\psi.
\end{equation}
De la ec.~(\ref{eq:denshal})
\begin{equation}
  \label{eq:186qft}
  \hat{H}= \boldsymbol{a}\cdot\mathbf{p},
\end{equation}
A este punto, sólo nos queda por determinar los parámetros $a^\mu$. 

La ec.~(\ref{eq:grlsch}) puede escribirse como
\begin{equation}
  \left(i\frac{\partial}{\partial t}-\hat{H}\right)\psi=0.
\end{equation}
El campo $\psi$ también debe satisfacer la ecuación de Klein-Gordon. Podemos derivar dicha ecuación aplicando el operador
\begin{equation*}
  \left(-i\frac{\partial}{\partial t}-\hat{H}\right)
\end{equation*}
De modo que, teniendo en cuenta que $\partial\hat H/\partial t=0$,
\begin{align}
  \label{eq:105qft}
 \left(-i\frac{\partial}{\partial t}-\hat{H}\right)\left(i\frac{\partial}{\partial t}-\hat{H}\right)\psi&=0\nonumber\\
 \left(-i\frac{\partial}{\partial t}-\hat{H}\right)\left(i\frac{\partial\psi}{\partial t}-\hat{H}\psi\right)&=0\nonumber\\
 \frac{\partial^2\psi}{\partial t^2}+i\left(\frac{\partial\hat{H}}{\partial t}\right)\psi
 +i\hat{H}\frac{\partial\psi}{\partial t}-i\hat{H}\frac{\partial\psi}{\partial t}+\hat{H}^2\psi&=0\nonumber\\
 \left(\frac{\partial^2}{\partial t^2}+\hat{H}^2\right)\psi&=0.
\end{align}
% 
De la ec.~(\ref{eq:186qft}), y usando la condición en ec.~(\ref{eq:gamma02}), tenemos
\begin{align}
\label{eq:106qft}
\hat{H}^2&=(\boldsymbol{a}\cdot\mathbf{p})(\boldsymbol{a}\cdot\mathbf{p})\,.
\end{align}

Sea $A$ una matriz y $\theta$ en un escalar. Entonces tenemos la identidad
\begin{align}
  \label{eq:206qft}
  (\mathbf{A}\cdot\boldsymbol{\theta})^2=\sum_i {A^i}^2 {\theta^i}^2+\sum_{i\lt j}\left\{A^i,A^j  \right\}\theta^i \theta^j 
\end{align}
\begin{itemize}
\item \textbf{Demostración}
  \begin{align}
    \left[\left(\mathbf{A}\cdot\boldsymbol{\theta}\right)\right]_{\alpha\beta}
    =&\sum_{i j}\sum_\gamma A^i_{\alpha\gamma}\theta^iA^j_{\gamma\beta}\theta^j\nonumber\\    
    =&\sum_{i j}\theta^i\theta^j\sum_\gamma A^i_{\alpha\gamma}A^j_{\gamma\beta}\nonumber\\    
    =&\sum_\gamma \sum_{i j}\theta^i\theta^jA^i_{\alpha\gamma}A^j_{\gamma\beta}\nonumber\\    
    =&\sum_\gamma \left(\sum_{i}{\theta^i}^2A^i_{\alpha\gamma}A^i_{\gamma\beta}+\sum_{i<j}\theta^i\theta^jA^i_{\alpha\gamma}A^j_{\gamma\beta}+\sum_{i>j}\theta^i\theta^jA^i_{\alpha\gamma}A^j_{\gamma\beta}\right)\nonumber\\    
    =&\sum_\gamma \left(\sum_{i}{\theta^i}^2A^i_{\alpha\gamma}A^i_{\gamma\beta}+\sum_{i<j}\theta^i\theta^jA^i_{\alpha\gamma}A^j_{\gamma\beta}+\sum_{j>i}\theta^j\theta^iA^j_{\alpha\gamma}A^i_{\gamma\beta}\right)\nonumber\\    
    =&\sum_\gamma \left[\sum_{i}{\theta^i}^2A^i_{\alpha\gamma}A^i_{\gamma\beta}+\sum_{i<j}\theta^i\theta^j\left(A^i_{\alpha\gamma}A^j_{\gamma\beta}+A^j_{\alpha\gamma}A^i_{\gamma\beta}\right)\right]\nonumber\\    
    =&\left[\sum_{i}{\theta^i}^2\left(A^iA^i\right)_{\alpha\beta}+\sum_{i<j}\theta^i\theta^j\left\{ A^i,A^j\right\}_{\alpha\beta}\right]\nonumber\\    
    =&\left[\sum_{i}{\theta^i}^2{A^i}^2+\sum_{i<j}\theta^i\theta^j\left\{ A^i,A^j\right\}\right]_{\alpha\beta}\,.
  \end{align}

\end{itemize}
Entonces
\begin{align}
  \hat{H}^2=& a_i^2p_i^2+\sum_{i\lt j}\left\{ a_i, a_j\right\}p_i p_j
\end{align}
(suma sobre índices repetidos). Si
\begin{align}
  \label{eq:107qft}
   a_i^2&=\mathbf{1}\nonumber\\
  \left\{ a_i, a_j\right\}&=0\qquad i\ne j\,.
\end{align}
que se puede resumir en
\begin{align}
  \left\{ a^i,a^j \right\}=&2\delta_{ij} \mathbf{1}\,.
\end{align}
una de las propiedas de las matrices de Pauli en  \eqref{eq:64qft}. De esta forma, podemos identificar
\begin{align}
  a^{i}=\pm \sigma^{i}\,.
\end{align}
de modo que
\begin{equation}
  \hat{H}^2=-\boldsymbol{\nabla}^2\,,
\end{equation}
y reemplazando en la ec.~\eqref{eq:105qft} llegamos a la ecuación de Klein-Gordon para $\psi$
\begin{align}
   \left(\frac{\partial^2}{\partial t^2}-\boldsymbol{\nabla}^2\right)\psi&=0\nonumber\\
   \Box\psi&=0
\end{align}
Debido a la ambigüedad  en el signo, podemos construir dos cuadrivectores independientes
   \begin{align}
 \sigma^{\mu}=& \left( \mathbf{1}_{2\times2},\boldsymbol{\sigma} \right)&
 \overline{\sigma}^{\mu}=& \left( \mathbf{1}_{2\times2},\overline{\boldsymbol{\sigma}} \right)
\end{align}
donde
\begin{align}
  \overline{\boldsymbol{\sigma}}=-\boldsymbol{\sigma}=\left(-\sigma^1,-\sigma^2,-\sigma^3\right)\,.
\end{align}
Como veremos luego, las componentes en el espacio interno son
$\sigma^{\mu}_{\alpha\dot{\alpha}}$ y $\overline{\sigma}^{\mu\;\alpha\dot{\alpha}}$, de modo que  las matrices apropiadas son $\overline{\sigma}^\mu$, y el Lagrangiano  y la ecuación de Weyl, son respectivamente de las ecs.~(\ref{eq:100qft}) y (\ref{eq:114qft})
\begin{align}
  \label{eq:115qft}
  \mathcal{L}=&i\psi^\dagger\overline{\sigma}^\mu\partial_\mu\psi \nonumber\\
      =&i\psi^\dagger_{\dot{\alpha}}\overline{\sigma}^{\mu\; \alpha\dot{\alpha}}\partial_\mu\psi_{\alpha}\,,
\end{align}
que da lugar a la ecuación de movimiento
\begin{equation}
  \label{eq:116qft}
  i\overline{\sigma}^\mu\partial_\mu\psi=0,
\end{equation}
Si $\psi$ no tiene ninguna carga continua se puede adicionar un término de masa (con su correspondiente hermítico conjugado)
\begin{align}
  \mathcal{L}=& i\psi^\dagger\overline{\sigma}^\mu\partial_\mu\psi +m \left( \psi\psi+\psi^{\dagger}\psi^{\dagger} \right)\,.
\end{align}

\subsection*{Ejercicio}

Considere un espinor de Weyl que además tiene una carga continua $U(1)$
  \begin{align}
    \psi\to \psi'=e^{i\alpha}\psi\,,
  \end{align}
de modo que su término de masa esta prohíbido.

Escriba el Lagrangiano más general posible para el campo $\psi$ y un campo escalar complejo $\phi$ que transforma como
\begin{align}
  \phi\to \phi'=e^{-2i\alpha}\phi\,.
\end{align}
con un potencial escalar $V(\phi)$ con ruptura espontánea de la simetría continua $U(1)$. Especifique explícitamente las dimensiones y el carácter positivo o negativo de cada uno de los coeficientes. 

Ayuda: sólo hay un término de interacción posible entre los campos fermiónicos y el campo escalar (más el correspondiente hermítico conjugado).

\hrulefill{}$\blacksquare$

\subsection{Ejercicio}

Para el Lagrangiano del punto anterior, encuentre el espectro de partículas cuando la parte real del campo escalar complejo adquiera un valor esperado de vacío, $v$:
  \begin{align*}
    \phi=\frac{H(x)+v+i J(x)}{\sqrt{2}}
  \end{align*}
y especifique cuales son los campos masivos y cuales quedan sin masas. Note entonces que un fermión sin masa también puede adquirir una masa después de la ruptura espontánea de la simetría, como consecuencia de su interacción con un campo escalar.
\label{item:ssb}

\hrulefill{}$\blacksquare$


\subsection{Lorentz invariance of the Weyl Action}
To show that $S(\Lambda)$ is in fact a Lorentz transformation, it is convinient to write this in covariant form. If we define
\begin{align}
  \sigma^{\mu\nu}=\frac{i}{4}\left[\sigma^\mu,\overline{\sigma}^\nu\right]\,.
\end{align}
We can obtain the proper boost and rotations generators:
\begin{align*}
 \mathbf{K}= \sigma^{0i}=&-i\frac{\sigma}{2}\nonumber\\
 L_{i}=\frac{1}{2}\epsilon_{ijk}\sigma^{jk}=&-4\frac{i}{8}\epsilon_{ijk}\left[\frac{\sigma^j}{2},\frac{\sigma^k}{2}  \right]\nonumber\\
=&-\tfrac{i}{2}\epsilon_{ijk}i\epsilon^{jkl}\frac{\sigma_l}{2}\nonumber\\
=&\tfrac{1}{2}\delta_i^l\sigma_l\nonumber\\
=&\tfrac{1}{2}\sigma_i\,.
\end{align*}
% In fact, the six set of non-zero independently generators are
% \begin{align}
%   \mathcal{S}^{0i}=&\frac{i}{4}\left(\gamma^0\gamma^i-\gamma^i\gamma^0\right)=\frac{i}{2}\gamma^0\gamma^i= i B^i\nonumber\\
%   \mathcal{S}^{i j}=&\frac{i}{4}\left(\gamma^i\gamma^j-\gamma^j\gamma^i\right)=\frac{i}{2}\gamma^i\gamma^j= i R^{i j}\,.
% \end{align}


\begin{english}
  We need to satisfy the following condition
\end{english}
\begin{spanish}
  Necesitamos satifacer la siguiente condición
\end{spanish}
\begin{align}
\label{eq:sss}
  S^\dagger\overline{\sigma}^\mu S=&{\Lambda^\mu}_\nu\overline{\sigma}^\nu
\end{align}
Ahora
\begin{align}
\label{eq:SLet}
  S(\Lambda)_{\left( \frac{1}{2},0 \right)}\equiv S(\Lambda)=
\exp\left( \boldsymbol{\xi}\cdot \frac{\boldsymbol{\sigma}}{2}+i\boldsymbol{\theta}\cdot \frac{\boldsymbol{\sigma}}{2} \right)\,,
\end{align}
y expandiendo \eqref{eq:sss}
\begin{align*}
\left(\mathbf{1}+\boldsymbol{\xi}\cdot \frac{\boldsymbol{\sigma}}{2} -i\boldsymbol{\theta}\cdot \frac{\boldsymbol{\sigma}}{2}  \right)
\overline{\sigma}^{\mu}
\left(\mathbf{1}+\boldsymbol{\xi}\cdot \frac{\boldsymbol{\sigma}}{2} +i\boldsymbol{\theta}\cdot \frac{\boldsymbol{\sigma}}{2}  \right)
=&\left[ \mathbf{1}+i\boldsymbol{\xi}\cdot \mathbf{K}+i\boldsymbol{\theta}\cdot \mathbf{L} \right]^{\mu}_{\ \nu}\overline{\sigma}^\nu \nonumber\\
\left(\overline{\sigma}^{\mu}+\boldsymbol{\xi}\cdot \frac{\boldsymbol{\sigma}}{2}\overline{\sigma}^{\mu} -i\boldsymbol{\theta}\cdot \frac{\boldsymbol{\sigma}}{2}\overline{\sigma}^{\mu}  \right)
\left(\mathbf{1}+\boldsymbol{\xi}\cdot \frac{\boldsymbol{\sigma}}{2} +i\boldsymbol{\theta}\cdot \frac{\boldsymbol{\sigma}}{2}  \right)
=&\left[ \mathbf{1}+i\boldsymbol{\xi}\cdot \mathbf{K}+i\boldsymbol{\theta}\cdot \mathbf{L} \right]^{\mu}_{\ \nu}\overline{\sigma}^\nu \,.
\end{align*}
Hasta primer orden en los parametros $\xi^i$ y $\theta^i$,
\begin{align*}
 \overline{\sigma}^{\mu}+\boldsymbol{\xi}\cdot \left(\overline{\sigma}^{\mu} \frac{\boldsymbol{\sigma}}{2} \right)  +i\boldsymbol{\theta}\cdot \left(\overline{\sigma}^{\mu} \frac{\boldsymbol{\sigma}}{2} \right)+\boldsymbol{\xi}\cdot \frac{\boldsymbol{\sigma}}{2}\overline{\sigma}^{\mu} -i\boldsymbol{\theta}\cdot \frac{\boldsymbol{\sigma}}{2}\overline{\sigma}^{\mu}
 =&\delta^{\mu}_{\nu}\overline{\sigma}^\nu+i\boldsymbol{\xi}\cdot {\mathbf{K}^{\mu}}_{\nu}\overline{\sigma}^\nu+i\boldsymbol{\theta}\cdot {\mathbf{L}^{\mu}}_{\nu}\overline{\sigma}^\nu \nonumber\\
   \overline{\sigma}^{\mu}+\boldsymbol{\xi}\cdot \left(\overline{\sigma}^{\mu}\frac{\boldsymbol{\sigma}}{2}+ \frac{\boldsymbol{\sigma}}{2}\overline{\sigma}^{\mu}\right)  +i\boldsymbol{\theta}\cdot \left(\overline{\sigma}^{\mu} \frac{\boldsymbol{\sigma}}{2}-\frac{\boldsymbol{\sigma}}{2}\overline{\sigma}^{\mu} \right)  
 =&\delta^{\mu}_{\nu}\overline{\sigma}^\nu+i\boldsymbol{\xi}\cdot {\mathbf{K}^{\mu}}_{\nu}\overline{\sigma}^\nu+i\boldsymbol{\theta}\cdot {\mathbf{L}^{\mu}}_{\nu}\overline{\sigma}^\nu \nonumber\\
  \boldsymbol{\xi}\cdot \left(\overline{\sigma}^{\mu}\frac{\boldsymbol{\sigma}}{2}+ \frac{\boldsymbol{\sigma}}{2}\overline{\sigma}^{\mu}\right)  +i\boldsymbol{\theta}\cdot \left(\overline{\sigma}^{\mu} \frac{\boldsymbol{\sigma}}{2}-\frac{\boldsymbol{\sigma}}{2}\overline{\sigma}^{\mu} \right)  
 =&i\boldsymbol{\xi}\cdot {\mathbf{K}^{\mu}}_{\nu}\overline{\sigma}^\nu+i\boldsymbol{\theta}\cdot {\mathbf{L}^{\mu}}_{\nu}\overline{\sigma}^\nu\,.
\end{align*}
Igualando coeficientes
\begin{align*}
\overline{\sigma}^{\mu}\frac{\boldsymbol{\sigma}}{2}+ \frac{\boldsymbol{\sigma}}{2}\overline{\sigma}^{\mu}  =&i{\mathbf{K}^{\mu}}_{\nu}\overline{\sigma}^\nu \nonumber\\
\overline{\sigma}^{\mu} \frac{\boldsymbol{\sigma}}{2}-\frac{\boldsymbol{\sigma}}{2}\overline{\sigma}^{\mu}=&
{\mathbf{L}^{\mu}}_{\nu}\overline{\sigma}^\nu
\end{align*}
La primera ecuación es
\begin{align*}
  \overline{\sigma}^{\mu}\frac{\sigma^i}{2} +\frac{\sigma^i}{2}\overline{\sigma}^{\mu}  =&i{\left[ K^i \right]^{\mu}}_{\nu}\overline{\sigma}^\nu \nonumber\\
 =&i{\left[ J^{0i} \right]^{\mu}}_{\nu}\overline{\sigma}^\nu \nonumber\\
  =&i{\left[ J^{0i} \right]^{\mu}}_{\nu}\overline{\sigma}^\nu \nonumber\\
  =&-\left(g^{0\mu}\delta^i_{\nu} -\delta^0_{\nu}g^{i\mu}  \right)\overline{\sigma}^\nu \nonumber\\
  =&-\left(g^{0\mu}\overline{\sigma}^i -g^{i\mu} \overline{\sigma}^0  \right)\,,
\end{align*}
para $\mu=0$
\begin{align*}
  \overline{\sigma}^{0}\frac{\sigma^i}{2} +\frac{\sigma^i}{2}\overline{\sigma}^{0}=&-\overline{\sigma}^i \nonumber\\
  \sigma^i=\sigma^i\,.
\end{align*}
Para $\mu=j$
\begin{align*}
  -\sigma^{j}\frac{\sigma^i}{2} -\frac{\sigma^i}{2}\sigma^j  =& +g^{ij}\sigma^0\nonumber\\
  -\delta^{ij}\mathbf{1}=-\delta^{ij}\mathbf{1}\,.
\end{align*}
La segunda ecuación es
\begin{align*}
\overline{\sigma}^{\mu} \frac{{\sigma^i}}{2}-\frac{{\sigma^i}}{2}\overline{\sigma}^{\mu}=&{\left(L^i  \right)^{\mu}}_{\nu}\overline{\sigma}^\nu \nonumber\\
 =&-{\left(L_i  \right)^{\mu}}_{\nu}\overline{\sigma}^\nu \nonumber\\
=&-\tfrac{1}{2}\epsilon_{ijk}{\left(J^{jk}  \right)^{\mu}}_{\nu}\overline{\sigma}^\nu \nonumber\\
 =&-\tfrac{i}{2}\epsilon_{ijk}\left(g^{j\mu}\delta^{k}_{\nu}-\delta^{j}_{\nu}g^{k\mu}  \right)\overline{\sigma}^\nu \nonumber\\
 =&-\tfrac{i}{2}\epsilon_{ijk}\left(g^{j\mu}\overline{\sigma}^k-g^{k\mu}\overline{\sigma}^j  \right) \nonumber\\
 =&\tfrac{i}{2}\epsilon_{ijk}\left(g^{j\mu}{\sigma}^k-g^{k\mu}{\sigma}^j  \right)\,.
\end{align*}
Para $\mu=0$
\begin{align*}
  \overline{\sigma}^{0} \frac{{\sigma^i}}{2}-\frac{{\sigma^i}}{2}\overline{\sigma}^{0}=& \frac{i}{2}\epsilon_{ijk}\left(g^{j0}{\sigma}^k-g^{k0}{\sigma}^j  \right)\nonumber\\
0=&0 \,.
\end{align*}
Para $\mu=l$
\begin{align*}
  \overline{\sigma}^l \frac{{\sigma^i}}{2}-\frac{{\sigma^i}}{2}\overline{\sigma}^l=&\frac{i}{2}\epsilon_{ijk}\left(g^{jl}{\sigma}^k-g^{kl}{\sigma}^j  \right)\nonumber\\
   \frac{{\sigma^i}}{2}{\sigma}^l -{\sigma}^l \frac{{\sigma^i}}{2}=&\frac{i}{2}\epsilon_{ijk}\left(-\delta^{jl}{\sigma}^k+\delta^{kl}{\sigma}^j  \right)\nonumber\\
  2\frac{\sigma^i}{2}\frac{\sigma^l}{2} -2\frac{\sigma^l}{2}\frac{\sigma^i}{2}=&\frac{i}{2}\left(-\epsilon_{ilk}{\sigma}^k+\epsilon_{ijl}{\sigma}^j  \right)\nonumber\\
 2\left[ \frac{\sigma^i}{2},\frac{\sigma^l}{2} \right]=&\frac{i}{2}\left(\epsilon_{lik}{\sigma}^k+\epsilon_{lik}{\sigma}^k  \right)\nonumber\\
 2i\epsilon_{lik}\frac{\sigma^{k}}{2}=&\frac{i}{2}\left(2\epsilon_{lik}  \sigma^{k}\right)\nonumber\\
 i\epsilon_{lik}{\sigma^{k}}=&i\epsilon_{lik}  \sigma^{k}\,.
\end{align*}





%
%%% Local Variables: 
%%% mode: latex
%%% TeX-master: "fullnotes"
%%% ispell-local-dictionary: "castellano8"
%%% End: 
