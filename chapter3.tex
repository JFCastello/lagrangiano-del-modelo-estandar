%instiki:category: FisicaSubatomica
\chapter{Escalares}
\label{cha:escalares}
Se mostrará como la invarianza de la Acción bajo transformaciones es el punto de partida en la construcción de densidades Lagrangianas únicas.

%\subsection{Transformaciones de Lorentz}
%\label{sec:transf-de-lorentz}



 

\section{Construción de Lagrangianos covariantes}
\begin{frame}[fragile,allowframebreaks]
En está sección vamos a conectar la discusión sobre el Lagrangiano de las vibraciones de la cuerda con las tranformaciones de Lorentz de la relatividad especial. Dicho Lagrangiano tiene hasta ahora la siguiente dependencia funcional $\mathcal{L}(\partial_{\mu}\phi)$.

En la formulación de la teoría clásica de campos debemos asegurarnos de que todos los términos posibles perimtidos por la simetrías asociadas al campo estén presentes en la densidad Lagrangiana. De inmediato surge la pregunta: ¿Qué posibles términos de la forma $\partial_{\mu}\phi$ podrían estar presentes en Lagrangiano?. Antes de responder está pregunta, abordemos el problema algo más general donde la densidad Lagrangiana también depende del campo como mismo
\begin{align*}
  \mathcal{L}(\phi,\partial_\mu \phi)\,.
\end{align*}
En la ecuación~\eqref{eq:dlc3d}, teníamos un Lagragiano en función de $\partial_{\mu}\phi$, y haciendo $v\to c=1$, tenemos
\begin{align}
\label{eq:Lpr}
  \mathcal{L}(\partial_{\mu}\phi)
    =&\frac{1}{2}\left[
      {\partial_0\phi}\,{\partial_0\phi}-\sum_i{\partial_i\phi}\,{\partial_i\phi}
   \right]\nonumber\\
    =&\frac{1}{2}\left[
      {\partial_0\phi}\,{\partial^0\phi}+{\partial_i\phi}\,{\partial^i\phi}
   \right]\nonumber\\
   =&\frac{1}{2}{\partial_\mu\phi}\,{\partial^\mu\phi}\,.
\end{align}
donde se ha usado la convención de suma para índices repetidos. Note que para que la velocidad de propagación sea independiente del sistema de coordenadas se requiere su identificación con la velocidad de la luz. De modo que si queremos interpretar los términos de la densidad Lagrangiana como objetos invariantes, necesariamente se tiene que hacer en el contexto de la relatividad especial.
\end{frame}

\section{Transformación de Lorentz para campos escalares}

\begin{frame}[fragile,allowframebreaks]
El campo escalar esta definido por sus propiedades bajo transformaciones de Lorentz. Vamos a estudiar el comportamiento de un campo escalar bajo una transformación general de Lorentz:
\begin{align}
\label{eq:179qft}
  x^\mu\to {x'}^\mu={\Lambda^\mu}_\nu x^\nu\,,
\end{align}
Por definición, el campo escalar no cambia bajo la transformación de Lorentz, es decir, su forma funcional queda inalterada. Por consiguiente el campo escalar debe satisfacer que
\begin{align}
 \phi(x)\to  \phi'(x')=\phi(x)\,,
\end{align}
como se ilustró en la Fig.~\ref{fig:trasla}. La prima en $\phi$ representa el cambio intrínseco en el campo $\phi$ como consecuencia de la transformación.


Usando la ec.~\eqref{eq:179qft}, tenemos que
\begin{align}
    \phi'(x')=\phi(\Lambda^{-1}x')\,.
\end{align}
Esto es, el campo transformado, evaluado en el punto transformado, da el mismo valor que el campo evaluado en el punto antes de la transformación.
\end{frame}

\begin{frame}[fragile,allowframebreaks]

%Therefore, for an arbitrary space-time point we have that the scalar field transforms under a Lorentz transformation as
Por consiguiente, para un punto del espacio tiempo arbitrario tenemos
que el campo escalar transforma bajo una transformación de Lorentz como
\begin{align}
  \label{eq:scalarlorentz}
 \phi(x)\to  \phi'(x)=\phi(\Lambda^{-1}x)\,.
\end{align}
\end{frame}



% Definimos el cambio correspondiente como
% \begin{align}
%   \delta \phi(x)=\phi'(x)-\phi(x)\,.
% \end{align}



\begin{frame}[fragile,allowframebreaks]
Para comprobar la invarianza de Lorentz de la Acción para el campo escalar, necesitamos las propiedades de transformación para $\partial_\mu$ dada por la ec.~\eqref{dmulrtran}
\begin{align}
  {\left(\Lambda^{-1}\right)^\mu}_\alpha{x'}^\alpha=&{\left(\Lambda^{-1}\right)^\mu}_\alpha{\Lambda^\alpha}_\nu x^\nu\nonumber\\
=&\delta^\mu_\nu x^\nu\nonumber\\
=&x^\mu\,,
\end{align}
\begin{align}
  \frac{1}{{x'}^\nu}= {\left(\Lambda^{-1}\right)^\mu}_\nu\frac{1}{x^\mu}\,,
\end{align}
o
\begin{align}
  \label{eq:183qft}
    \frac{1}{{x'}^\mu}= {\left(\Lambda^{-1}\right)^\nu}_\mu\frac{1}{x^\nu}\,,
\end{align}
de modo que la transformación de Lorentz para $\partial_\mu=\partial/\partial x^\mu$, es
\begin{align}
  \label{dmulrtran}
   \frac{\partial}{{\partial x'}^\mu}=& {\left(\Lambda^{-1}\right)^\nu}_\mu\frac{\partial}{\partial x^\nu}\nonumber\\
   {\partial\,}'_\mu=& {\left(\Lambda^{-1}\right)^\nu}_\mu\partial_\nu\,,
\end{align}
Podemos ahora demostrar que la Acción obtenida del Lagrangiano en la ec.\eqref{eq:Lpr} es invariante bajo transformaciones de Lorentz haciendo uso de la ec.~(\ref{eq:Lambdacontra}). Para hacer la demostración más general, podemos agregar una función general que solo dependa del campo $\phi$ pero no de sus derivadas, $V(\phi)$
\begin{align}
  \mathcal{L}(\phi(x),\partial_{\mu}\phi(x))\to  \mathcal{L}'=& \frac{1}{2}\partial'_\mu\phi'(x)\partial^{'\mu}\phi'(x)-V(\phi'), \nonumber\\
  =&{\left(\Lambda^{-1}\right)^\nu}_\mu g^{\mu \rho}{\left(\Lambda^{-1}\right)^\sigma}_\rho \partial_\nu\phi(\Lambda^{-1}x) \partial_\sigma \phi(\Lambda^{-1}x) -V(\phi')\nonumber\\
  =& g^{\nu \sigma}\partial_\nu\phi(\Lambda^{-1}x) \partial_\sigma \phi(\Lambda^{-1}x) -V(\phi(\Lambda^{-1}x))\nonumber\\
  =& \partial_\nu\phi(\Lambda^{-1}x) \partial^{\nu} \phi(\Lambda^{-1}x) -V(\phi(\Lambda^{-1}x)) \nonumber\\
  =&\mathcal{L}(\phi(\Lambda^{-1}x),\partial_{\mu}\phi(\Lambda^{-1}x))\,.
\end{align}
Ya que la Acción involucra la integración sobre todos los puntos, esta es invariante bajo transformaciones de Lorentz. Explícitamente
\begin{align}
  S\to S'=\int \operatorname{d}^4x\, \mathcal{L}
\end{align}

Note que en unidades naturales
\begin{align}
  [S]=[\hbar]=1,
\end{align}
y ya que $[d^4x]=E^{-4}$, entonces
\begin{align}
  [\mathcal{L}]=E^{4}\,.
\end{align}
Como $[\partial_{\mu}]=E$, entonces
\begin{align}
  [\phi]=E\,.
\end{align}

\end{frame}
 
\section{Principio de mínima acción para campos}
Hemos visto que la acción para una oscilación mecánica en tres dimensiones se puede escribir en una notación similar a la del producto escalar de un cuadrivector de Lorentz $\partial_\mu\phi$. Cuando dicho campo se interpreta como un campo fundamental, es decir, que su velocidad de propagación es independiente de los sistemas de referencia inerciales, entonces la Acción queda invariante bajo dicho producto escalar. También hemos visto que adicionar una función del campo $V(\phi)$, la invarianza de la acción se mantiene.

Establecemos el \emph{principio de mínima acción para campos} de la siguiente manera: La acción más general posible para un conjunto de campos contiene todos los posibles productos escalares entres los campos y sus derivadas, con las siguientes restricciones
\begin{frame}[fragile,allowframebreaks]
\begin{enumerate}
\item La dimensión de los campos y derivadas en cada término de la correspondiente densidad lagrangiana debe ser menor o igual a cuatro.
\item La densidad Lagrangiana no debe contener derivadas altas (máximo dos derivadas)
\item Los campos fundamentales se deben anular a espacio infinito.
\end{enumerate}
Con estas restricciones es suficiente mantener los primeros cuatro términos de la expansión de Taylor de $V(\phi)$ (el término constante se puede remover redefiniendo el estado de mínima energía)
\begin{align}
  \label{eq:fullV}
  V(\phi)=a \phi + b\phi^2+c\phi^3+d\phi^4\,,
\end{align}

La invarianza de la Acción bajo términos con derivadas totales excluye términos del tipo
\begin{align}
    \phi\partial_\mu \partial^\mu \phi\,.
\end{align}
De modo que la densidad Lagrangiana más general posible para un campo escalar real es
\begin{align}
  \frac{1}{2}{\partial_\mu\phi}\,{\partial^\mu\phi}-\left( a \phi + b\phi^2+c\phi^3+d\phi^4 \right)\,.
\end{align}
La ecuación de movimiento es
\begin{align}
  \left( \partial_\mu \partial^{\mu} +2b \right)\phi= J(\phi)\,,
\end{align}
donde
\begin{align}
  J(\phi)=- \left( a + 3c\phi^3 + 4d\phi^4 \right),
\end{align}
es el término de fuente. En ausencia de fuentes el campo $\phi$ se propaga libremente a través de la ecuación
\begin{align}
  \left( \partial_\mu \partial^{\mu} +2b \right)\phi= 0.
\end{align}

A continuación procedemos a encontrar una interpretación física al coeficiente $2b$
\end{frame}
\subsection{Ecuaciones covariantes}
\label{sec:ecuac-covar}


Con el cuadrivector \eqref{eq:cv_hatpmu} podemos construir la
siguiente ecuaci\'on
\begin{align}
  \hat{p}_\mu\hat{p}^\mu\phi&=m^2\phi\nonumber\\
  i\partial_\mu i\partial^\mu\phi&=m^2\phi\nonumber\\
  -\partial_\mu\partial^\mu\phi&=m^2\phi\nonumber\\
  \label{eq:waveec}
  \left(\frac{\partial^2}{\partial t^2}-\nabla^2+m^2\right)\phi&=0.
\end{align}
Que corresponde a la ecuaci\'on de Klein-Gordon \eqref{eq:152}. Una expresi\'on escrita en t\'erminos de productos escalares de Lorentz se dice que esta en \emph{forma covariante}. El Lagrangiano covariante que da lugar a
\'esta ecuaci\'on es (ver ec. \eqref{eq:150}
\eqref{eq:15}). %noinstiki[in Cap. I](/wiki/show/Cap%C3%ADtulo+I#eq:15)). 

\begin{equation}
  \label{eq:wavelagtrue}
  \mathcal{L}=\frac{1}{2}\partial_\mu\phi\partial^\mu\phi-\frac{1}{2}m^2\phi^2
\end{equation}
El Lagrangiano m\'as general posible para el campo $\phi$ es en general bastante arbitrario:
\begin{equation}
  \mathcal{L}=\frac{1}{2}\partial_\mu\phi\partial^\mu\phi-V(\phi)\,,
\end{equation}
donde $f(\phi)$ es una función de campos escalar real $\phi$. Si $V(\phi)$ es una función polinómica del campo $\phi$, tenemos por ejemplo.
\begin{equation}
  \label{eq:wavelag}
  \mathcal{L}=\frac{1}{2}\partial_\mu\phi\partial^\mu\phi-\frac{1}{2}m^2\phi^2-\frac{1}{4}\lambda\phi^4-a\phi+b\phi^3.
\end{equation}

En asencia de fuentes $a=c=d=0$ en la ec.~\eqref{eq:fullV}, y usando las
ecuaciones de Euler-Lagrange \eqref{eq:eelcallfmu}, se obtiene
\begin{align}
  (\hat{p}_\mu\hat{p}^\mu-m^2)\phi&=0\nonumber\\
  \label{eq:k-gpmu} %noinstiki
(\hat{E}^2-\hat{\mathbf{P}}^2-m^2)\phi&=0\\
\label{eq:kg} 
  (\Box+m^2)\phi&=0,
\end{align}
donde
\begin{equation}
  \label{eq:dalambertiano}
  \Box\equiv\partial_\mu\partial^\mu=\frac{\partial^2}{\partial t^2}-\nabla^2. 
\end{equation}
Es el D'Alembartiano~\cite{daelembertiano}. 
Ec.~\eqref{eq:k-gpmu} %noinstikiEc.~\eqref{eq:kg}
corresponde a la forma de operadores de la
ecuaci\'on de energ\'\i a-momentum relativista. La
ec.~\eqref{eq:kg} se conoce como la ecuaci\'on de Klein-Gordon, con
Lagrangiano
\begin{equation}
  \label{eq:kglag}
  \mathcal{L}=\frac{1}{2}\partial_\mu\phi\partial^\mu\phi-\frac{1}{2}m^2\phi^2, 
\end{equation}
Una expresi\'on escrita en t\'erminos de productos escalares de
Lorentz se dice que esta en \emph{forma covariante}. Por lo tanto la
ecuaci\'on de Klein--Gordon y su correspondiente Lagrangiano est\'an
en forma covariante. Tambi\'en tienen la simetr\'\i a
$\phi\to-\phi$. A $\phi$ se le denomina \emph{campo escalar}.

Hemos identificado que el término $2a=m^2$ se podría interpretar como la masa al cuadrado del campo escalar. Dicha interpretación surge de analizar la segunda cuantización del campo $\phi$ que se traduce en una interpretación en términos del oscilador armónico.
%TODO: FALTA

Realizaciones propuestas para este campo escalar incluye el de un campo de materia oscura con un potencial tipo oscilador armónico.
% figura
Pero una realización ya encontrada en la naturaleza corresponde al campo de Higgs con un potencial tipo ruptura expontánea de vacio

Una realización física para la acción del campo escalar real con un potencia escalar tipo ``slow-roll''  corresponde al campo del inflatón en cosmología.



\section{Campos escalares complejos}
Entre más simetrías posea una Acción menos arbitraría es. Podemos
ilustrar esta afirmación si consideramos una Acción para un campo escalar
complejo que además de ser invariante de Lorentz, es además invariante
bajo transformaciónes de fase.

El sistema que podríamos describir es el de los pares de Cooper de carga eléctrica $-2$ y dos veces la masa del electrón en el interior de un superconductor de Tipo-I.



\begin{frame}[fragile,allowframebreaks]


En ese caso la Acción, y la correspondiente densidad Lagrangiana son
únicas y están dadas por una función polinómica de $\phi^{*}\phi$
\begin{align}
  \label{eq:41qftnew}
  \mathcal{L}=\partial_{\mu}\phi^{*} \partial^{\mu}\phi-m^2\phi^{*}\phi-\lambda \left(\phi^{*}\phi \right)^2\,.
\end{align}
Términos de orden superior se pueden obtener a partir de esa
Lagrangiana única y por eso no se consideran. 
\end{frame}


Por lo tanto
\begin{align}
  \left[ m \right]=&E& \left[ \lambda \right]=&1\,.
\end{align}






De las ecuaciones de Euler-Lagrange para $\phi^*$, usando el Lagrangiano en ec.~(\ref{eq:41qftnew}) y para $\lambda=0$
\begin{align}
  \partial_\mu\left[
      \frac{\partial\mathcal{L}}{\partial(\partial_\mu\phi^*)}\right]-\frac{\partial\mathcal{L}}{\partial\phi^*}&=0\nonumber\\
    \partial_\mu\partial^\mu\phi+m^2\phi&=0\nonumber\\
    \label{eq:43qft}
    (\Box+m^2)\phi&=0,
\end{align}
y de la ecuaciones de Euler-Lagrange para $\phi$,
\begin{equation}
  \label{eq:44qft}
    (\Box+m^2)\phi^*=0.
\end{equation}
De este modo tanto $\phi$, como $\phi^*$, satisfacen la ecuación de Klein-Gordon. Cada campo además corresponde a una partícula de masa $m$ como en el caso de $\phi_1$ y $\phi_2$

Además de la invarianza de Lorentz, el Lagrangiano en ec,~(\ref{eq:41qft}) también es invariante bajo el grupo de transformaciones U(1) definido en las sección~\ref{sec:lagr-electr}, pero con una fase constante
\begin{equation*}
  U=e^{i\theta}\approx1+i\theta.
\end{equation*}
Entonces
\begin{align}
  \phi\overset{U}{\longrightarrow}\phi'&=e^{i\theta}\phi\approx(1+i\theta)\phi\nonumber\\
  &=\phi+i\theta\phi.
\end{align}
Para $\theta$ infinitesimal
\begin{align}
\label{eq:deltaphi}
  \delta\phi=&\phi'-\phi=i\theta\phi\,,&  \delta\phi^*=&{\phi'}^*-\phi^*=-i\theta\phi^*\,,
\end{align}
Comparando con la expresión para la transformación de un conjunto de campos $\phi_i$, eq.~\eqref{eq:dfi}, con la identificación $\phi_1\to \phi$ y $\phi_2\to \phi^{*}$, tenemos que
\begin{align}
  a_1=& i\phi\,, & a_2 =-i\phi^{*}\,.
\end{align}
Reemplazando en la expresión para el Teorema 1 de Noether relacionado con simetrías internas globales,eq.~\eqref{eq:thn1} tenemos que:
\begin{align}
  J^\mu=&\frac{\partial\mathcal{L}}{\partial(\partial_\mu\phi)}i\phi-i \phi^{*}\frac{\partial\mathcal{L}}{\partial(\partial_\mu\phi^*)}\nonumber\\
  \label{eq:45qft}
  J^\mu=&i\partial^\mu\phi^*\phi+\text{h.c}\,.
\end{align}
Y por lo tanto la corriente es real.
Como $\rho=J^{0}$ puede ser negativo, entonces no puede interpretarse como una
probalidad, como se hizo con la función de onda de la ecuación de
Scrödinger. Esto presentó un obstaculo en la interpretación inicial de
la ecuación de Klein-Gordon. Sin embargo una vez se cuantiza el
campo escalar la probabilidad de los estados cuánticos queda bien
definida \cite{Gross}.

Al interpretarse como carga eléctrica se predice la existencia de antipartículas!

Podemos calcular también el tensor de momentum-energía para el campo escalar
\begin{align}
  T^\mu_\nu=&\frac{\partial\mathcal{L}}{\partial\left(\partial_\mu\phi\right)} \partial_\nu \phi + \partial_\nu \phi^* \frac{\partial\mathcal{L}}{\partial\left(\partial_\mu\phi^*\right)}-\delta_\nu^\mu \mathcal{L} \nonumber\\
   =&\partial^\mu\phi^* \partial_\nu \phi + \partial_\nu \phi^* \partial^\mu\phi-\delta_\nu^\mu \left(\partial_{\alpha}\phi^{*} \partial^{\alpha}\phi-m^2\phi^{*}\phi-\lambda \left(\phi^{*}\phi \right)^2\right)\nonumber\\
\end{align}

\textbf{Ejercicio}: Calcular el Hamiltoniano para el campo escalar complejo. (Ver Sec. 1.2 de \cite{Greiner:1990tz}):
En particular, para la densidad Hamiltoniana tenemos
\begin{align}
 \mathcal{H}=  T^0_0 =&\partial^0\phi^* \partial_0 \phi + \partial_0 \phi^* \partial^0\phi-\partial_{\alpha}\phi^{*} \partial^{\alpha}\phi+m^2\phi^{*}\phi+\lambda \left(\phi^{*}\phi \right)^2 \nonumber\\
  =&\partial^0\phi^* \partial_0 \phi + \partial_0 \phi^* \partial^0\phi-\partial_{0}\phi^{*} \partial^{0}\phi-\partial_{i}\phi^{*} \partial^i\phi+m^2\phi^{*}\phi+\lambda \left(\phi^{*}\phi \right)^2 \nonumber\\
  =&\partial^0\phi^* \partial_0 \phi -\boldsymbol{\nabla}\phi^{*}\cdot\boldsymbol{\nabla}\phi+m^2\phi^{*}\phi+\lambda \left(\phi^{*}\phi \right)^2 \,.
\end{align}






\chapter{Fermiones}
\label{cha:fermiones} %noinstiki
%instiki:
%instiki:***
%instiki:
%instiki:[[NotasFS|Tabla de Contenidos]]
%instiki:
%instiki:***
%instiki:
%instiki:* [Ecuaci\'on de klein-Gordon](#ecuacion-de-klein)
%instiki:
%instiki:* [Campos escalares complejos](#camp-escal-compl)
%instiki:
%instiki:* [Invarianza gauge local abeliana](#invar-gauge-local)
%instiki:
%instiki:* [Aplicaci\'on a la mec\'anica cu\'antica](#aplic-la-mecan)
%instiki:
%instiki:* [Invarianza gauge local no abeliana](#invar-gauge-local-2)
%instiki:
%instiki:* [Invarianza gauge local para un grupo semisimple](#invar-gauge-local-1)
%instiki:
%instiki:* [$\Phi$ como un triplete de $SU(2)$](#phi-como-un)
%instiki:
%instiki:* [Problemas](#problemas-3)
%instiki:
%instiki:***
%instiki:

\section{Preliminares}

Para detalles adicionales ver por ejemplo \url{https://indico.cern.ch/event/243629}


\section{Transformaciones de Lorentz para campos fermiónicos}
\label{sec:transf-de-lorentz-1}

Para un campo escalar, un término en la densidad Lagrangiana con sólo una derivada de primer orden de la forma
\begin{align}
\label{eq:nolor}
  \phi^*(x)a^\mu\partial_\mu\phi(x)\,,
\end{align}
no deja a la Acción invariante. Un término de ese tipo con una derivada temporal de primer orden se requiere para tener una probabilidad bien definida en mecánica cuántica.
Para tener una formulación de la mecánica cuántica necesitamos garantizar la siguiente ecuación general
\begin{align}
  i\frac{\partial}{\partial t}\psi=\hat{H} \psi\,,  
\end{align}
con algún operador Hamiltoniano relativista  $\widehat{H}$ a determinar luego. Si dicha ecuación corresponde a una ecuación de Euler-Lagrange para el campo $\psi$,
entonces la Acción debe contener una densidad Lagrangiana con derivadas temporales de primer orden. Por consiguiente, la invarianza bajo transformaciones de Lorentz require que todas la derivadas en la densidad Lagrangiana sean de orden uno.

Ver \url{http://www.damtp.cam.ac.uk/user/examples/3P7.pdf}

\begin{frame}[fragile,allowframebreaks]
Considere los campos espinoriales, que se transforma como
\begin{align}
\label{eq:184qft}
  \psi_\alpha(x)\to\psi'_\alpha(x)={\left[ S(\Lambda) \right]_\alpha}^\beta\psi_\beta(\Lambda^{-1}x)\,, 
\end{align}
donde $S(\Lambda)$ es la representación espinorial $(\frac{1}{2},0)$ del grupo de Lorentz. 
\end{frame}
% En forma matricial, si $\Psi(x)$ es un vector columna de dos componentes,
% \begin{align*}
%   \Psi=
%   \begin{pmatrix}
%    \psi_1\\
%    \psi_2\\
%   \end{pmatrix}
% \end{align*}
% entonces, una representación $2\times2$ del grupo de Lorentz, denotado con $(\frac{1}{2},0)$, puede ser escrita como
% \begin{align}
%   \Psi(x)\to \Psi'(x)=S\Psi \left(\Lambda^{-1}x  \right).
% \end{align}
% In such a case,

% $S^{*}$ es otra representación independiente $2\times 2$ del Grupo de Lorentz.
% Esta se denota por 
% $\left( 0,\frac{1}{2}\right)$, y para enfatizar su diferencia se tiene la convención de denotar sus componentes con índices con puntos, $\dot{\alpha},\dot{\beta},\cdots$.
% %We can get $S^{*}$ from $S^{\dagger}$. In fact, writing out the fields without arguments to avoid clutter, 
% % \begin{align}
% %     \Psi^{\dagger}\to \Psi'^{\dagger}=\Psi^{\dagger}S^{\dagger}.
% % \end{align}
% En  componentes, y
% anticipando los índices con puntos
% para $S^{*}$, tenemos
% \begin{align}
%   \left( \psi'_\alpha \right)^{\dagger}=&\left( \psi_\beta \right)^{\dagger}{\left( S^\dagger\right)^{\dot{\beta}}}_{\dot{\alpha}}\nonumber\\
% =&\left( \psi_\beta \right)^{\dagger}{\left( S^*\right)_{\dot{\alpha}}}^{\dot{\beta}}\nonumber\\
% =&\left( \psi_\beta \right)^{\dagger}{\left( S^*\right)_{\dot{\alpha}}}^{\dot{\beta}}\,,
% \end{align}
\begin{frame}[fragile,allowframebreaks]
Si interpretamos  $\left( \psi_\beta \right)^{\dagger}$ como las componentes de un nuevo vector fila transformando bajo la representación $(0,\frac{1}{2})$,
 $S^{*}(\Lambda)$, con componentes con puntos
\begin{align}
 \psi_{\dot{\alpha}}^{\dagger}\equiv \left( \psi_\alpha \right)^{\dagger}
\end{align}
entonces tenemos que 
\begin{align}
{\psi}_{\dot{\alpha}}^{\dagger}\to  {\psi'}_{\dot{\alpha}}^{\dagger}=& \psi_{\dot{\beta}}^{\dagger}{\left[{S^{\dagger}}\right]^{\dot{\beta}}}_{\dot{\alpha}}\,.
\end{align}
\end{frame}
% or in matricial form
% \begin{align}
% {{\Psi'}^{\dagger}}^{T}=S^{*}{{\Psi}^{\dagger}}^{T}
% \end{align}
% \begin{align}
%   K=&\begin{pmatrix}
%     0 & 1\\
%     1 & 0\\
%   \end{pmatrix}& K^2=&\begin{pmatrix}
%     1 & 0\\
%     0 & 1\\
%   \end{pmatrix}
% \end{align}

\begin{frame}[fragile,allowframebreaks]
In summary we have the following Lorentz's transformation properties for the fields
\begin{align}
   \phi(x)\to \phi'(x)=&\phi(\Lambda^{-1}x) && \text{Scalar field,}\nonumber\\
   A^\mu(x)\to {A'}^\mu(x)=&{\Lambda^\mu}_\nu A^\nu(\Lambda^{-1}x)&&\text{Vector field,}\nonumber\\
  \psi_\alpha(x)\to\psi'_\alpha(x)=&{\left[ S(\Lambda) \right]_\alpha}^\beta\psi_\beta(\Lambda^{-1}x)
&& \text{Left-handed spinor field,}\nonumber\\
  {\psi}_{\dot{\alpha}}^{\dagger}(x)\to  {\psi'}_{\dot{\alpha}}^{\dagger}(x)=&
\psi_{\dot{\beta}}^{\dagger}(\Lambda^{-1}x){\left[{S^{\dagger}(\Lambda)}\right]^{\dot{\beta}}}_{\dot{\alpha}}&& \text{Rigth-handed anti-spinor field,}\,,
 \end{align}
\end{frame}




En lo siguiente usaremos la notación con puntos y sin puntos para los espinores de Lorentz, pero no su forma matricial.
\begin{frame}[fragile,allowframebreaks]
Hay otras dos representaciones adicionales del grupo de Lorentz: $\left( S^{-1}
\right)^T$ and $\left( S^{-1} \right)^{\dagger}$, pero estas son equivalentes a las representaciones $\left( \frac{1}{2},0 \right)$ y  $\left(
  0,\frac{1}{2}\right)$  respectivamente.
Los espinores que transforman bajo estas representaciones tienen los índices arriba,
  $\psi^{\alpha}$ and $\psi^{\dagger\dot{\alpha}}$, con las leyes de transformación
\begin{align}
  \psi^{\alpha}\to {\psi'}^{\alpha}=&{\left[ \left( S^{-1} \right)^T \right]^{\alpha}}_{\beta}\psi^{\beta}\nonumber\\
  \psi^{\dagger\dot{\alpha}}\to {\psi'}^{\dagger\dot{\alpha}}=&\psi^{\dagger\dot{\beta}}{\left[ \left( S^{-1} \right)^\dagger \right]^{\dot{\alpha}}}_{\dot{\beta}}\,,
\end{align}
donde
\begin{align}
  \psi^{\dagger\dot{\alpha}}\equiv \left( \psi^\alpha \right)^{\dagger}
\end{align}
Al interpretar  $\psi$ and $\psi^{\dagger}$ como espinores de dos componentes en este espacio interno, hemos definido el producto escalar usando la convención índices contraídos  \emph{descendiendo} e índices con puntos contrídos \emph{ascendiendo}
\begin{align}
\label{eq:conven}
  {{}^{\alpha}}\,{}_{\alpha}\qquad \text{and}\qquad {{}_{\dot{\alpha}}}\,{}^{\dot{\alpha}}\,.
\end{align}
De esta forma podemos definir el producto escalar entre dos espinores como
\begin{align}
\psi\psi\equiv  \psi^{\alpha}\psi_{\alpha}\to {\psi'}^{\alpha}{\psi'}_{\alpha}=& {\left[ \left( S^{-1} \right)^T \right]^{\alpha}}_{\beta}\psi^{\beta} {S_\alpha}^\gamma\psi_\gamma \nonumber\\
 =& {\left( S^{-1} \right)_{\beta}}^{\alpha}{S_\alpha}^\gamma\psi^{\beta}\psi_\gamma \nonumber\\
  =& \delta_{\beta}^{\gamma}\psi^{\beta}\psi_\gamma \nonumber\\
  =& \psi^{\beta}\psi_\beta\,.
\end{align}
and similarmente el producto escalar de dos anti-espinores como
\begin{align}
  \psi^{\dagger}\psi^{\dagger}\equiv {\psi}^{\dagger}_{\dot{\alpha}}{\psi}^{\dagger\dot{\alpha}}
\to &{\psi'}^{\dagger}_{\dot{\alpha}}{\psi'}^{\dagger\dot{\alpha}}\nonumber\\
=&\psi^{\dagger}\psi^{\dagger}\,.
\end{align}
\end{frame}
To construct Lorentz invariant Lagrangians, one needs to first combine products of spinors to make objects that transforms as Lorentz tensors.
When constructing Lorentz  tensors from fermion fields the lowered indices must only be contracted with raised indices following the same convention established in eq.~(\ref{eq:conven}). A contravariant  Lorentz tensor of rank $(n\times n)$  in this space must have an index structure with $n$ undotted (dotted) indices follow by $n$ dotted (undotted) indices, as for example,   $\alpha_1\alpha_2\ldots\alpha_n\dot{\alpha}_1\dot{\alpha}_2\ldots\dot{\alpha}_n$.


\section{Lagrangiano fermiónico}
\label{sec:dirac-equation}
La ecuación de Scrödinger puede escribirse como
\begin{align}
    i\frac{\partial}{\partial t}\psi=\hat{H}_{S} \psi\,,  
\end{align}
donde
\begin{align}
  \hat{H}_{S}=\frac{1}{2m}\hat p^2+\widehat V\,.
\end{align}


Para tener una probabilidad bien definida en Mecánica cuántica relativista es necesario que la densidad Lagrangiana sea lineal en sus derivadas temporales.
Definimos entonces una ecuación general mecánico-cuántica como
\begin{align}
  i\frac{\partial}{\partial t}\psi=\hat{H} \psi\,.
\end{align}

La relatividad especial implica automáticamente que la densidad Lagrangian también tine que ser lineal en las derivadas espaciales.
Como un campo escalar no permite escribir un término en la densidad Lagrangiana con derivadas de primer orden (ver ec.~\eqref{eq:nolor}) (ver Sec. 1.6 de~\cite{Greiner:1990tz}), debemos usar
un campo que se vea afectado por las transformaciones de Lorentz. Por consiguiente el campo debe estar formado por campos componentes
\begin{align}
  \psi=  \begin{pmatrix}
\psi_1\\
\psi_2\\
\vdots\\
\psi_n    
  \end{pmatrix}
\end{align}

%\end{document}

\begin{frame}[fragile,allowframebreaks]
We could need a new representation acting in and internal space upon a
two-component field $\psi_a$ ($a=1,2$). An Action with a Lagrangian term linear in the derivatives, could be Lorentz invariant if, taking into account the convention in eq.~(\ref{eq:conven}) and the dotted-undotted structure of the tensor in this internal space, we have that if $a^{\mu}$ is to be a 2th rank tensor of the internal space, it must have components, e.g, $\left( a^{\mu} \right)^{\dot{\alpha}\beta}$. Therefore, a posible Lorentz invariant with a single derivative could be 
 \begin{align}
   {\psi}^{\dagger}(x)a^\mu\partial_\mu\psi(x)\to  {\psi'}^{\dagger}(x)a^\mu\partial'_\mu\psi'(x)
&={\psi'}^{\dagger}_{\dot{\alpha}}(\Lambda^{-1}x)a^{\mu\dot{\alpha}\gamma}{\left(\Lambda^{-1}\right)^\rho}_\mu\partial_\rho \psi'_{\gamma}(\Lambda^{-1}x)\,,
\end{align}
with the first letters of the Greek alphabet are used to denote the indices of the internal Lorentz space, and the others the external one.
Para suavizar la notación, vamos a ignorar en adelante la dependencia en las coordenadas transformadas las cuales serán integradas en el cálculo de la Acción
\begin{align}
{\psi}^{\dagger}a^\mu\partial_\mu\psi\to  {\psi'}^{\dagger}(x)a^\mu\partial_\mu\psi'
&=
 {S^*_{\dot{\alpha}}}^{\dot{\beta}}{\psi}^{\dagger}_{\dot{\beta}}a^{\mu\dot{\alpha\gamma}}{\left(\Lambda^{-1}\right)^\rho}_\mu\partial_\rho \left( {S_{\gamma}}^{\delta}\psi_\delta \right)\,.
 \end{align} 
 Como los parámtetros de la transformación, $\xi_i$ y $\theta_i$, en la ec.~\eqref{eq:SLet} están en el espacio interno de Lorentz, 
 del conjunto de cuatro matrices constantes $\sigma^{\mu}$ no cambia bajo una transofrmación de Lorentz asociada a cambios de las coordenadas externas
\begin{align}
  {\psi}^{\dagger}a^\mu\partial_\mu\psi\to  {\psi'}^{\dagger}(x)a^\mu\partial_\mu\psi'&=
{\psi}^{\dagger}_{\dot{\beta}}{\left(\Lambda^{-1}\right)^\rho}_\mu {S^{\dagger\dot{\beta}}}_{\dot{\alpha}}a^{\mu\dot{\alpha\gamma}}{S_{\gamma}}^{\delta} \partial_\rho  \psi_\delta \nonumber\\
&=\psi^\dagger {\left(\Lambda^{-1}\right)^\rho}_\mu \left(S^\dagger a^\mu S\right)\partial_\rho\psi\nonumber\\
&=\psi^{\dagger}a^\rho\partial_\rho\psi\,,
\end{align}
De este modo, el término permanece invariante si la siguiente condición se satisface
\begin{align}
 {\left(\Lambda^{-1}\right)^\rho}_\mu  S^{\dagger}a^\mu S=a^\rho\,,
\end{align}
o
\begin{align}
\label{eq:ltrincond}
{\left(\Lambda\right)^\nu}_\rho{\left(\Lambda^{-1}\right)^\rho}_\mu   S^{\dagger}a^\mu S=&
{\left(\Lambda\right)^\nu}_\rho a^\rho\,,\nonumber\\
\delta^{\nu}_{\mu}   S^{\dagger}a^\mu S=&
{\left(\Lambda\right)^\nu}_\rho a^\rho \nonumber\\
S^{\dagger}a^\nu S=&{\left(\Lambda\right)^\nu}_\rho a^\rho\,,
\end{align}
La solución para esta identidad es única y puede expresarse en términos de las matries  $2\times2$ de Pauli más la identidad
   \begin{align} 
 a^{\mu}=\overline{\sigma}^{\mu}=& \left( \mathbf{1}_{2\times2},\overline{\boldsymbol{\sigma}} \right) \nonumber\\
 =& \left( \sigma^0,\overline{\boldsymbol{\sigma}} \right)\,, 
\end{align}
donde
\begin{align}
\sigma^0=& \mathbf{1} & \overline{\boldsymbol{\sigma}}=&-\boldsymbol{\sigma}=\left(-\sigma^1,-\sigma^2,-\sigma^3\right)\,.
\end{align}
\end{frame}
\begin{frame}[fragile,allowframebreaks]
Por consiguiente, el Lagrangiano más general posible para espinores de dos componentes contiene al menos
\begin{align}
    \mathcal{L}\supset&\frac{i}{2}{\psi}^{\dagger}_{\dot{\alpha}}\left(\overline{\sigma}^{\mu}\right)^{\dot{\alpha}\alpha}\partial_\mu\psi_{\alpha}-m\,\psi^{\alpha}\psi_{\alpha} \nonumber\\
\supset&\frac{i}{2}{\psi}^{\dagger}\overline{\sigma}^\mu\partial_\mu\psi-m\,\psi\psi\,.
\end{align}
Los coeficientes $i/2$ y $m$ se han escogido para que las ecuaciones de Euler-Lagrange den lugar a las ecuaciones de movimiento apropiadas.
Para garantizar que la Acción sea un real, debemos imponer que  $\mathcal{L}^{\dagger}=\mathcal{L}$. La forma más simple de Lograrlo es simplemente adicionar el hermítico conjugado de cada uno de los términos (h.c de las siglas en inglés). De modo que el Lagrangiano más general posible para espinores de dos componentes es
\begin{align*}
  \mathcal{L}=&\frac{i}{2}{\psi}^{\dagger}\overline{\sigma}^\mu\partial_\mu\psi-m\,\psi\psi+
\text{h.c} \nonumber\\
=&\frac{i}{2}{\psi}^{\dagger}\overline{\sigma}^\mu\partial_\mu\psi-m\,\psi\psi+
\left(\frac{i}{2} {\psi}^{\dagger}\overline{\sigma}^\mu\partial_\mu\psi \right)^{\dagger}-m \left(\psi\psi  \right)^{\dagger}\,,
\end{align*}
Entonces
\begin{align*}
  \mathcal{L}=&\frac{i}{2}{\psi}^{\dagger}\overline{\sigma}^\mu\partial_\mu\psi-\frac{i}{2} \partial_\mu\psi^{\dagger}{\overline{\sigma}^\mu}^{\dagger}\psi-m \left( \psi\psi+\psi^{\dagger}\psi^{\dagger} \right)
\end{align*}
Ya que, de la hermiticidad de las matrices de Paulo
\begin{align}
{\overline{\sigma}^\mu}^{\dagger}=\overline{\sigma}^\mu\,.
\end{align}
De modo que
\begin{align}
\mathcal{L}=&\frac{i}{2}{\psi}^{\dagger}\overline{\sigma}^\mu\partial_\mu\psi-\frac{i}{2} \partial_\mu \left(  \psi^{\dagger} \overline{\sigma}^\mu\psi\right)
+\frac{i}{2}{\psi}^{\dagger}\overline{\sigma}^\mu\partial_\mu\psi
-m \left( \psi\psi+\psi^{\dagger}\psi^{\dagger} \right)\,.
\end{align}
Descartando la derivada total que no altera la Acción, podemos obtener la forma final de la densidad Lagrangiana para un campo espinorial de dos componentes
\begin{align}
  \mathcal{L}=&i{\psi}^{\dagger}\overline{\sigma}^\mu\partial_\mu\psi-
m \left( \psi\psi+\psi^{\dagger}\psi^{\dagger} \right)\,.
\end{align}
Teniendo en cuenta de nuevo que la Acción es adimensional, que implica $[\mathcal{L}]=E^4$, y considerando además que $\sigma^{\mu}$ son matrices constantes adimensionales,entonces
\begin{align*}
  [\psi]=&E^{3/2}&\to&& [m]=&E\,.
\end{align*}



Si al campo $\psi$ se la asocia además una carga conservada asociada a una simetría continua tipo $U(1)$, tal que
\begin{align}
  \psi\to\psi'=e^{-i\alpha}\psi\,,
\end{align}
podemos imponer que el Lagrangiano sea invariante bajo cambios de fase $\psi$. En tal caso, el término con coeficiente $m$ debe se ser cero y la densidad Lagrangiana se simplifica a
\begin{align}
  \label{eq:masslesweylL}
   \mathcal{L}=&i{\psi}^{\dagger}\overline{\sigma}^\mu\partial_\mu\psi.
\end{align}

The previous Lagrangian which is invariant under
\begin{align}
  \psi\to \psi'=e^{i\alpha}\psi\,,
\end{align}
is the most general one if $\psi$ have any conserved charge, and will be the one the will use in the subsequent discussions.
\end{frame}


\subsection{Corriente conservada y Lagrangiano de Weyl}
\label{sec:corriente-conservada}
\begin{frame}[fragile,allowframebreaks]
En general
\begin{align}
   J^\mu&\propto\left[\frac{\partial\mathcal{L}}{\partial\left(\partial_\mu\psi\right)}\right]\delta\psi+\delta\psi^\dagger\left[\frac{\partial\mathcal{L}}{\partial\left(\partial_\mu\psi^\dagger\right)}\right]\nonumber\\
   &\propto i\psi^\dagger \overline{\sigma}^\mu(-i\alpha\psi)\nonumber\\
   &\propto i\psi^\dagger \overline{\sigma}^\mu(-i\alpha\psi)\nonumber\\
   &=\psi^\dagger \overline{\sigma}^\mu\psi
\end{align}
y
\begin{equation}
     J^\mu=\psi^\dagger  \overline{\sigma}^\mu\psi\,.
\end{equation}
La  densidad de corriente es
\begin{align}
  J^0&\propto \psi^\dagger \sigma^0\psi=\psi^{\dagger}\psi\,.
\end{align}
Que podemos interpretar como una densidad de probabilidad. Por consiguiente, la ecuación de movimiento se puede interpretar directamente como una ecuación de una función de onda de la mecánica cuántica.
\end{frame}
\subsection{Tensor momento-energía}
\label{sec:tens-momento-energi}
\begin{frame}[fragile,allowframebreaks]


Para el Lagrangiano
\begin{align*}
    \mathcal{L}=i \psi^{\dagger} \overline{\sigma}^{\mu} \partial_{\mu} \psi\,,
\end{align*}
calcular $T^{\mu}_{\nu}$ y a partir de  las densidades of carga conservadas demostrar que
\begin{align*}
  \langle\hat{H}_{W}\rangle=&\int_{V} \operatorname{d}^{3} x \psi^{\dagger} \hat{H}_{W} \psi \\
 \langle\hat{\boldsymbol{p}}\rangle=&\int_{V} \operatorname{d}^{3} x \psi^{\dagger}\hat{\boldsymbol{p}} \psi\,. 
\end{align*}
donde $\hat{H}_{W}=-\boldsymbol{\sigma} \cdot \hat{\boldsymbol{p}}$ y $\hat{\boldsymbol{p}}=-i \boldsymbol{\nabla}$


  Usando $\sigma^{0}=\mathbf{1}$,
\begin{align}
  T^0_0&=\frac{\partial\mathcal{L}}{\partial\left(\partial_0\psi\right)}\partial_0\psi+\partial_0\psi^\dagger\frac{\partial\mathcal{L}}{\partial\left(\partial_0\psi^\dagger\right)}-\mathcal{L}\nonumber\\
  &=i\psi^\dagger\partial_0\psi-\mathcal{L}\nonumber\\
  &=-i\psi^\dagger \overline{\sigma}^i\partial_i\psi\nonumber\\
  &=-\psi^\dagger \sigma^i \left( -i \partial_i\right)\psi\nonumber\\
  &=-\psi^\dagger(\boldsymbol{\sigma}\cdot\widehat{\mathbf{p}})\psi,\nonumber\\
  \label{eq:118qft}
  &=\psi^\dagger\hat{H}_{W} \psi,
\end{align}
donde hemos definido el Hamiltoniano de Weyl como
\begin{equation}
  \label{eq:denshal}
  \hat{H}_W= -\boldsymbol{\sigma}\cdot\widehat{\mathbf{p}}
\end{equation}
Que corresponde a la proyección del espín en la dirección de movimiento. El signo menos justica la definción de $\psi_{\alpha}$ como un espinor de Weyl izquierdo. Como la ecuación de Scröndinger es de validez general, tenemos entonces que

\begin{equation}
  i\frac{\partial}{\partial t}\psi=\hat{H}_W \psi
\end{equation}
y, como en mecánica clásica usual
\begin{equation}
  \label{eq:99qft}
  \langle\hat{H}_W\rangle=\int \psi^\dagger\hat{H}_W \psi\,d^3x.
\end{equation}
Note que esta relación no es posible para el Hamiltoniado de Weyl con término de masa.

Además
\begin{align}
    T^0_i&=\frac{\partial\mathcal{L}}{\partial\left(\partial_0\psi\right)}\partial_i\psi+\partial_i\psi^\dagger\frac{\partial\mathcal{L}}{\partial\left(\partial_0\psi^\dagger\right)}\nonumber\\
    &=i\psi^\dagger\partial_i\psi\nonumber\\
    &=-\psi^\dagger(-i\partial_i)\psi
\end{align}

de modo que
\begin{align}
\langle\hat{\mathbf{p}}\rangle=\int\psi^\dagger\hat{\mathbf{p}}\psi\,d^3 x
\end{align}
\end{frame}

\subsection{Ecuaciones de Euler-Lagrange}
\label{sec:ecuaciones-de-euler}

Queremos que el Lagrangiano de lugar a la ecuación de Scröndinger de validez general
\begin{equation}
  \label{eq:grlsch}
  i\frac{\partial}{\partial t}\psi=\hat{H} \psi
\end{equation}
con el Hamiltoniano dado en la ec.~(\ref{eq:99qft}), que corresponde a un Lagrangiano de sólo derivadas de primer orden y covariante, en lugar del Hamiltoniano para el caso no relativista. 

De hecho, aplicando las ecuaciones de Euler-Lagrange para el campo $\psi^\dagger$ al Lagrangiano en ec.~(\ref{eq:100qft}) ,tenemos
\begin{align}
  \partial_\mu\left[\frac{\partial\mathcal{L}}{\partial\left(\partial_\mu\psi^\dagger\right)}\right]-\frac{\partial\mathcal{L}}{\partial\psi^\dagger}&=0\nonumber\\
  \frac{\partial\mathcal{L}}{\partial\psi^\dagger}&=0\nonumber\\
  \label{eq:114qftm}
  i \overline{\sigma}^\mu\partial_\mu\psi&=0.
\end{align}
Expandiendo
\begin{align*}
    i \sigma^0\partial_0\psi+i \overline{\sigma}^i\partial_i\psi&=0\\
  i \sigma^0\partial_0\psi-i \sigma^i\partial_i\psi&=0\\
  i \sigma^0\partial_0\psi+\boldsymbol\sigma\cdot(-i\boldsymbol{\nabla})\psi&=0,\\
  i \sigma^0\partial_0\psi&=-(\boldsymbol\sigma\cdot\hat{\mathbf{p}})\psi,
\end{align*}
Como $\sigma^0=\mathbf{1}$,
\begin{equation}
  \label{eq:wspinorL}
    i\frac{\partial}{\partial t}\psi=-\boldsymbol\sigma\cdot\mathbf{p}\psi.
\end{equation}
De la ec.~(\ref{eq:denshal})
\begin{equation}
  \label{eq:186qft}
  \hat{H}= -\boldsymbol\sigma\cdot\mathbf{p}.
\end{equation}
Si identificamos a $\boldsymbol{\sigma}/2$, como el operador de espín, el signo menos del Hamiltoniano quiere decir que el espín del campo $\psi$ se proyecta en la dirección opuesta a su momentum. 
De esta forma podemos interpretar el espinor de Weyl de dos components $\psi$ como un espinor de Weyl izquierdo


A este punto, podemos comprobar la consistencia relacionada con que las matrices $\overline{\sigma}^\mu$ corresponden a las matrices de Pauli más la identidad $2\times2$.

\begin{frame}[fragile,allowframebreaks]
La ec.~(\ref{eq:grlsch}) puede escribirse como
\begin{equation}
  \left(i\frac{\partial}{\partial t}-\hat{H}\right)\psi=0.
\end{equation}
El campo $\psi$ también debe satisfacer la ecuación de Klein-Gordon. Podemos derivar dicha ecuación aplicando el operador
\begin{equation*}
  \left(-i\frac{\partial}{\partial t}-\hat{H}\right)
\end{equation*}
De modo que, teniendo en cuenta que $\partial\hat H/\partial t=0$,
\begin{align}
  \label{eq:2qftw}
 \left(-i\frac{\partial}{\partial t}-\hat{H}\right)\left(i\frac{\partial}{\partial t}-\hat{H}\right)\psi&=0\nonumber\\
 \left(-i\frac{\partial}{\partial t}-\hat{H}\right)\left(i\frac{\partial\psi}{\partial t}-\hat{H}\psi\right)&=0\nonumber\\
 \frac{\partial^2\psi}{\partial t^2}+i\left(\frac{\partial\hat{H}}{\partial t}\right)\psi
 +i\hat{H}\frac{\partial\psi}{\partial t}-i\hat{H}\frac{\partial\psi}{\partial t}+\hat{H}^2\psi&=0\nonumber\\
 \left(\frac{\partial^2}{\partial t^2}+\hat{H}^2\right)\psi&=0.
\end{align}
% 
De la ec.~(\ref{eq:186qft}), y usando la condición en ec.~(\ref{eq:gamma02}), tenemos
\begin{align}
\label{eq:106qft}
\hat{H}^2&=(\boldsymbol\sigma\cdot\mathbf{p})(\boldsymbol\sigma\cdot\mathbf{p})\,.
\end{align}
\end{frame}

\begin{frame}[fragile,allowframebreaks]
Sea $A$ una matriz y $\theta$ en un escalar. Entonces tenemos la identidad
\begin{align}
  \label{eq:206qftw}
  (\mathbf{A}\cdot\boldsymbol{\theta})^2=\sum_i {A^i}^2 {\theta^i}^2+\sum_{i\lt j}\left\{A^i,A^j  \right\}\theta^i \theta^j 
\end{align}
\end{frame}
\begin{itemize}
\item \textbf{Demostración}
  \begin{align}
    \left[\left(\mathbf{A}\cdot\boldsymbol{\theta}\right)\right]_{\alpha\beta}
    =&\sum_{i j}\sum_\gamma A^i_{\alpha\gamma}\theta^iA^j_{\gamma\beta}\theta^j\nonumber\\    
    =&\sum_{i j}\theta^i\theta^j\sum_\gamma A^i_{\alpha\gamma}A^j_{\gamma\beta}\nonumber\\    
    =&\sum_\gamma \sum_{i j}\theta^i\theta^jA^i_{\alpha\gamma}A^j_{\gamma\beta}\nonumber\\    
    =&\sum_\gamma \left(\sum_{i}{\theta^i}^2A^i_{\alpha\gamma}A^i_{\gamma\beta}+\sum_{i<j}\theta^i\theta^jA^i_{\alpha\gamma}A^j_{\gamma\beta}+\sum_{i>j}\theta^i\theta^jA^i_{\alpha\gamma}A^j_{\gamma\beta}\right)\nonumber\\    
    =&\sum_\gamma \left(\sum_{i}{\theta^i}^2A^i_{\alpha\gamma}A^i_{\gamma\beta}+\sum_{i<j}\theta^i\theta^jA^i_{\alpha\gamma}A^j_{\gamma\beta}+\sum_{j>i}\theta^j\theta^iA^j_{\alpha\gamma}A^i_{\gamma\beta}\right)\nonumber\\    
    =&\sum_\gamma \left[\sum_{i}{\theta^i}^2A^i_{\alpha\gamma}A^i_{\gamma\beta}+\sum_{i<j}\theta^i\theta^j\left(A^i_{\alpha\gamma}A^j_{\gamma\beta}+A^j_{\alpha\gamma}A^i_{\gamma\beta}\right)\right]\nonumber\\    
    =&\left[\sum_{i}{\theta^i}^2\left(A^iA^i\right)_{\alpha\beta}+\sum_{i<j}\theta^i\theta^j\left\{ A^i,A^j\right\}_{\alpha\beta}\right]\nonumber\\    
    =&\left[\sum_{i}{\theta^i}^2{A^i}^2+\sum_{i<j}\theta^i\theta^j\left\{ A^i,A^j\right\}\right]_{\alpha\beta}\,.
  \end{align}

\end{itemize}

\begin{frame}[fragile,allowframebreaks]
Entonces
\begin{align}
  \hat{H}^2=& \sigma_i^2p_i^2+\sum_{i\lt j}\left\{ \sigma_i, \sigma_j\right\}p_i p_j
\end{align}
(suma sobre índices repetidos). Si
\begin{align}
  \label{eq:107qft}
   \sigma_i^2&=\mathbf{1}\nonumber\\
  \left\{ \sigma_i, \sigma_j\right\}&=0\qquad i\ne j\,.
\end{align}
que se puede resumir en
\begin{align}
  \left\{ \sigma^i,\sigma^j \right\}=&2\delta_{ij} \mathbf{1}\,.
\end{align}
todo consistente con las propiedades de las matrices de Pauli en  \eqref{eq:64qftw}. 
De modo que
\begin{equation}
  \hat{H}^2=-\boldsymbol{\nabla}^2\,,
\end{equation}
y reemplazando en la ec.~\eqref{eq:2qftw} llegamos a la ecuación de Klein-Gordon para $\psi$
\begin{align}
   \left(\frac{\partial^2}{\partial t^2}-\boldsymbol{\nabla}^2\right)\psi&=0\nonumber\\
   \Box\psi&=0
\end{align}
\end{frame}
Debido a la ambigüedad  en el signo, podemos construir dos cuadrivectores independientes
   \begin{align}
 \sigma^{\mu}=& \left( \mathbf{1}_{2\times2},\boldsymbol{\sigma} \right)&
 \overline{\sigma}^{\mu}=& \left( \mathbf{1}_{2\times2},\overline{\boldsymbol{\sigma}} \right)
\end{align}
donde
\begin{align}
  \overline{\boldsymbol{\sigma}}=-\boldsymbol{\sigma}=\left(-\sigma^1,-\sigma^2,-\sigma^3\right)\,.
\end{align}
Como hemos visto, las componentes en el espacio interno son
$\sigma^{\mu}_{\alpha\dot{\alpha}}$ y $\overline{\sigma}^{\mu\;\alpha\dot{\alpha}}$, de modo que  las matrices apropiadas son $\overline{\sigma}^\mu$, y el Lagrangiano  y la ecuación para un espinor de Weyl izquierdo, son respectivamente de las ecs.~(\ref{eq:masslesweylL}) y (\ref{eq:wspinorL})
\begin{align}
  \label{eq:115qft}
  \mathcal{L}=&i\psi^\dagger\overline{\sigma}^\mu\partial_\mu\psi \nonumber\\
      =&i\psi^\dagger_{\dot{\alpha}}\overline{\sigma}^{\mu\; \alpha\dot{\alpha}}\partial_\mu\psi_{\alpha}\,,
\end{align}
que da lugar a la ecuación de movimiento
\begin{equation}
  \label{eq:116qft}
  i\overline{\sigma}^\mu\partial_\mu\psi=0,
\end{equation}
Si $\psi$ no tiene ninguna carga continua se puede adicionar un término de masa (con su correspondiente hermítico conjugado)
\begin{align}
  \mathcal{L}=& i\psi^\dagger\overline{\sigma}^\mu\partial_\mu\psi -m \left( \psi\psi+\psi^{\dagger}\psi^{\dagger} \right)\,.
\end{align}

\subsection{Lorentz invariance of the Weyl Action}

\begin{frame}[fragile,allowframebreaks]
To show that $S(\Lambda)$ is in fact a Lorentz transformation, it is convinient to write this in covariant form. If we define
\begin{align}
  \sigma^{\mu\nu}=\frac{i}{4}\left[\sigma^\mu,\overline{\sigma}^\nu\right]\,.
\end{align}
We can obtain the proper boost and rotations generators:
\begin{align*}
 \mathbf{K}= \sigma^{0i}=&-i\frac{\sigma}{2}\nonumber\\
 L_{i}=\frac{1}{2}\epsilon_{ijk}\sigma^{jk}=&-4\frac{i}{8}\epsilon_{ijk}\left[\frac{\sigma^j}{2},\frac{\sigma^k}{2}  \right]\nonumber\\
=&-\tfrac{i}{2}\epsilon_{ijk}i\epsilon^{jkl}\frac{\sigma_l}{2}\nonumber\\
=&\tfrac{1}{2}\delta_i^l\sigma_l\nonumber\\
=&\tfrac{1}{2}\sigma_i\,.
\end{align*}
% In fact, the six set of non-zero independently generators are
% \begin{align}
%   \mathcal{S}^{0i}=&\frac{i}{4}\left(\gamma^0\gamma^i-\gamma^i\gamma^0\right)=\frac{i}{2}\gamma^0\gamma^i= i B^i\nonumber\\
%   \mathcal{S}^{i j}=&\frac{i}{4}\left(\gamma^i\gamma^j-\gamma^j\gamma^i\right)=\frac{i}{2}\gamma^i\gamma^j= i R^{i j}\,.
% \end{align}
It is worth notices that in fact $\sigma^{\mu\nu}$ satisfy the Lorentz algebra, and therefore are the generators of the Lorentz group elements:
\begin{align}
  S(\Lambda)=&\exp\left(-i \omega_{\mu\nu}\frac{\sigma^{\mu\nu}}{2}\right)\nonumber\\
  \approx&1-\frac{i}{2} \omega_{\mu\nu}{\sigma^{\mu\nu}}\,.
\end{align}


  Necesitamos satisfacer la siguiente condición
\begin{align}
\label{eq:sss}
  S^\dagger\overline{\sigma}^\mu S=&{\Lambda^\mu}_\nu\overline{\sigma}^\nu
\end{align}
Ahora
\begin{align}
\label{eq:SLet}
  S(\Lambda)_{\left( \frac{1}{2},0 \right)}\equiv S(\Lambda)=
\exp\left( \boldsymbol{\xi}\cdot \frac{\boldsymbol{\sigma}}{2}+i\boldsymbol{\theta}\cdot \frac{\boldsymbol{\sigma}}{2} \right)\,,
\end{align}
y expandiendo \eqref{eq:sss}
\begin{align*}
\left(\mathbf{1}+\boldsymbol{\xi}\cdot \frac{\boldsymbol{\sigma}}{2} -i\boldsymbol{\theta}\cdot \frac{\boldsymbol{\sigma}}{2}  \right)
\overline{\sigma}^{\mu}
\left(\mathbf{1}+\boldsymbol{\xi}\cdot \frac{\boldsymbol{\sigma}}{2} +i\boldsymbol{\theta}\cdot \frac{\boldsymbol{\sigma}}{2}  \right)
=&\left[ \mathbf{1}+i\boldsymbol{\xi}\cdot \mathbf{K}+i\boldsymbol{\theta}\cdot \mathbf{L} \right]^{\mu}_{\ \nu}\overline{\sigma}^\nu \nonumber\\
\left(\overline{\sigma}^{\mu}+\boldsymbol{\xi}\cdot \frac{\boldsymbol{\sigma}}{2}\overline{\sigma}^{\mu} -i\boldsymbol{\theta}\cdot \frac{\boldsymbol{\sigma}}{2}\overline{\sigma}^{\mu}  \right)
\left(\mathbf{1}+\boldsymbol{\xi}\cdot \frac{\boldsymbol{\sigma}}{2} +i\boldsymbol{\theta}\cdot \frac{\boldsymbol{\sigma}}{2}  \right)
=&\left[ \mathbf{1}+i\boldsymbol{\xi}\cdot \mathbf{K}+i\boldsymbol{\theta}\cdot \mathbf{L} \right]^{\mu}_{\ \nu}\overline{\sigma}^\nu \,.
\end{align*}
Hasta primer orden en los parametros $\xi^i$ y $\theta^i$,
\begin{align*}
 \overline{\sigma}^{\mu}+\boldsymbol{\xi}\cdot \left(\overline{\sigma}^{\mu} \frac{\boldsymbol{\sigma}}{2} \right)  +i\boldsymbol{\theta}\cdot \left(\overline{\sigma}^{\mu} \frac{\boldsymbol{\sigma}}{2} \right)+\boldsymbol{\xi}\cdot \frac{\boldsymbol{\sigma}}{2}\overline{\sigma}^{\mu} -i\boldsymbol{\theta}\cdot \frac{\boldsymbol{\sigma}}{2}\overline{\sigma}^{\mu}
 =&\delta^{\mu}_{\nu}\overline{\sigma}^\nu+i\boldsymbol{\xi}\cdot {\mathbf{K}^{\mu}}_{\nu}\overline{\sigma}^\nu+i\boldsymbol{\theta}\cdot {\mathbf{L}^{\mu}}_{\nu}\overline{\sigma}^\nu \nonumber\\
   \overline{\sigma}^{\mu}+\boldsymbol{\xi}\cdot \left(\overline{\sigma}^{\mu}\frac{\boldsymbol{\sigma}}{2}+ \frac{\boldsymbol{\sigma}}{2}\overline{\sigma}^{\mu}\right)  +i\boldsymbol{\theta}\cdot \left(\overline{\sigma}^{\mu} \frac{\boldsymbol{\sigma}}{2}-\frac{\boldsymbol{\sigma}}{2}\overline{\sigma}^{\mu} \right)  
 =&\delta^{\mu}_{\nu}\overline{\sigma}^\nu+i\boldsymbol{\xi}\cdot {\mathbf{K}^{\mu}}_{\nu}\overline{\sigma}^\nu+i\boldsymbol{\theta}\cdot {\mathbf{L}^{\mu}}_{\nu}\overline{\sigma}^\nu \nonumber\\
  \boldsymbol{\xi}\cdot \left(\overline{\sigma}^{\mu}\frac{\boldsymbol{\sigma}}{2}+ \frac{\boldsymbol{\sigma}}{2}\overline{\sigma}^{\mu}\right)  +i\boldsymbol{\theta}\cdot \left(\overline{\sigma}^{\mu} \frac{\boldsymbol{\sigma}}{2}-\frac{\boldsymbol{\sigma}}{2}\overline{\sigma}^{\mu} \right)  
 =&i\boldsymbol{\xi}\cdot {\mathbf{K}^{\mu}}_{\nu}\overline{\sigma}^\nu+i\boldsymbol{\theta}\cdot {\mathbf{L}^{\mu}}_{\nu}\overline{\sigma}^\nu\,.
\end{align*}
Igualando coeficientes
\begin{align*}
\overline{\sigma}^{\mu}\frac{\boldsymbol{\sigma}}{2}+ \frac{\boldsymbol{\sigma}}{2}\overline{\sigma}^{\mu}  =&i{\mathbf{K}^{\mu}}_{\nu}\overline{\sigma}^\nu \nonumber\\
\overline{\sigma}^{\mu} \frac{\boldsymbol{\sigma}}{2}-\frac{\boldsymbol{\sigma}}{2}\overline{\sigma}^{\mu}=&
{\mathbf{L}^{\mu}}_{\nu}\overline{\sigma}^\nu
\end{align*}
La primera ecuación es
\begin{align*}
  \overline{\sigma}^{\mu}\frac{\sigma^i}{2} +\frac{\sigma^i}{2}\overline{\sigma}^{\mu}  =&i{\left[ K^i \right]^{\mu}}_{\nu}\overline{\sigma}^\nu \nonumber\\
 =&i{\left[ J^{0i} \right]^{\mu}}_{\nu}\overline{\sigma}^\nu \nonumber\\
  =&i{\left[ J^{0i} \right]^{\mu}}_{\nu}\overline{\sigma}^\nu \nonumber\\
  =&-\left(g^{0\mu}\delta^i_{\nu} -\delta^0_{\nu}g^{i\mu}  \right)\overline{\sigma}^\nu \nonumber\\
  =&-\left(g^{0\mu}\overline{\sigma}^i -g^{i\mu} \overline{\sigma}^0  \right)\,,
\end{align*}
para $\mu=0$
\begin{align*}
  \overline{\sigma}^{0}\frac{\sigma^i}{2} +\frac{\sigma^i}{2}\overline{\sigma}^{0}=&-\overline{\sigma}^i \nonumber\\
  \sigma^i=\sigma^i\,.
\end{align*}
Para $\mu=j$
\begin{align*}
  -\sigma^{j}\frac{\sigma^i}{2} -\frac{\sigma^i}{2}\sigma^j  =& +g^{ij}\sigma^0\nonumber\\
  -\delta^{ij}\mathbf{1}=-\delta^{ij}\mathbf{1}\,.
\end{align*}
La segunda ecuación es
\begin{align*}
\overline{\sigma}^{\mu} \frac{{\sigma^i}}{2}-\frac{{\sigma^i}}{2}\overline{\sigma}^{\mu}=&{\left(L^i  \right)^{\mu}}_{\nu}\overline{\sigma}^\nu \nonumber\\
 =&-{\left(L_i  \right)^{\mu}}_{\nu}\overline{\sigma}^\nu \nonumber\\
=&-\tfrac{1}{2}\epsilon_{ijk}{\left(J^{jk}  \right)^{\mu}}_{\nu}\overline{\sigma}^\nu \nonumber\\
 =&-\tfrac{i}{2}\epsilon_{ijk}\left(g^{j\mu}\delta^{k}_{\nu}-\delta^{j}_{\nu}g^{k\mu}  \right)\overline{\sigma}^\nu \nonumber\\
 =&-\tfrac{i}{2}\epsilon_{ijk}\left(g^{j\mu}\overline{\sigma}^k-g^{k\mu}\overline{\sigma}^j  \right) \nonumber\\
 =&\tfrac{i}{2}\epsilon_{ijk}\left(g^{j\mu}{\sigma}^k-g^{k\mu}{\sigma}^j  \right)\,.
\end{align*}
Para $\mu=0$
\begin{align*}
  \overline{\sigma}^{0} \frac{{\sigma^i}}{2}-\frac{{\sigma^i}}{2}\overline{\sigma}^{0}=& \frac{i}{2}\epsilon_{ijk}\left(g^{j0}{\sigma}^k-g^{k0}{\sigma}^j  \right)\nonumber\\
0=&0 \,.
\end{align*}
Para $\mu=l$
\begin{align*}
  \overline{\sigma}^l \frac{{\sigma^i}}{2}-\frac{{\sigma^i}}{2}\overline{\sigma}^l=&\frac{i}{2}\epsilon_{ijk}\left(g^{jl}{\sigma}^k-g^{kl}{\sigma}^j  \right)\nonumber\\
   \frac{{\sigma^i}}{2}{\sigma}^l -{\sigma}^l \frac{{\sigma^i}}{2}=&\frac{i}{2}\epsilon_{ijk}\left(-\delta^{jl}{\sigma}^k+\delta^{kl}{\sigma}^j  \right)\nonumber\\
  2\frac{\sigma^i}{2}\frac{\sigma^l}{2} -2\frac{\sigma^l}{2}\frac{\sigma^i}{2}=&\frac{i}{2}\left(-\epsilon_{ilk}{\sigma}^k+\epsilon_{ijl}{\sigma}^j  \right)\nonumber\\
 2\left[ \frac{\sigma^i}{2},\frac{\sigma^l}{2} \right]=&\frac{i}{2}\left(\epsilon_{lik}{\sigma}^k+\epsilon_{lik}{\sigma}^k  \right)\nonumber\\
 2i\epsilon_{lik}\frac{\sigma^{k}}{2}=&\frac{i}{2}\left(2\epsilon_{lik}  \sigma^{k}\right)\nonumber\\
 i\epsilon_{lik}{\sigma^{k}}=&i\epsilon_{lik}  \sigma^{k}\,.
\end{align*}

\end{frame}


\subsection{Rigth-handed fermion}

\begin{frame}[fragile,allowframebreaks]
Para el campo de dos componentes derecho  ${\eta^{\dagger}}^{\dot{\alpha}}$ ($\dot{\alpha}=\dot{1},\dot{2}$), el término invariante de Lorentz con derivada de primer orden debería ser
 \begin{align}
   {\eta}a^\mu\partial_\mu\eta^{\dagger}\to  {\eta'}\left(a^\mu\right)\partial'_\mu{\eta^{\dagger}}'
&={\eta'}^{\alpha}\left( a^\mu \right)_{\alpha\dot{\alpha}}{\left(\Lambda^{-1}\right)^\nu}_\mu\partial_\nu \,{\eta'}^{\dagger\dot{\alpha}}\nonumber\\
&={\left[ \left( S^{-1} \right)^T \right]^{\alpha}}_{\beta}{\eta}^{\beta}\left( a^\mu \right)_{\alpha\dot{\alpha}}{\left(\Lambda^{-1}\right)^\nu}_\mu {\left[ \left( S^{-1} \right)^\dagger \right]^{\dot{\alpha}}}_{\dot{\beta}}\partial_\nu \,{\eta}^{\dagger\dot{\alpha}}\nonumber\\
&={\left(\Lambda^{-1}\right)^\nu}_\mu{\left[ \left( S^{-1} \right) \right]_{\beta}}^{\alpha}\left( a^\mu \right)_{\alpha\dot{\alpha}}{\left[ \left( S^{-1} \right)^\dagger \right]^{\dot{\alpha}}}_{\dot{\beta}} {\eta}^{\beta}\partial_\nu \,{\eta}^{\dagger\dot{\alpha}}\,.
\end{align}
siempre y cuando la siguiente propiedad se satisfaga
\begin{align}
\label{eq:ltrincond}
S^{-1}a^\mu \left( S^{-1} \right)^{\dagger}=&{\left(\Lambda\right)^\mu}_\nu a^\nu\,.
\end{align}
De hecho, la única solución se puede expresar en términos de las matrices $2\times2$
   \begin{align} 
 a^{\mu}={\sigma}^{\mu} =& \left( \sigma^0,{\boldsymbol{\sigma}} \right)\,, 
\end{align}
y siguiendo un método similar al anterior, podemos arribar a la densidad Lagrangiana más general para un espinor derecho de dos componentes
\begin{align}
  \mathcal{L}=&i{\eta}{\sigma}^\mu\partial_\mu\eta^{\dagger}-
m \left( \eta\eta+\eta^{\dagger}\eta^{\dagger} \right)\,.
\end{align}
El cálculo del tensor de momento energía en este caso, nos permite  definir el Hamiltoniano de Weyl para un espinor de Weyl derecho como
\begin{equation}
  \label{eq:denshal}
  \hat{H}_W= \mathbf{\sigma}\cdot\widehat{\mathbf{p}}\,,
\end{equation}
que corresponde, en efecto, a la proyección del espín en la dirección de movimiento. El signo positivo justica la definción de $\eta^{\dot{\alpha}}$ como un espinor de Weyl derecho.
\end{frame}



\section{Espinores de Dirac}

Para ir de un cuadrivector de spín barrado al sin barrar
   \begin{align}
 \overline{\sigma}^{\mu}=\left( \mathbf{1}_{2\times2},-{\boldsymbol{\sigma}} \right) \rightarrow
\sigma^{\mu}= \left( \mathbf{1}_{2\times2},\boldsymbol{\sigma} \right)
\end{align}
podemos usar la \emph{métrica de Weyl}
\begin{align}
  \left( \sigma^{\mu} \right)_{\alpha\dot{\alpha}}=&\epsilon_{\alpha\beta}\epsilon_{\dot{\alpha}\dot{\beta}}\overline{\sigma}^{\mu\ \dot{\beta}\beta} \nonumber\\
  \left( \overline{\sigma}^{\mu} \right)^{\mu\;\dot{\alpha}\alpha}=&\epsilon^{\alpha\beta}\epsilon^{\dot{\alpha}\dot{\beta}}{\sigma}^{\mu}_{\beta\dot{\beta}} \,,
\end{align}
Estas expresiones se pueden comprobar explicitamente. Por ejemplo, en el caso de la primera matriz de Pauli
\begin{align*}
  \overline{\sigma}^{1}=
  \begin{pmatrix}
   0 & 1\\
   1 & 0 
  \end{pmatrix}=
  \begin{pmatrix}
   0                                         &\left[\overline{\sigma}^1 \right]^{\dot{1}2}\\
 \left[\overline{\sigma}^1 \right]^{\dot{2}1}&0
  \end{pmatrix}=&
  \begin{pmatrix}
   0                                         &\left[\overline{\sigma}^1 \right]^{\dot{1}2}\\
 \left[\overline{\sigma}^1 \right]^{\dot{2}1}&0
  \end{pmatrix}\nonumber\\
=&\begin{pmatrix}
   0                                         &\epsilon^{2\beta}\epsilon^{\dot{1}\dot{\beta}}\left[\sigma^1 \right]_{\beta\dot{\beta}}\\
\epsilon^{1\beta}\epsilon^{\dot{2}\dot{\beta}} \left[\sigma^1 \right]_{\beta\dot{\beta}}&0
  \end{pmatrix}\nonumber\\
=&\begin{pmatrix}
   0                                         &\epsilon^{21}\epsilon^{\dot{1}\dot{2}}\left[\sigma^1 \right]_{1\dot{2}}\\
\epsilon^{12}\epsilon^{\dot{2}\dot{1}} \left[\sigma^1 \right]_{2\dot{1}}&0
  \end{pmatrix}\nonumber\\
=&\begin{pmatrix}
   0                                         &-\left[\sigma^1 \right]_{1\dot{2}}\\
-\left[\sigma^1 \right]_{2\dot{1}}&0
  \end{pmatrix}\nonumber\\
=&\begin{pmatrix}
   0                                         &-1\\
-1&0
  \end{pmatrix}\nonumber\\
=&-\sigma^1\,.
\end{align*}

Con esto podemos reescribir el siguiente producto de espinores  de Weyl que transforma como un vector en el espacio externo de Lorentz
\begin{align}
\label{eq:exxe}
  \xi^{\dagger}\overline{\sigma}^{\mu}\eta=& \xi^{\dagger}_{\dot{\alpha}}\overline{\sigma}^{\mu\;\dot{\alpha}\alpha}\eta_{\alpha}\nonumber\\
=&\xi^{\dagger}_{\dot{\alpha}}\epsilon^{\alpha\beta}\epsilon^{\dot{\alpha}\dot{\beta}}{\sigma}^{\mu}_{\beta\dot{\beta}} \eta_{\alpha}\nonumber\\
  =&\epsilon^{\dot{\alpha}\dot{\beta}}\xi^{\dagger}_{\dot{\alpha}}{\sigma}^{\mu}_{\beta\dot{\beta}} \epsilon^{\alpha\beta}\eta_{\alpha}\nonumber\\
  =&\xi^{\dagger\;\dot{\beta}}{\sigma}^{\mu}_{\beta\dot{\beta}} \eta^{\beta}\nonumber\\
  =&\pm\eta^{\beta}{\sigma}^{\mu}_{\beta\dot{\beta}}\xi^{\dagger\;\dot{\beta}} \nonumber\\
  =&\pm\eta\,{\sigma}^{\mu}\,\xi^{\dagger} \,,
\end{align}
donde el signo $+$ ($-$) es para campos clásicos (anticonmutantes).
\begin{frame}[fragile,allowframebreaks]
Para describir completamente un electrón, que conserva carga eléctrica bajo $U(1)$, necesitamos todas las componentes detalladas en la Tabla~\ref{tab:electron}
\begin{table}
  \centering
  \begin{tabular}{llll}
    Nombre & Símbolo & Lorentz & $U(1)$\\\hline
    $e_L$: electrón izquierdo & $\xi_{\alpha}$ & ${\left[ S \right]_{\alpha}}^{\beta}$ & $e^{i\theta}$\\
   $\left( e_L \right)^{\dagger}=e^{\dagger}_R$: positrón derecho   & $\left( \xi_{\alpha} \right)^{\dagger}=\xi^{\dagger}_{\dot{\alpha}}$ & ${\left[{S^{*}}\right]_{\dot{\alpha}}}^{\dot{\beta}}$ & $e^{-i\theta}$\\
   $e_R$: electrón derecho   & $\left( \eta^{\alpha} \right)^{\dagger}=\eta^{\dagger\;\dot{\alpha}}$ & ${\left[ \left( S^{-1} \right)^\dagger \right]^{\dot{\alpha}}}_{\dot{\beta}}$& $e^{i\theta}$\\
   $\left( e_R \right)^{\dagger}=e^{\dagger}_L$: positrón izquierdo&$\eta^{\alpha}$& ${\left[ \left( S^{-1} \right)^T \right]^{\alpha}}_{\beta}$ & $e^{-i\theta}$\\\hline
  \end{tabular}
  \caption{Componentes del electrón}
  \label{tab:electron}
\end{table}

\end{frame}
\begin{frame}[fragile,allowframebreaks]
Podemos especificar el Lagrangiano completo para el electrón invariante bajo $U(1)$ sin perdidad de generalidad, usando los dos fermiones izquierdos de cargas opuestas, $\xi_{\alpha}$ y $\eta^{\alpha}$:
\begin{align}
  \xi\to\xi'=&e^{i\theta}\xi &   \eta\to\eta'=&e^{-i\theta}\eta &
\end{align}
como
\begin{align}
\label{eq:dwlag}
  \mathcal{L}=&i\xi^{\dagger}_{\dot{\alpha}}\overline{\sigma}^{\mu\;\dot{\alpha}\alpha}\partial_{\mu}\xi_{\alpha}+i\eta^{\alpha}\sigma^{\mu}_{\alpha\dot{\alpha}}\partial_{\mu}\eta^{\dagger\;\dot{\alpha}}
-m \left(\eta^{\alpha} \xi_{\alpha}+\xi^{\dagger}_{\dot{\alpha}} \eta^{\dagger\;\dot{\alpha}}\right)\nonumber\\
=&i\xi^{\dagger}\overline{\sigma}^{\mu}\partial_{\mu}\xi+i\eta\sigma^{\mu}\partial_{\mu}\eta^{\dagger}
-m \left(\eta\xi+\xi^{\dagger}\eta^{\dagger} \right)\,.
\end{align}
Definiendo el \emph{espinor de Dirac}  y su hermítico conjugado como
\begin{align}
\label{eq:psid}
  \Psi\equiv&  \begin{pmatrix}
   e_L\\
   e_R\\    
  \end{pmatrix}=
  \begin{pmatrix}
   \xi_{\alpha}\\
   \eta^{\dagger\;\dot{\alpha}}    
  \end{pmatrix}&  \Psi^{\dagger}=&
  \begin{pmatrix}
   \left( \xi_{\alpha} \right)^{\dagger} & \left( \eta^{\dagger\;\dot{\alpha}} \right)^{\dagger}    
  \end{pmatrix}=  \begin{pmatrix}
  \xi^{\dagger}_{\dot{\alpha}}& \eta^{\alpha}
  \end{pmatrix}
\end{align}
y usando \eqref{eq:exxe}, tenemos en primer lugar que
\begin{align}
\label{eq:psidm}
  \begin{pmatrix}
  \xi^{\dagger}_{\dot{\alpha}}& \eta^{\alpha}
  \end{pmatrix}
  \begin{pmatrix}
    0           &\mathbf{1}\\
    \mathbf{1} &0\\
  \end{pmatrix} \begin{pmatrix}
   \xi_{\alpha}\\
   \eta^{\dagger\;\dot{\alpha}}    
  \end{pmatrix}
=&
\begin{pmatrix}
  \eta^{\alpha} & \xi^{\dagger}_{\dot{\alpha}}
\end{pmatrix}
\begin{pmatrix}
   \xi_{\alpha}\\
   \eta^{\dagger\;\dot{\alpha}}    
  \end{pmatrix}
\nonumber\\
=&\eta^{\alpha}\xi_{\alpha} +\xi^{\dagger}_{\dot{\alpha}}\eta^{\dagger\;\dot{\alpha}}\nonumber\\
=&\eta\xi+\xi^{\dagger}\eta^{\dagger}\,,
\end{align}
mientras que para los términos cinéticos
\begin{align}
\label{eq:psidd}
  \begin{pmatrix}
  \xi^{\dagger}_{\dot{\alpha}}& \eta^{\alpha}
  \end{pmatrix}
 \begin{pmatrix}
    0           &\mathbf{1}\\
    \mathbf{1} &0\\
  \end{pmatrix}
  \begin{pmatrix}
    0           &\sigma^{\mu}_{\alpha\dot{\alpha}}\\
    \overline{\sigma}^{\mu\;\dot{\alpha}\alpha} &0\\
  \end{pmatrix} \begin{pmatrix}
   \partial_{\mu}\xi_{\alpha}\\
   \partial_{\mu}\eta^{\dagger\;\dot{\alpha}}    
  \end{pmatrix}
=&
 \begin{pmatrix}
  \xi^{\dagger}_{\dot{\alpha}}& \eta^{\alpha}
  \end{pmatrix}
  \begin{pmatrix}
    \overline{\sigma}^{\mu\;\dot{\alpha}\alpha}&0\\
     0                    &\sigma^{\mu}_{\alpha\dot{\alpha}}\\
  \end{pmatrix} \begin{pmatrix}
   \partial_{\mu}\xi_{\alpha}\\
   \partial_{\mu}\eta^{\dagger\;\dot{\alpha}}    
  \end{pmatrix}\nonumber\\
=&\xi^{\dagger}_{\dot{\alpha}}\overline{\sigma}^{\mu\;\dot{\alpha}\alpha}\partial_{\mu}\xi_{\alpha}+\eta^{\alpha}\sigma^{\mu}_{\alpha\dot{\alpha}}\partial_{\mu}\eta^{\dagger\;\dot{\alpha}}\nonumber\\
=&\xi^{\dagger}\overline{\sigma}^{\mu}\partial_{\mu}\xi+\eta\sigma^{\mu}\partial_{\mu}\eta^{\dagger}\,.
\end{align}
Definiendo las matrices de Dirac (en la representación quiral) como
\begin{align}
  \gamma^\mu= \begin{pmatrix}
    0           &\sigma^{\mu}\\
    \overline{\sigma}^{\mu} &0\\
  \end{pmatrix},
\end{align}
las cuales satisfacen el algebra
\begin{align}
  \left\{ \gamma^{\mu},\gamma^{\nu} \right\}=2 g^{\mu\nu}\mathbf{1}\,.
\end{align}

tememos en particular que, 
\begin{align}
  \gamma^0= \begin{pmatrix}
    0           &\mathbf{1}\\
    \mathbf{1} &0\\
  \end{pmatrix}
\end{align}
tal que
\begin{align}
  \left( \gamma^{0} \right)^2=\mathbf{1}_{\text{4$\times 4$}}\,.
\end{align}
Usando (\ref{eq:psid}), (\ref{eq:psidm}) y (\ref{eq:psidd})
\begin{align}
  \mathcal{L}=&i\xi^{\dagger}\overline{\sigma}^{\mu}\partial_{\mu}\xi+i\eta\sigma^{\mu}\partial_{\mu}\eta^{\dagger}
-m \left(\eta\xi+\xi^{\dagger}\eta^{\dagger} \right)\nonumber\\
=&i\Psi^{\dagger}\gamma^{0}\gamma^{\mu}\partial_{\mu}\Psi-m\Psi^{\dagger}\gamma^{0}\Psi\,.
\end{align}
Definiendo finalmente el \emph{espinor de Dirac adjunto}
\begin{align}
  \overline{\Psi}=\Psi^{\dagger}\gamma^{0}\,,
\end{align}
podemos escribir finalmente el Lagrangiano para espinores de Dirac  como
\begin{align}
\label{eq:115qftnew}
  \mathcal{L}=&i\overline{\Psi}\gamma^{\mu}\partial_{\mu}\Psi-m\overline{\Psi}\Psi\,.
\end{align}
\end{frame}
En adelante para simplificar la notación nos referiremos al electrón de Dirac simplemente como $\psi$.

En términos de los campos de Weyl un escalar y un vector de espinores Dirac se escriben como
\begin{align}
\label{eq:mwd}
  \overline{\Psi}\Psi =& \left( e_R \right)^\dagger e_L+\left( e_L \right)^{\dagger}e_R \nonumber\\
  \overline{\Psi}\gamma^{\mu}\Psi=& (e_L)^{\dagger}\overline{\sigma}^{\mu}e_L+(e_R)^{\dagger}\sigma^{\mu}e_{R}\,,
\end{align}


Un tratamiento de la invarianza de Lorentz para la Acción de Dirac esta dado en el Apéndice~\ref{cha:dirac-action}. 

\begin{frame}[fragile,allowframebreaks]
La ecuación de Dirac se obtiene facilmente de la ecuación de Euler Lagrange para el espinor adjunto $\overline{\Psi}$
\begin{align}
  \partial_{\mu}\left[ \frac{\partial\mathcal{L}}{\partial \left( \partial_{\mu} \overline{\Psi} \right)} \right]-\frac{\partial \mathcal{L}}{\partial \overline{\Psi}}=&0 \nonumber\\
-\frac{\partial \mathcal{L}}{\partial \overline{\Psi}}=&0 \nonumber\\
\left( i\gamma^{\mu}\partial_{\mu}-m \right)\Psi=0\,.
\end{align}
Escrito en la forma de la ecuación de Scrodinger general, da lugar al Hamiltoniando de Dirac
\begin{align}
  i\gamma^0 \partial_{0}\Psi=&\left[ \gamma^{i} \left( -i \partial_{i}\right)+m \right]\Psi \nonumber\\
  i \left( \gamma^0 \right)^2 \partial_{0}\Psi=&\gamma^0\left( \sum_i\gamma^{i} \widehat{p}\;^i +m \right)\Psi \nonumber\\
  i \frac{\partial}{\partial t}\Psi=&\gamma^0\left( \boldsymbol{\gamma}\cdot \widehat{\mathbf{p}} +m \right)\Psi \nonumber\\
  i \frac{\partial}{\partial t}\Psi=&\widehat{H}_D\Psi\,,
\end{align}
donde
\begin{align}
  \widehat{H}_D=\gamma^0\left( \boldsymbol{\gamma}\cdot \widehat{\mathbf{p}} +m \right)
\end{align}
\end{frame}
Antes de intentar tomar el límite no relativista, y para ser consistentes con relatividad espacial, se debe cambiar el Lagangiano por uno que tenga la invarianza de fase local. En tal caso, el Hamiltoniano resultante se debe reducir al Hamiltoniano de  Sch\"odinger en presencia de un campo electrómagnético (ver Apéndice \ref{sec:limite-no-relat}).

\textbf{Ejercicio} Calcule la corriente de carga conservada asociada la corriente $\operatorname{U}(1)$ y el tensor de energía-momentum para el Lagrangiano de Dirac.

% \begin{subappendices}
% \end{subappendices}


\section{Problemas}


\subsection{Escalares}

\begin{enumerate}
\item  Demuestre que los terminos con derivada de la densidad Lagrangiana para un campo escalar complejo
\begin{align}
  \phi\equiv&\frac{\phi_1+i\phi_2}{\sqrt{2}} &\to&&   \phi^{*}=&\frac{\phi_1-i\phi_2}{\sqrt{2}}\,,
\end{align}
que sea invariante bajo el Grupo $U(1)$ de sus cambios de fase~\eqref{eq:phchg}, se puede escribir de forma única como
\begin{align}
  \mathcal{L}(\partial_{\mu} \phi,\partial_{\mu} \phi^{*})=  {\partial_\mu\phi^{*}}\,{\partial^\mu\phi}=\frac{1}{2}\partial_{\mu}\phi_1 \partial^{\mu}\phi_1+\frac{1}{2}\partial_{\mu}\phi_2 \partial^{\mu}\phi_2\,,
\end{align}
es decir, como la suma de la densidad Lagrangiana para dos campos reales independientes.

  
\item Calcular el Hamiltoniano para el campo escalar complejo. (Ver Sec. 1.2 de \cite{Greiner:1990tz}).
  
\item Calcular el Tensor de Momento energía para el campo escalar complejo. (Ver Sec. 1.2 de \cite{Greiner:1990tz}).
\end{enumerate}


\subsection{Fermiones}


\begin{enumerate}
 \item \textquestiondown Que cambios se requieren al Lagrangiano de la ecuaci\'on de Dirac para que la Acci\'on sea invariante bajo transformaciones Gauge Locales?. Ver secci\'on \ref{sec:aplic-la-mecan}.


\label{item:pch1.3} %noinstiki

\end{enumerate}

%
%%% Local Variables: 
%%% mode: latex
%%% TeX-master: "fullnotes"
%%% ispell-local-dictionary: "castellano8"
%%% End: