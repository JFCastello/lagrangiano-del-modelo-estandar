%instiki:category: FisicaSubatomica
\chapter{Fermiones}
\label{cha:princ-gauge-local} %noinstiki
%instiki:
%instiki:***
%instiki:
%instiki:[[NotasFS|Tabla de Contenidos]]
%instiki:
%instiki:***
%instiki:
%instiki:* [Ecuaci\'on de klein-Gordon](#ecuacion-de-klein)
%instiki:
%instiki:* [Campos escalares complejos](#camp-escal-compl)
%instiki:
%instiki:* [Invarianza gauge local abeliana](#invar-gauge-local)
%instiki:
%instiki:* [Aplicaci\'on a la mec\'anica cu\'antica](#aplic-la-mecan)
%instiki:
%instiki:* [Invarianza gauge local no abeliana](#invar-gauge-local-2)
%instiki:
%instiki:* [Invarianza gauge local para un grupo semisimple](#invar-gauge-local-1)
%instiki:
%instiki:* [$\Phi$ como un triplete de $SU(2)$](#phi-como-un)
%instiki:
%instiki:* [Problemas](#problemas-3)
%instiki:
%instiki:***
%instiki:

\section{Preliminares}

\subsection{Representaciones de grupos}

\subsubsection{$SO(2)$ y $U(1)$}
Considere el grupo de rotaciones de dos ejes reales $SO(2)$. Una representación matricial corresponde al Grupo de matrices $2\times 2$ ortogonales de determinante 1
\begin{align*}
  R(\theta)=
  \begin{pmatrix}
  \cos\theta &\sin\theta\\  
  -\sin\theta&\cos\theta\\  
  \end{pmatrix},
\end{align*}
donde
\begin{align*}
  \det[R(\theta)]=\cos^2\theta+\sin^2\theta=1\,.
\end{align*}

Para generar esta matriz, podemos usar la matriz de traza nula y hermítica ($n$ entrero)
\begin{align*}
  \tau=&
  \begin{pmatrix}
   0 &-i\\ %0 &-i\\
   i &0\\  %i &0\\ 
  \end{pmatrix},&\tau^{2n}=&  \begin{pmatrix}
   1 &0\\
   0 &1\\ 
  \end{pmatrix},&\tau^{2n+1}=&  \begin{pmatrix}
   0 &-i\\
   i &0\\ 
  \end{pmatrix}.
\end{align*}
Entonces
\begin{align}
\label{eq:so2}
  R(\theta)=\exp \left( i \theta\tau \right)=&\sum_{n=0}^{\infty}\frac{\left(i \theta\tau \right)^{n}}{n!}\nonumber\\
=&\sum_{n=0}^{\infty}(i)^{2n}\frac{\left( \theta\tau \right)^{2n}}{2n!}+\sum_{n=0}^{\infty}(i)^{2n+1}\frac{\left( \theta\tau \right)^{2n+1}}{(2n+1)!}\nonumber\\
  =&\sum_{n=0}^{\infty}(-1)^{n}\frac{\theta^{2n}}{2n!}
  \begin{pmatrix}
    1 & 0\\
    0 & 1\\
  \end{pmatrix}
+\sum_{n=0}^{\infty}i(-1)^{n}\frac{ \theta^{2n+1}}{(2n+1)!}
\begin{pmatrix}
  0 & -i \\
  i & 0
\end{pmatrix}
\nonumber\\
    =&
  \begin{pmatrix}
    \cos\theta & 0\\
    0 & \cos\theta \\
  \end{pmatrix}
+
\begin{pmatrix}
  0 & \sin\theta \\
  -\sin\theta & 0
\end{pmatrix}
\nonumber\\
    =&
  \begin{pmatrix}
    \cos\theta & \sin\theta\\
     -\sin\theta& \cos\theta \\
  \end{pmatrix}
\end{align}
Este grupo es Abeliano, ya que
\begin{align}
  R(\theta_1)R(\theta_2)=R(\theta_2)R(\theta_1)
\end{align}

De otro lado el Grupo $U(1)$ corresponde a las rotaciones de un eje complejo y tiene elementos
\begin{align}
  U(\theta)=e^{i \theta Y}\,,
\end{align}
donde $Y$ es el generador de los elementos del Grupo y su representación es un número real. 

Estos dos grupos son isomorfos: para un elemento complejo $U(\theta)$ el correspondiente elemento en $SO(2)$ es la rotación por el ángulo el cual es el argumento de $U(\theta)$
\subsubsection{$SU(N)$}
\begin{english}
First, let us consider a simpler group, corresponding to the rotation group in tree dimensions. The generators are the angular momentum operators $J^i$, which satisfy the commutation relations
\end{english}
\begin{spanish}
Consideremos primero un grupo más simple, el correspondiente a las rotaciones en tres dimensiones. 
\end{spanish}
Para conocer las relaciones de conmutación de los generadores del grupo de rotaciones, podemos escribir los generadores como operadores diferenciales; de la expresión 
\begin{align}
  \mathbf{J}=&\mathbf{r}\times \mathbf{p}=\mathbf{r}\times (-i\boldsymbol{\nabla})
\end{align}
o en componentes
\begin{align}
\label{eq:rxpi}
  J^k=\left[\mathbf{x}\times (-i\boldsymbol{\nabla})\right]^k=&
-i\epsilon_{ijk}x^i\partial_j=i\epsilon_{ijk}x^i\partial^j\,
\end{align}

las relaciones de conmutación del momento angular \eqref{eq:rotgr} se obtienen de forma directa
\begin{align*}
  \left[ J^i,J^j \right]\psi=&-\left[ \epsilon_{ilm}x^{l}\partial_m ,\epsilon_{jpq}x^{p}\partial_q \right]\psi \nonumber\\
=&-\epsilon_{ilm}\epsilon_{jpq}\left[ x^{l}\partial_m ,x^{p}\partial_q \right]\psi \nonumber\\
=&-\epsilon_{ilm}\epsilon_{jpq}\left[ x^{l}\partial_m \left(x^{p}\partial_q\psi  \right)-x^{p}\partial_q \left( x^{l}\partial_m\psi \right) \right] \nonumber\\
   =&-\epsilon_{ilm}\epsilon_{jpq}\left( x^{l}\delta_{mp}\partial_q\psi +x^{l}x^{p}\partial_m \partial_q\psi  -x^{p}\delta_{ql}\partial_m\psi-x^{p}  x^{l}\partial_q\partial_m\psi \right)\,,
\end{align*}
cancelando las derivadas cruzadas
\begin{align*}
\phantom{\left[ J^i,J^j \right]\psi}   =&-\epsilon_{ilm}\epsilon_{jpq}\left( x^{l}\delta_{mp}\partial_q\psi   -x^{p}\delta_{ql}\partial_m\psi \right) \nonumber\\
   =&-\epsilon_{ilm}\epsilon_{jpq} x^{l}\delta_{mp}\partial_q\psi +i\epsilon_{ilm}\epsilon_{jpq}x^{p}\delta_{ql}\partial_m\psi \nonumber\\
   =&-\epsilon_{ilm}\epsilon_{jmq} x^{l}\partial_q\psi +i\epsilon_{ilm}\epsilon_{jpl}x^{p}\partial_m\psi \nonumber\\
   =&-\epsilon_{ilm}\epsilon_{jmq} x^{l}\partial_q\psi +i\epsilon_{iml}\epsilon_{jqm}x^{q}\partial_l\psi \qquad (l\leftrightarrow m)\ \text{in 2nd term}\nonumber\\
   =&-\epsilon_{ilm}\epsilon_{jmq} x^{l}\partial_q\psi +i\epsilon_{imq}\epsilon_{jlm}x^{l}\partial_q\psi  \qquad (l\leftrightarrow q)\ \text{in 2nd term}\,,
\end{align*}
y finalmente
\begin{align*}
  \left[ J^i,J^j \right]\psi  =& \left(-\epsilon_{ilm}\epsilon_{jmq}+\epsilon_{imq}\epsilon_{jlm}  \right) x^{l}\partial_q\psi \nonumber\\
   =& \left(\epsilon_{ilm}\epsilon_{jqm}-\epsilon_{iqm}\epsilon_{jlm}  \right) x^{l}\partial_q\psi \nonumber\\
   =& \left(\cancel{\delta_{ij}\delta_{lq}}-\delta_{iq}\delta_{lj}-\cancel{\delta_{ij}\delta_{ql}}+\delta_{il}\delta_{qj}  \right) x^{l}\partial_q\psi \nonumber\\
  =& \left(\delta_{il}\delta_{qj}-\delta_{iq}\delta_{lj}  \right) x^{l}\partial_q\psi \nonumber\\
  =& \epsilon_{kij}\epsilon_{klq} x^{l}\partial_q\psi \nonumber\\
  =&i \epsilon_{kij}\left(-i\epsilon_{klq} x^{l}\partial_q  \right)\psi \nonumber\\
  =&i \epsilon_{kij}J^{k}\psi \,.
\end{align*}

Por consiguiente, Los generadores son los operadores de momento angular $ J^i$, que satisfacen las relaciones de conmutación
\begin{align}
\label{eq:rotgr}
  \left[J^i,J^j\right]=i\epsilon_{ijk}J^k\,.
\end{align}
donde $\epsilon_{ijk}$ son las constantes de estructura del Grupo $SU(3)$

Una representación matricial de esta álgebra se puede obtener con la llamada representación adjunta del Grupo de rotaciones en 3 dimensiones, $SO(3)$, definida a partir de las constantes de estructura \cite{Veltman}
\begin{align}
  (L^i)_{jk}=-i\epsilon_{ijk}\,.
\end{align}
Estos generan los elementos de $SO(3)$
\begin{align}
  R(\boldsymbol{\theta})=&\exp(i \theta_j L^{j})\nonumber\\
                        =&\exp(i\theta_1 L_1+i\theta_2 L_2+i\theta_3 L_3)\nonumber\\
                        =&R(\theta_1)R(\theta_2)R(\theta_3)
\end{align}
donde, haciendo los mismos pasos que para $SO(2)$ en \eqref{eq:so2},
\begin{align}
  R(\boldsymbol{\theta})=
  \begin{pmatrix}
   1 &   0        &0\\
   0 &\cos\theta_1  & \sin\theta_1\\
   0 & -\sin\theta_1& \cos\theta_1\\
  \end{pmatrix}
  \begin{pmatrix}
     \cos\theta_2 &0& -\sin\theta_2\\
     0          &1& 0          \\
    \sin\theta_2  &0&  \cos\theta_2\\
  \end{pmatrix}
  \begin{pmatrix}
     \cos\theta_3 & \sin\theta_3&0\\
     -\sin\theta_3& \cos\theta_3&0\\
      0         &     0     &1\\
  \end{pmatrix}
\end{align}
Claramente, el Grupo $SO(3)$ es no Abeliano, es decir
\begin{align*}
  R(\boldsymbol{\theta}_1)R(\boldsymbol{\theta}_2)\ne R(\boldsymbol{\theta}_2)R(\boldsymbol{\theta}_1)
\end{align*}



\begin{english}
  The Pauli matrices are set of matrices satisfying this commutation relations:
\end{english}
\begin{spanish}
Las matrices de Pauli son un conjunto de matrices que satisfacen estas mismas condiciones de conmutación: 
\end{spanish}
\begin{equation}
  \label{eq:paulialg}
  \left[\frac{\tau^i}{2},\frac{\tau^j}{2} \right]=i\,\epsilon_{ijk}\frac{\tau^k}{2}
\end{equation}
donde $\tau^i$ 
\begin{equation}
  \label{eq:paulimatr}
  \tau_1=
  \begin{pmatrix}
    0&1\\
    1&0
  \end{pmatrix} \qquad
 \tau_2=
  \begin{pmatrix}
    0&-i\\
    i&0
  \end{pmatrix}\qquad 
 \tau_3=
  \begin{pmatrix}
    1&0\\
    0&-1
  \end{pmatrix}
 \end{equation}
dividas por dos, corresponden a los generadores del Grupo. Las constantes de estructura del Grupo corresponden a $\epsilon_{ijk}$. Como los generadores no conmutan, $SU(2)$ es un Grupo de Lie no Abeliano. Definiendo los generadores de $SU(2)$ como
\begin{equation}
  T^i=\frac{\tau_i}{2},
\end{equation}

Las matrices de Pauli y por consiguiente $T_i$ satisfacen 
\begin{align}
  \tau_i^\dagger&=\tau_i\nonumber\\
  \operatorname{Tr}  \left(
    \tau_i
  \right)&=0
\end{align}
Además
\begin{align}
  \label{eq:64qft}
  \det
  \left(
    \tau_i
  \right)&=-1\nonumber\\
  \left\{ 
    \tau_i,\tau_j
  \right\}&=2\delta_{ij}\cdot I\Rightarrow\tau_i^2=I\nonumber \\
\operatorname{Tr} \left(\tau^i\tau^j\right)&=2\delta^{ij}\nonumber\\
\tau_i\tau_j&=i\epsilon_{ijk}\tau_k+\delta_{ij}\,.
\end{align}

Un elemento del Grupo puede escribirse como
\begin{equation}
  \label{eq:63qft}
  U(\boldsymbol{\theta})=e^{iT^i \theta_i }\approx1+iT^i\theta_i=1+i\frac{\tau^i}{2}\theta_i\,.
\end{equation}
Como antes, $\theta_i$ son los parámetros de la transformación.  Usando las propiedades $T_i$, podemos mostrar que la representación matricial $2\times 2$, $U(\boldsymbol{\theta})$, satisface
\begin{enumerate}
\item Unitariedad: $U^{-1}(\boldsymbol{\theta})=U^{\dagger}(\boldsymbol{\theta})$. En efecto
  \begin{align*}
    U^{\dagger}(\boldsymbol{\theta})U(\boldsymbol{\theta})=&e^{-i{T^i}^{\dagger} \theta_i }e^{iT^i \theta_i }\nonumber\\
=&e^{-i T^i \theta_i }e^{iT^i \theta_i } \nonumber\\
=&e^{\mathbf{0}}\nonumber\\
=&\mathbf{1}\,,
  \end{align*}
la identidad $2\times 2$.
\item Especial (Special): Usando la formula de Jacobi para la exponencial de una matrix, $A$, $e^{A}=e^{\operatorname{Tr}A}$, tenemos que
  \begin{align*}
   \det[U(\boldsymbol{\theta})]=&\det\left\{\exp\left[  i \operatorname{Tr}\left( T^i \right)\theta_i \right]  \right\}\nonumber\\
                           =&e^{0}\nonumber\\
                           =&1\,.
  \end{align*}
\end{enumerate}
De esta manera $T_i$ genera el grupo de matrices $2\times 2$ unitarias y de determinante 1: $SU(2)$. 

El grupo $SU(2)$ de rotaciones de dos ejes complejos, es isomorfo al Grupo $SO(3)$ de rotaciones sobre tres ejes reales.

En general, si $N^{2}-1$ generadores $\Lambda_i$, satisfacen el álgebra
\begin{align}
  \left[ \Lambda_a,\Lambda_b \right]=f_{abc}\Lambda_{c}\,,
\end{align}
con
\begin{align}
  \Lambda^{\dagger}=&\Lambda\,, & \operatorname{Tr}(\Lambda)=0\,,
\end{align}
entonces las matrices $N\times N$  
\begin{align}
  U(\boldsymbol{\theta})=\exp\left( i \Lambda_{a}\theta_{a} \right)
\end{align}
son unitarias y de determinante 1, y constituyen la representación fundamental de $SU(N)$.

En el caso de $U(1)$, el único generador conmutativo satisface trivialmente el álgebra y da lugar al elemento de grupo
\begin{align}
  U(\theta)=e^{i\Lambda \theta}
\end{align}
que automáticamente tienen norma 1
\begin{align*}
|U(\theta)|^2= U^{*}(\theta)U(\theta)=1\,.
\end{align*}


\subsubsection{Grupo de Lorentz }
Para estudiar otros posibles tipos de campos además de los escalares y vectoriales, debemos explorar las representaciones del Grupo de Lorentz en $n$ dimensiones. 

Seguiremos el mismo método de encontrar representaciones matriciales a partir de representaciones matriciales de generadores del Grupo (los cuales deben satisfacer la relaciones de conmutación apropiadas) para luego exponenciar estas representaciones infinitesimales.

Para el presente problema, necesitamos conocer las relaciones de conmutación de los generadores del grupo de transformaciones de Lorentz. Hemos mostrado en la ec.~ \eqref{eq:rotgr}  que, a partir de la relación (haciendo expícito el caracter de operadores)
\begin{align}
\label{eq:rxp}
  \widehat{\mathbf{J}}=&\widehat{\mathbf{r}}\times \widehat{\mathbf{p}}=
\widehat{\mathbf{r}}\times (-i\boldsymbol{\nabla})
\end{align}
la parte correspondiente al grupo de rotaciones es
\begin{align*}
  \left[\widehat{J}^i,\widehat{J}^j\right]=i\epsilon_{ijk}\widehat{J}^k\,.
\end{align*}

La ecuación \eqref{eq:rxp} en términos de componentes esta dada en~\eqref{eq:rxpi} y corresponde a
\begin{align}
  \widehat{J}^k=i\epsilon_{ijk}x^i\partial^j
\end{align}
Definimos una representación matricial de los operadores de momento angular como
\begin{align}
  \widehat{J}^{l m}\equiv\epsilon_{lmk}\widehat{J}^k=&i\epsilon_{lmk}\epsilon_{ijk}x^i\partial^j\nonumber\\
=&i(\delta_{li}\delta_{mj}-\delta_{lj}\delta_{mi})x^i\partial^j\nonumber\\
=&i(x^l\partial^m-x^m\partial^l)\,.
\end{align}
\begin{spanish}
  Que involucran tres generadores. La generalización a cuatro dimensiones da lugar a generadores adicionales $\widehat{J}^{0i}$:
\end{spanish}
\begin{english}
  Involving three generators. The generalization to four-dimensions give to arise three further generators $\widehat{J}^{0i}$:
\end{english}
\begin{align}
  \widehat{J}^{\mu\nu}=i(x^\mu\partial^\nu-x^\nu\partial^\mu)\,.
\end{align}
\begin{spanish}
Los seis generadores satisfacen el álgebra
\end{spanish}
\begin{english}
  The six generators $\widehat{J}^{\mu\nu}$ satisfy the algebra
\end{english}
\begin{align}
\label{eq:lrtalg}
  \left[\widehat{J}^{\mu\nu},\widehat{J}^{\rho\sigma}\right]=&
i(g^{\nu\rho}\widehat{J}^{\mu\sigma}-g^{\mu\rho}\widehat{J}^{\nu\sigma}-g^{\nu\sigma}\widehat{J}^{\mu\rho}+g^{\mu\sigma}\widehat{J}^{\nu\rho})\,.
\end{align}
%From \cite{Peskin}:
Cualquier representación matricial de estos operadores que vaya a representar esta álgebra debe obedecer las mismas reglas de conmuación.

\begin{spanish}
  La exponenciación de los generadores da lugar al grupo de elementos 
\end{spanish}
\begin{english}
  The exponentiation of the generators give to arise to group elements
\end{english}
\begin{align}
  \widehat{\Lambda}=\exp\left(-i\omega_{\mu\nu}\frac{\widehat{J}^{\mu\nu}}{2}\right)
\end{align}
\begin{spanish}
  Para encontrar una representación matricial de los boosts y las rotaciones usuales, 
  consideremos un boost
\end{spanish}
\begin{english}
  To find a representation of the usual boosts and rotations, 
consider a boost
\end{english}
\begin{equation}
  \left\{x^\mu\right\}=\begin{pmatrix}
    t\\
    x\\
    y\\
    z
  \end{pmatrix}\to
  \begin{pmatrix}
    t'\\
    x'\\
    y'\\
    z'
  \end{pmatrix}=
  \begin{pmatrix}
    \frac{t+vx}{\sqrt{1-v^2}}\\
    \frac{x+vt}{\sqrt{1-v^2}}\\
    y\\
    z
  \end{pmatrix}=
  \begin{pmatrix}
    \cosh\xi&\sinh\xi&0&0\\
    \sinh\xi&\cosh\xi&0&0\\
    0     &  0  &1&0\\
    0     &  0  &0&1
  \end{pmatrix}
  \begin{pmatrix}
    t\\
    x\\
    y\\
    z
  \end{pmatrix}=\left\{{\Lambda^\mu}_{\nu}\right\}\left\{x^\nu\right\},
\end{equation}
\begin{spanish}
  Ya que
\end{spanish}
\begin{english}
  Since
\end{english}
\begin{align}
  \cosh\xi=&\sum_{n=0}^{\infty}\frac{\xi^{2n}}{2n!}\approx 1+\mathcal{O}(\xi^2)\nonumber\\
  \sinh\xi=&\sum_{n=0}^{\infty}\frac{\xi^{2n+1}}{(2n+1)!}\approx \xi+\mathcal{O}(\xi^2)\,,
\end{align}
\begin{spanish}
  Un boost infinitesimal a lo largo de $x$ es
\end{spanish}
\begin{english}
  one infinitesimal boost along $x$ is
\end{english}
\begin{align}
  \left\{{\Lambda^\mu}_{\nu}\right\}_{x-\text{boost }}\approx
  \begin{pmatrix}
    1&\xi&0&0\\
    \xi&1&0&0\\
    0&0&1&0\\
    0&0&0&1
  \end{pmatrix}.
\end{align}
\begin{spanish}
  Similarmente una rotación por un ángulo infinitesimal $\theta=\theta_3$ alrededor del plano $xy$ (o sobre el eje $z$)
\end{spanish}
\begin{english}
  Similarly a rotation by an infinitesimal angle $\theta=\theta_3$ along $xy$--plane (or about the $z$--axis)
\end{english}
\begin{align}
  \left\{{\Lambda^\mu}_{\nu}\right\}_{xy-\text{rotation }}\approx
  \begin{pmatrix}
    1&0&0&0\\
    0&1&\theta&0\\
    0&-\theta&1&0\\
    0&0&0&1
  \end{pmatrix}.
\end{align}
Que como hemos visto, puede obtenerse a partir de los generadores del Grupo de rotaciones $SO(3)$, generalizados a matrices $4\times4$
\begin{align}
  \{L^{i}\}\equiv
  \begin{pmatrix}
    1 & 0 & 0& 0\\
    0 &   &  &  \\
    0 &   & L^i_{3\times3}  &  \\
    0 &   &  &  \\
  \end{pmatrix}
\end{align}

In general we define the six independent Lorentz--Group parameters:
\begin{align}
  \omega_{i0}=-\omega_{0i}\equiv&\xi_{i} \nonumber\\
  \omega_{12}=-\omega_{21}\equiv&2\theta^3 &   \omega_{32}=-\omega_{23}\equiv&-2\theta^2 &   \omega_{13}=-\omega_{31}\equiv&2\theta^1\,.
\end{align}
Por lo tanto
\begin{align}
\xi^i=&\omega^{i0}=-\omega^{0i}&\theta^i=&\frac{1}{2}\epsilon^{ijk}\omega_{jk}\,.  
\end{align}

The $4\times 4$ matrices
\begin{align}
  \left(J^{\mu\nu}\right)_{\alpha\beta}=&i\epsilon^{\mu\nu\rho\sigma}\epsilon_{\rho\sigma\alpha\beta}\nonumber\\
=&i\left({\delta^\mu}_\alpha{\delta^\nu}_\beta-{\delta^\mu}_\beta{\delta^\nu}_\alpha\right)\nonumber\\
 {\left(J^{\mu\nu}\right)^{\alpha}}_{\beta}=&ig^{\gamma\alpha}\left({\delta^{\mu}}_{\gamma}{\delta^\nu}_\beta-{\delta^\mu}_\beta{\delta^{\nu}}_\gamma\right) \nonumber\\
 =&i\left(g^{\mu\alpha}{\delta^\nu}_\beta-{\delta^\mu}_\beta g^{\nu\alpha}\right)\nonumber\\
{\left(J^{\mu\nu}\right)^{\alpha\beta}}=&i \left( g^{\mu\alpha}g^{\nu\beta}-g^{\mu\beta}g^{\nu\alpha} \right)
\end{align}
where $\mu$ and $\nu$ label which of the six matrices we want, while $\alpha$ and $\beta$ label components of the matrices. These matrices satisfy the commutations relations \eqref{eq:lrtalg}, and generate the three boosts and three rotations of the ordinary Lorentz 4-vectors:
\begin{align}
  {\Lambda^\alpha}_\beta\approx{\delta^\alpha}_\beta-\frac{i}{2}\omega_{\mu\nu}{\left(J^{\mu\nu}\right)^\alpha}_\beta\,.
\end{align}
%ver programa mathematica
En particular
\begin{align*}
  (J^{ij})_{lm}=&i\epsilon^{ij\rho\sigma}\epsilon_{\rho\sigma lm}\\
             =&i\epsilon^{ij\rho 0}\epsilon_{\rho 0 lm}\\
             =&-i\epsilon^{ijk}\epsilon_{klm}\\
             =&\epsilon^{ijk}(L_{k})_{lm}
\end{align*}
o, en términos matriciales
\begin{align}
  L^{i}=\tfrac{1}{2}\epsilon^{ijk}J_{jk}
\end{align}
De modo que
\begin{align}
\label{eq:klkl}
  i\sum_i \theta^{i}L^{i}=-&i\theta^iL_i\nonumber\\
=&- \frac{i}{2}\epsilon^{ikl}\omega_{kl}\frac{1}{2}\epsilon_{imn}J^{mn}\nonumber\\
=&- \frac{i}{4}(\delta^k_m\delta^l_n-\delta^k_n\delta^l_m)\omega_{kl}J^{mn}\nonumber\\
=&- \frac{i}{4}(\omega_{kl}J^{kl}-\omega_{kl}J^{lk})\nonumber\\
=&- \frac{i}{4}(\omega_{kl}J^{kl}+\omega_{kl}J^{kl})\nonumber\\
=&- \frac{i}{2}\omega_{kl}J^{kl}\,.
\end{align}

usando la notación de \cite{0812.1594}, definimos también
\begin{align}
  K^i\equiv J^{0i}=-J^{i0}\,,
\end{align}
Entonces
\begin{align}
-i\omega_{\mu\nu}\frac{\widehat{J}^{\mu\nu}}{2} =&-i\omega_{0\nu}\frac{\widehat{J}^{0\nu}}{2}
-i\omega_{i\nu}\frac{\widehat{J}^{i\nu}}{2}\nonumber\\
=&-i\omega_{0i}\frac{\widehat{J}^{0i}}{2}
-i\omega_{i0}\frac{\widehat{J}^{i0}}{2}
-i\omega_{ij}\frac{\widehat{J}^{ij}}{2}\nonumber\\
=&  -i\omega_{i0}\widehat{J}^{i0}
-i\omega_{ij}\frac{\widehat{J}^{ij}}{2}\nonumber\\
=&  i\omega_{i0}\widehat{J}^{0i}
-i\omega_{ij}\frac{\widehat{J}^{ij}}{2}\,,
\end{align}
y usando \eqref{eq:klkl}
\begin{align}
=&  -i\xi_{i}K^{i}-i\omega_{ij}\frac{\widehat{J}^{ij}}{2}\nonumber\\
=&\sum_i \left(i\xi^i K^i+i\theta^i L^i  \right)\,.
\end{align}
Entonces
\begin{align}
  \{\Lambda\}=\exp\left(-i\omega_{\mu\nu}\frac{\widehat{J}^{\mu\nu}}{2}\right)=
\exp\left( i\boldsymbol{\xi}\cdot\mathbf{K}+i\boldsymbol{\theta}\cdot\mathbf{L} \right)\,.
\end{align}


%ver programa mathematica




\section{Lorentz transformation of the fields}

Note again, that a term like
\begin{align}
\label{eq:nolor}
  \phi^*(x)a^\mu\partial_\mu\phi(x)\,,
\end{align}
does not left the Action invariant. To have a proper formulation of the quantum mechanics through the general equation
\begin{align}
  i\frac{\partial}{\partial t}\psi=\hat{H} \psi\,,  
\end{align}
with some, to be determined, relativistic Hamiltonian operator $\widehat{H}$, we should be able to build a Lagrangian with temporal derivatives of order one. Therefore, the Lorentz invariant requires all the derivatives of order one.  

Consider spinor fields, which transforms as
\begin{align}
\label{eq:184qft}
  \psi_\alpha(x)\to\psi'_\alpha(x)={\left[ S(\Lambda) \right]_\alpha}^\beta\psi_\beta(\Lambda^{-1}x)\,, 
\end{align}
where $S(\Lambda)$ is some spinorial representation of the Lorentz Group. In matricial form, if $\Psi(x)$ is a two-component column vector, 
\begin{align*}
  \Psi=
  \begin{pmatrix}
   \psi_1\\
   \psi_2\\
  \end{pmatrix}
\end{align*}
then, one representation $2\times2$ of the Lorentz Group, denoted with $(\frac{1}{2},0)$, can be written as
\begin{align}
  \Psi(x)\to \Psi'(x)=S\Psi \left(\Lambda^{-1}x  \right).
\end{align}
In such a case, $S^{*}$ is another independent $2\times2$ representation of the Lorentz Group. It is denoted by $\left( 0,\frac{1}{2}\right)$, and, in order to emphasize the difference, it is convenient to denote their components with dotted indices $\dot{\alpha},\dot{\beta},\cdots$. We can get $S^{*}$ from $S^{\dagger}$. In fact, writing out the fields without arguments to avoid clutter, 
\begin{align}
    \Psi^{\dagger}\to \Psi'^{\dagger}=\Psi^{\dagger}S^{\dagger}.
\end{align}
In components, and anticipating the dotted indices for $S^{*}$, we have
\begin{align}
  \left( \psi'_\alpha \right)^{\dagger}=&\left( \psi_\beta \right)^{\dagger}{\left( S^\dagger\right)^{\dot{\beta}}}_{\dot{\alpha}}\nonumber\\
=&\left( \psi_\beta \right)^{\dagger}{\left( S^*\right)_{\dot{\alpha}}}^{\dot{\beta}}\nonumber\\
=&\left( \psi_\beta \right)^{\dagger}{\left( S^*\right)_{\dot{\alpha}}}^{\dot{\beta}}\,,
\end{align}
If we interpret $\left( \psi_\beta \right)^{\dagger}$ as the components of a new column vector transforming under $S^{*}(\Lambda)$, with dotted components
\begin{align}
 \psi_{\dot{\alpha}}^{\dagger}\equiv \left( \psi_\beta \right)^{\dagger}
\end{align}
then we have
\begin{align}
  {\psi'}_{\dot{\alpha}}^{\dagger}=&{\left[ S^*(\Lambda)\right]_{\dot{\alpha}}}^{\dot{\beta}} \psi_{\dot{\beta}}^{\dagger}\,,
\end{align}
or in matricial form
\begin{align}
{{\Psi'}^{\dagger}}^{T}=S^{*}{{\Psi}^{\dagger}}^{T}
\end{align}



In summary we have the following Lorentz's transformation properties for the fields
\begin{align}
   \phi(x)\to \phi'(x')=&\phi(x) && \text{Scalar field,}\nonumber\\
   A^\mu(x)\to {A'}^\mu=&{\Lambda^\mu}_\nu A^\nu(\Lambda^{-1}x)&&\text{Vector field,}\nonumber\\
   \Psi(x)\to\Psi'(x)=&S(\Lambda)\Psi(\Lambda^{-1}x)&&\text{Spinor field.}\nonumber\\
   \Psi^{\dagger T}(x)\to{\Psi'}^{\dagger T}(x)=&S^{*}(\Lambda)\Psi^{\dagger T}(\Lambda^{-1}x)&&\text{Conjugate spinor field.}
\end{align}

In the following we use only the dotted and undotted components for
spinors but not the matricial form. There are two additional spin-1/2
irreducible representations of the Lorentz group. $\left( S^{-1}
\right)^T$ and $\left( S^{-1} \right)^{\dagger}$, but these are
equivalent to the $\left( \frac{1}{2},0 \right)$ and the $\left(
  0,\frac{1}{2}\right)$ representations repectively. The spinors that
  transform under these representations have raised spinor indices,
  $\psi^{\alpha}$ and $\psi^{\dagger\dot{\alpha}}$, with
  transformation laws
\begin{align}
  \psi^{\alpha}\to {\psi'}^{\alpha}=&{\left[ \left( S^{-1} \right)^T \right]^{\alpha}}_{\beta}\psi^{\beta}\nonumber\\
  \psi^{\dagger\dot{\alpha}}\to {\psi'}^{\dagger\dot{\alpha}}=&{\left[ \left( S^{-1} \right)^\dagger \right]^{\dot{\alpha}}}_{\dot{\beta}}\psi^{\dagger\dot{\beta}}\nonumber\\
\end{align}
where
\begin{align}
  \psi^{\dagger\dot{\alpha}}\equiv \left( \psi^\alpha \right)^{\dagger}
\end{align}
If we interpret $\psi$ and $\psi^{\dagger}$ as two-component vectors in this internal space, we can define the scalar product by using the convention of \emph{descending} contracted indices and \emph{ascending} contracted dotted indices
\begin{align}
\label{eq:conven}
  {{}^{\alpha}}\,{}_{\alpha}\qquad \text{and}\qquad {{}_{\dot{\alpha}}}\,{}^{\dot{\alpha}}\,.
\end{align}
In this way the we can define the scalar product between two spinors as
\begin{align}
\psi\psi\equiv  \psi^{\alpha}\psi_{\alpha}\to {\psi'}^{\alpha}{\psi'}_{\alpha}=& {\left[ \left( S^{-1} \right)^T \right]^{\alpha}}_{\beta}\psi^{\beta} {S_\alpha}^\gamma\psi_\gamma \nonumber\\
 =& {\left( S^{-1} \right)_{\beta}}^{\alpha}{S_\alpha}^\gamma\psi^{\beta}\psi_\gamma \nonumber\\
  =& \delta_{\beta}^{\gamma}\psi^{\beta}\psi_\gamma \nonumber\\
  =& \psi^{\beta}\psi_\beta\,.
\end{align}
and similarly
\begin{align}
  \psi^{\dagger}\psi^{\dagger}\equiv {\psi}^{\dagger}_{\dot{\alpha}}{\psi}^{\dagger\dot{\alpha}}
\to &{\psi'}^{\dagger}_{\dot{\alpha}}{\psi'}^{\dagger\dot{\alpha}}\nonumber\\
=&\psi^{\dagger}\psi^{\dagger}\,.
\end{align}
To construct Lorentz invariant Lagrangians, one needs to first combine products of spinors to make objects that transforms as Lorentz tensors.
When constructing Lorentz  tensors from fermion fields the lowered indices must only be contracted with raised indices following the same convention established in eq.~(\ref{eq:conven}). A contravariant  Lorentz tensor of rank $(n\times n)$  in this space must have an index structure with $n$ undotted (dotted) indices follow by $n$ dotted (undotted) indices, as for example,   $\alpha_1\alpha_2\ldots\alpha_n\dot{\alpha}_1\dot{\alpha}_2\ldots\dot{\alpha}_n$.

\section{Lagrangian}
\label{sec:dirac-equation}
The Scrodinger equation can be written as
\begin{align}
    i\frac{\partial}{\partial t}\psi=\hat{H}_{S} \psi\,,  
\end{align}
where
\begin{align}
  \hat{H}_{S}=\frac{1}{2m}\hat p^2+\widehat V\,.
\end{align}


In order to have a well defined probabilty in relativistic quantum mechanics it is necessary that Lagrangian be linear in the time derivative, in order to obtain the general Sccödinger equation:
\begin{align}
  i\frac{\partial}{\partial t}\psi=\hat{H} \psi\,,  
\end{align}
like the Scrödinger Lagrangian. However, this automatically imply that the Lagrangian will be also linear in the spacial derivatives. A pure scalar field cannot involve a Lorentz invariant term of only first derivatives (see eq.~\eqref{eq:nolor}). Therefore the proposed field must have some internal structure associated with some representation of the Lorentz Group. Therefore we build the Lagrangian for a field of several components
\begin{align}
  \psi=  \begin{pmatrix}
\psi_1\\
\psi_2\\
\vdots\\
\psi_n    
  \end{pmatrix}
\end{align}

\subsection{Lorentz transformation}

If the field is to describe the electron. it must have spin and in this way it must transform under some spin representation of the Lorentz Group
\begin{align}
  \psi(x)\to \psi'(x)=S(\Lambda)\psi\left(\Lambda^{-1}x\right)\,.
\end{align}
We work in the simpler case of $2\times2$ representation of the
Lorentz Group. As the electron has both spin and electric charge, we
need at least 4 degrees of freedom to describe it, or equivalently two
complex numbers: We assume for the following discussion that $\psi$ is
one of this complex numbers.
 
Returning back to the space-time Lorentz transformation 
\begin{align}
\label{eq:4lt}
  \{\Lambda\}=\exp\left(-i\omega_{\mu\nu}\frac{\widehat{J}^{\mu\nu}}{2}\right)=
\exp\left( i\boldsymbol{\xi}\cdot\mathbf{K}+i\boldsymbol{\theta}\cdot\mathbf{L} \right)\,,
\end{align}
Here we focus on the simplest non-trivial irreducible representations of the Lorentz algebra. These are the two-dimensional (inequivalent) representations: $(\frac{1}{2},0)$ and $(0,\frac{1}{2})$. In the $(\frac{1}{2},0)$ representation, $\mathbf{L}=\boldsymbol{\sigma}/2$ and $\mathbf{K}=-i\boldsymbol{\sigma}/2$ in
Eq. \eqref{eq:4lt}, which yields
\begin{align}
\label{eq:SLet}
  S(\Lambda)_{\left( \frac{1}{2},0 \right)}\equiv S(\Lambda)=
\exp\left( \boldsymbol{\xi}\cdot \frac{\boldsymbol{\sigma}}{2}+i\boldsymbol{\theta}\cdot \frac{\boldsymbol{\sigma}}{2} \right)\,,
\end{align}
donde $\boldsymbol{\sigma}=(\sigma_1,\sigma_2,\sigma_3)$ are the Pauli matrices \eqref{eq:paulimatr}.

we could need a new representation acting in and internal space upon a
two-component field $\psi_a$ ($a=1,2$). A Action with a Lagrangian term linear in the derivatives, could be Lorentz invariant if, taking into account the convention in eq.~(\ref{eq:conven}) and the dotted-undotted structure of the tensor in this internal space, we have that if $a^{\mu}$ is to be a 2th rank tensor of the internal space, it must have components, e.g, $\left( a^{\mu} \right)^{\dot{\alpha}\beta}$. Therefore, a posible Lorentz invariant with a single derivative could be 
 \begin{align}
   {\psi}^{\dagger}a^\mu\partial_\mu\psi\to  {\psi'}^{\dagger}(x)a^\mu\partial'_\mu\psi'
&={\psi'}^{\dagger}_{\dot{\alpha}}a^{\mu\dot{\alpha}\gamma}{\left(\Lambda^{-1}\right)^\rho}_\mu\partial_\rho \psi'_{\gamma}\,,
\end{align}
with the first letters of the Greek alphabet are used to denote the indices of the internal Lorentz space, and the others the external one. 
\begin{align}
{\psi}^{\dagger}a^\mu\partial_\mu\psi\to  {\psi'}^{\dagger}(x)a^\mu\partial_\mu\psi'
&=
 {S^*_{\dot{\alpha}}}^{\dot{\beta}}{\psi}^{\dagger}_{\dot{\beta}}a^{\mu\dot{\alpha\gamma}}{\left(\Lambda^{-1}\right)^\rho}_\mu\partial_\rho \left( {S_{\gamma}}^{\delta}\psi_\delta \right)\,.
 \end{align} 
As the coordinates $\eta_i$ and $\theta_i$ in eq.~\eqref{eq:SLet} are in the internal Lorentz space, the corresponding Lorentz transformation is constant in the external Lorentz space and
\begin{align}
  {\psi}^{\dagger}a^\mu\partial_\mu\psi\to  {\psi'}^{\dagger}(x)a^\mu\partial_\mu\psi'&=
{\psi}^{\dagger}_{\dot{\beta}}{\left(\Lambda^{-1}\right)^\rho}_\mu {S^{\dagger\dot{\beta}}}_{\dot{\alpha}}a^{\mu\dot{\alpha\gamma}}{S_{\gamma}}^{\delta} \partial_\rho  \psi_\delta \nonumber\\
&=\psi^\dagger {\left(\Lambda^{-1}\right)^\rho}_\mu \left(S^\dagger a^\mu S\right)\partial_\rho\psi\nonumber\\
&=\psi^{\dagger}a^\rho\partial_\rho\psi\,,
\end{align}
if the following condition is satisfied:
\begin{align}
 {\left(\Lambda^{-1}\right)^\rho}_\mu  S^{\dagger}a^\mu S=a^\rho\,,
\end{align}
or
\begin{align}
\label{eq:ltrincond}
{\left(\Lambda\right)^\nu}_\rho{\left(\Lambda^{-1}\right)^\rho}_\mu   S^{\dagger}a^\mu S=&
{\left(\Lambda\right)^\nu}_\rho a^\rho\,,\nonumber\\
\delta^{\nu}_{\mu}   S^{\dagger}a^\mu S=&
{\left(\Lambda\right)^\nu}_\rho a^\rho \nonumber\\
S^{\dagger}a^\nu S=&{\left(\Lambda\right)^\nu}_\rho a^\rho\,,
\end{align}
At this point we could show that $2\times2$ matrices $a^{\mu}$ are in fact the Pauli matrices plus the identity by using the internal Lorentz transformation. However we will postpone this demonstration and, as a self consistency check, we will obtain the structure of $a^{\mu}$ from the consequences of the Noether's theorem. 

The other possibility to have a Lorentz  vector  is the combination ${\psi}b^\mu\partial_\mu\psi^{\dagger}$, but it can be shown to be equivalent to the original $\psi^{\dagger}a^\mu\partial_\mu\psi$ with some specific relation between $a^{\mu}$ and $b^{\mu}$. 

Other possible combinations like $\psi^{\dagger}\psi$ or $\psi a^\mu\partial_\mu\psi$ or $\psi^{\dagger} a^\mu\partial_\mu\psi^{\dagger}$ do not have the proper internal Lorentz index structure. 

Therefore  the most general Lagrangian for two-component spinors is
\begin{align*}
  \mathcal{L}=\frac{i}{2}{\psi}^{\dagger}a^\mu\partial_\mu\psi+m\psi\psi+
\left(\frac{i}{2} {\psi}^{\dagger}a^\mu\partial_\mu\psi \right)^{\dagger}+m \left(\psi\psi  \right)^{\dagger}\,,
\end{align*}
where the last two terms guarantee that $\mathcal{L}^{\dagger}=\mathcal{L}$ so that the Action be real, and the coefficients have been choosing before hand to give the proper equations of motion. Then,
\begin{align*}
  \mathcal{L}=&\frac{i}{2}{\psi}^{\dagger}a^\mu\partial_\mu\psi-\frac{i}{2} \partial_\mu\psi^{\dagger}{a^\mu}^{\dagger}\psi+m \left( \psi\psi+\psi^{\dagger}\psi^{\dagger} \right)
\end{align*}
If
\begin{align}
{a^{\mu}}^{\dagger}=a^{\mu}  
\end{align}
as expected to the going to be Pauli matrices, then
\begin{align}
\mathcal{L}=&\frac{i}{2}{\psi}^{\dagger}a^\mu\partial_\mu\psi-\frac{i}{2} \partial_\mu \left(  \psi^{\dagger} a^\mu\psi\right)
+\frac{i}{2}{\psi}^{\dagger}a^\mu\partial_\mu\psi
+m \left( \psi\psi+\psi^{\dagger}\psi^{\dagger} \right)\,,
\end{align}
and dropping out the total derivative, we have finally the most general Action for two-component spinors:
\begin{align}
  \mathcal{L}=&i{\psi}^{\dagger}a^\mu\partial_\mu\psi+
m \left( \psi\psi+\psi^{\dagger}\psi^{\dagger} \right)\,.
\end{align}
If in addition we impose that the Lagrangian be invariant under changes of phase of $\psi$ then the mass of the field must be zero and
\begin{align}
   \mathcal{L}=&i{\psi}^{\dagger}a^\mu\partial_\mu\psi.
\end{align}
The previos Lagrangian which is invariant under
\begin{align}
  \psi\to \psi'=e^{i\alpha}\psi\,,
\end{align}
is the most general one if $\psi$ have any conserved charge, and will be the one the will use in the subsequent discussions.



\subsection{Corriente conservada y Lagrangiano de Dirac}
\label{sec:corriente-conservada}
De la ec.~\eqref{eq:197qft}
\begin{align}
  J^0&=\left[\frac{\partial\mathcal{L}}{\partial\left(\partial_0\psi\right)}\right]\delta\psi+\delta{\psi}^{\dagger}\left[\frac{\partial\mathcal{L}}{\partial\left(\partial_0{\psi}^\dagger\right)}\right]
\end{align}
El Lagrangiano es invariante bajo transformaciones de fase globales, $U(1)$
\begin{equation}
  \psi\to\psi'=e^{-i\alpha}\psi\approx\psi-i\alpha\psi,
\end{equation}
de modo que
\begin{equation}
  \delta\psi=-i\alpha\psi.
\end{equation}
Por consiguiente
\begin{align}
  J^0=&i\psi^{\dagger}a^0 (-i\alpha\psi)+0 \nonumber\\
     =&\alpha{\psi}^{\dagger}a^0\psi 
\end{align}
La  densidad de corriente es
\begin{align}
  J^0&\propto \psi^\dagger a^0\psi\,.
\end{align}
Que podemos interpretar como una densidad de probabilidad $\psi^{\dagger}\psi$ si
\begin{align}
  a^0=\mathbf{1}_{2\times2}\,.
\end{align}

En general
\begin{align}
   J^\mu&\propto\left[\frac{\partial\mathcal{L}}{\partial\left(\partial_\mu\psi\right)}\right]\delta\psi+\delta\psi^\dagger\left[\frac{\partial\mathcal{L}}{\partial\left(\partial_\mu\psi^\dagger\right)}\right]\nonumber\\
   &\propto i\psi^\dagger a^\mu(-i\alpha\psi)\nonumber\\
   &\propto i\psi^\dagger a^\mu(-i\alpha\psi)\nonumber\\
   &=\psi^\dagger a^\mu\psi
\end{align}
y
\begin{equation}
     J^\mu=\psi^\dagger  a^\mu\psi\,.
\end{equation}
donde $a^{\mu}$ es el cuadrivector de matrices
\begin{align}
  a^{\mu}=\left(\mathbf{1},a^{1},a^{2},a^{3}\right)\,,
\end{align}
y $a^i$ están aún por determinar
\subsection{Tensor momento-energía}
\label{sec:tens-momento-energi}
Usando $a^{0}=\mathbf{1}$,
\begin{align}
  T^0_0&=\frac{\partial\mathcal{L}}{\partial\left(\partial_0\psi\right)}\partial_0\psi+\partial_0\psi^\dagger\frac{\partial\mathcal{L}}{\partial\left(\partial_0\psi^\dagger\right)}-\mathcal{L}\nonumber\\
  &=i\psi^\dagger\partial_0\psi-\mathcal{L}\nonumber\\
  &=-i\psi^\dagger a^i\partial_i\psi\nonumber\\
  &=\psi^\dagger a^i \left( -i \partial_i\right)\psi\nonumber\\
  &=\psi^\dagger(\mathbf{a}\cdot\mathbf{p})\psi,\nonumber\\
  \label{eq:118qft}
  &=\psi^\dagger\hat{H} \psi,
\end{align}
donde
\begin{equation}
  \label{eq:denshal}
  \hat{H}= \mathbf{a}\cdot\mathbf{p}
\end{equation}
la ecuación de Scröndinger de validez general es entonces:
\begin{equation}
  i\frac{\partial}{\partial t}\psi=\hat{H} \psi
\end{equation}
y, como en mecánica clásica usual
\begin{equation}
  \label{eq:99qft}
  \langle\hat{H}\rangle=\int \psi^\dagger\hat{H} \psi\,d^3x.
\end{equation}


Además
\begin{align}
    T^0_i&=\frac{\partial\mathcal{L}}{\partial\left(\partial_0\psi\right)}\partial_i\psi+\partial_i\psi^\dagger\frac{\partial\mathcal{L}}{\partial\left(\partial_0\psi^\dagger\right)}\nonumber\\
    &=i\psi^\dagger\partial_i\psi\nonumber\\
    &=-\psi^\dagger(-i\partial_i)\psi
\end{align}
de modo que
\begin{equation}
  \langle\hat{\mathbf{p}}\rangle=\int\psi^\dagger\hat{\mathbf{p}}\psi\,d^3 x
\end{equation}
\subsection{Ecuaciones de Euler-Lagrange}
\label{sec:ecuaciones-de-euler}
Queremos que el Lagrangiano de lugar a la ecuación de Scröndinger de validez general
\begin{equation}
  \label{eq:grlsch}
  i\frac{\partial}{\partial t}\psi=\hat{H} \psi
\end{equation}
con el Hamiltoniano dado en la ec.~(\ref{eq:99qft}), que corresponde a un Lagrangiano de sólo derivadas de primer orden y covariante, en lugar del Hamiltoniano para el caso no relativista. 

De hecho, aplicando las ecuaciones de Euler-Lagrange para el campo $\psi^\dagger$ al Lagrangiano en ec.~(\ref{eq:100qft}) ,tenemos
\begin{align}
  \partial_\mu\left[\frac{\partial\mathcal{L}}{\partial\left(\partial_\mu\psi^\dagger\right)}\right]-\frac{\partial\mathcal{L}}{\partial\psi^\dagger}&=0\nonumber\\
  \frac{\partial\mathcal{L}}{\partial\psi^\dagger}&=0\nonumber\\
  \label{eq:114qftm}
  i a^\mu\partial_\mu\psi&=0.
\end{align}
Expandiendo
\begin{align*}
  i a^0\partial_0\psi+i a^i\partial_i\psi&=0\\
  i a^0\partial_0\psi-\boldsymbol{a}\cdot(-i\boldsymbol{\nabla})\psi&=0,\\
  i a^0\partial_0\psi&=(\boldsymbol{a}\cdot\hat{\mathbf{p}})\psi,
\end{align*}
Como $a^{0}=\mathbf{1}$,
\begin{equation}
    i\frac{\partial}{\partial t}\psi=\boldsymbol{a}\cdot\mathbf{p}\psi.
\end{equation}
De la ec.~(\ref{eq:denshal})
\begin{equation}
  \label{eq:186qft}
  \hat{H}= \boldsymbol{a}\cdot\mathbf{p},
\end{equation}
A este punto, sólo nos queda por determinar los parámetros $a^\mu$. 

La ec.~(\ref{eq:grlsch}) puede escribirse como
\begin{equation}
  \left(i\frac{\partial}{\partial t}-\hat{H}\right)\psi=0.
\end{equation}
El campo $\psi$ también debe satisfacer la ecuación de Klein-Gordon. Podemos derivar dicha ecuación aplicando el operador
\begin{equation*}
  \left(-i\frac{\partial}{\partial t}-\hat{H}\right)
\end{equation*}
De modo que, teniendo en cuenta que $\partial\hat H/\partial t=0$,
\begin{align}
  \label{eq:105qft}
 \left(-i\frac{\partial}{\partial t}-\hat{H}\right)\left(i\frac{\partial}{\partial t}-\hat{H}\right)\psi&=0\nonumber\\
 \left(-i\frac{\partial}{\partial t}-\hat{H}\right)\left(i\frac{\partial\psi}{\partial t}-\hat{H}\psi\right)&=0\nonumber\\
 \frac{\partial^2\psi}{\partial t^2}+i\left(\frac{\partial\hat{H}}{\partial t}\right)\psi
 +i\hat{H}\frac{\partial\psi}{\partial t}-i\hat{H}\frac{\partial\psi}{\partial t}+\hat{H}^2\psi&=0\nonumber\\
 \left(\frac{\partial^2}{\partial t^2}+\hat{H}^2\right)\psi&=0.
\end{align}
% 
De la ec.~(\ref{eq:186qft}), y usando la condición en ec.~(\ref{eq:gamma02}), tenemos
\begin{align}
\label{eq:106qft}
\hat{H}^2&=(\boldsymbol{a}\cdot\mathbf{p})(\boldsymbol{a}\cdot\mathbf{p})\,.
\end{align}

Sea $A$ una matriz y $\theta$ en un escalar. Entonces tenemos la identidad
\begin{align}
  \label{eq:206qft}
  (\mathbf{A}\cdot\boldsymbol{\theta})^2=\sum_i {A^i}^2 {\theta^i}^2+\sum_{i\lt j}\left\{A^i,A^j  \right\}\theta^i \theta^j 
\end{align}
\begin{itemize}
\item \textbf{Demostración}
  \begin{align}
    \left[\left(\mathbf{A}\cdot\boldsymbol{\theta}\right)\right]_{\alpha\beta}
    =&\sum_{i j}\sum_\gamma A^i_{\alpha\gamma}\theta^iA^j_{\gamma\beta}\theta^j\nonumber\\    
    =&\sum_{i j}\theta^i\theta^j\sum_\gamma A^i_{\alpha\gamma}A^j_{\gamma\beta}\nonumber\\    
    =&\sum_\gamma \sum_{i j}\theta^i\theta^jA^i_{\alpha\gamma}A^j_{\gamma\beta}\nonumber\\    
    =&\sum_\gamma \left(\sum_{i}{\theta^i}^2A^i_{\alpha\gamma}A^i_{\gamma\beta}+\sum_{i<j}\theta^i\theta^jA^i_{\alpha\gamma}A^j_{\gamma\beta}+\sum_{i>j}\theta^i\theta^jA^i_{\alpha\gamma}A^j_{\gamma\beta}\right)\nonumber\\    
    =&\sum_\gamma \left(\sum_{i}{\theta^i}^2A^i_{\alpha\gamma}A^i_{\gamma\beta}+\sum_{i<j}\theta^i\theta^jA^i_{\alpha\gamma}A^j_{\gamma\beta}+\sum_{j>i}\theta^j\theta^iA^j_{\alpha\gamma}A^i_{\gamma\beta}\right)\nonumber\\    
    =&\sum_\gamma \left[\sum_{i}{\theta^i}^2A^i_{\alpha\gamma}A^i_{\gamma\beta}+\sum_{i<j}\theta^i\theta^j\left(A^i_{\alpha\gamma}A^j_{\gamma\beta}+A^j_{\alpha\gamma}A^i_{\gamma\beta}\right)\right]\nonumber\\    
    =&\left[\sum_{i}{\theta^i}^2\left(A^iA^i\right)_{\alpha\beta}+\sum_{i<j}\theta^i\theta^j\left\{ A^i,A^j\right\}_{\alpha\beta}\right]\nonumber\\    
    =&\left[\sum_{i}{\theta^i}^2{A^i}^2+\sum_{i<j}\theta^i\theta^j\left\{ A^i,A^j\right\}\right]_{\alpha\beta}\,.
  \end{align}

\end{itemize}
Entonces
\begin{align}
  \hat{H}^2=& a_i^2p_i^2+\sum_{i\lt j}\left\{ a_i, a_j\right\}p_i p_j
\end{align}
(suma sobre índices repetidos). Si
\begin{align}
  \label{eq:107qft}
   a_i^2&=\mathbf{1}\nonumber\\
  \left\{ a_i, a_j\right\}&=0\qquad i\ne j\,.
\end{align}
que se puede resumir en
\begin{align}
  \left\{ a^i,a^j \right\}=&2\delta_{ij} \mathbf{1}\,.
\end{align}
una de las propiedas de las matrices de Pauli en  \eqref{eq:64qft}. De esta forma, podemos identificar
\begin{align}
  a^{i}=\pm \sigma^{i}\,.
\end{align}
de modo que
\begin{equation}
  \hat{H}^2=-\boldsymbol{\nabla}^2\,,
\end{equation}
y reemplazando en la ec.~\eqref{eq:105qft} llegamos a la ecuación de Klein-Gordon para $\psi$
\begin{align}
   \left(\frac{\partial^2}{\partial t^2}-\boldsymbol{\nabla}^2\right)\psi&=0\nonumber\\
   \Box\psi&=0
\end{align}
Debido a la ambigüedad  en el signo, podemos construir dos cuadrivectores independientes
   \begin{align}
 \sigma^{\mu}=& \left( \mathbf{1}_{2\times2},\boldsymbol{\sigma} \right)&
 \overline{\sigma}^{\mu}=& \left( \mathbf{1}_{2\times2},\overline{\boldsymbol{\sigma}} \right)
\end{align}
donde
\begin{align}
  \overline{\boldsymbol{\sigma}}=-\boldsymbol{\sigma}=\left(-\sigma^1,-\sigma^2,-\sigma^3\right)\,.
\end{align}
Como veremos luego, las componentes en el espacio interno son
$\sigma^{\mu}_{\alpha\dot{\alpha}}$ y $\overline{\sigma}^{\mu\;\alpha\dot{\alpha}}$, de modo que  las matrices apropiadas son $\overline{\sigma}^\mu$, y el Lagrangiano  y la ecuación de Weyl, son respectivamente de las ecs.~(\ref{eq:100qft}) y (\ref{eq:114qft})
\begin{align}
  \label{eq:115qft}
  \mathcal{L}=&i\psi^\dagger\overline{\sigma}^\mu\partial_\mu\psi \nonumber\\
      =&i\psi^\dagger_{\dot{\alpha}}\overline{\sigma}^{\mu\; \alpha\dot{\alpha}}\partial_\mu\psi_{\alpha}\,,
\end{align}
\begin{equation}
  \label{eq:116qft}
  i\overline{\sigma}^\mu\partial_\mu\psi=0,
\end{equation}

\subsection{Lorentz invariance of the Dirac Action}
To show that $S(\Lambda)$ is in fact a Lorentz transformation, it is convinient to write this in covariant form. If we define
\begin{align}
  \sigma^{\mu\nu}=\frac{i}{4}\left[\sigma^\mu,\overline{\sigma}^\nu\right]\,.
\end{align}
We can obtain the proper boost and rotations generators:
\begin{align*}
 \mathbf{K}= \sigma^{0i}=&-i\frac{\sigma}{2}\nonumber\\
 L_{i}=\frac{1}{2}\epsilon_{ijk}\sigma^{jk}=&-4\frac{i}{8}\epsilon_{ijk}\left[\frac{\sigma^j}{2},\frac{\sigma^k}{2}  \right]\nonumber\\
=&-\tfrac{i}{2}\epsilon_{ijk}i\epsilon^{jkl}\frac{\sigma_l}{2}\nonumber\\
=&\tfrac{1}{2}\delta_i^l\sigma_l\nonumber\\
=&\tfrac{1}{2}\sigma_i\,.
\end{align*}
% In fact, the six set of non-zero independently generators are
% \begin{align}
%   \mathcal{S}^{0i}=&\frac{i}{4}\left(\gamma^0\gamma^i-\gamma^i\gamma^0\right)=\frac{i}{2}\gamma^0\gamma^i= i B^i\nonumber\\
%   \mathcal{S}^{i j}=&\frac{i}{4}\left(\gamma^i\gamma^j-\gamma^j\gamma^i\right)=\frac{i}{2}\gamma^i\gamma^j= i R^{i j}\,.
% \end{align}
It is worth notices that in fact $\sigma^{\mu\nu}$ satisfy the Lorentz algebra, and therefore are the generators of the Lorentz group elements:
\begin{align}
  S(\Lambda)=&\exp\left(-i \omega_{\mu\nu}\frac{\sigma^{\mu\nu}}{2}\right)\nonumber\\
  \approx&1-\frac{i}{2} \omega_{\mu\nu}{\sigma^{\mu\nu}}\,.
\end{align}


\begin{english}
  We need to satisfy the following condition
\end{english}
\begin{spanish}
  Necesitamos satifacer la siguiente condición
\end{spanish}
\begin{align}
\label{eq:sss}
  S^\dagger\overline{\sigma}^\mu S=&{\Lambda^\mu}_\nu\overline{\sigma}^\nu
\end{align}
Ahora
\begin{align}
\label{eq:SLet}
  S(\Lambda)_{\left( \frac{1}{2},0 \right)}\equiv S(\Lambda)=
\exp\left( \boldsymbol{\xi}\cdot \frac{\boldsymbol{\sigma}}{2}+i\boldsymbol{\theta}\cdot \frac{\boldsymbol{\sigma}}{2} \right)\,,
\end{align}
y expandiendo \eqref{eq:sss}
\begin{align*}
\left(\mathbf{1}+\boldsymbol{\xi}\cdot \frac{\boldsymbol{\sigma}}{2} -i\boldsymbol{\theta}\cdot \frac{\boldsymbol{\sigma}}{2}  \right)
\overline{\sigma}^{\mu}
\left(\mathbf{1}+\boldsymbol{\xi}\cdot \frac{\boldsymbol{\sigma}}{2} +i\boldsymbol{\theta}\cdot \frac{\boldsymbol{\sigma}}{2}  \right)
=&\left[ \mathbf{1}+i\boldsymbol{\xi}\cdot \mathbf{K}+i\boldsymbol{\theta}\cdot \mathbf{L} \right]^{\mu}_{\ \nu}\overline{\sigma}^\nu \nonumber\\
\left(\overline{\sigma}^{\mu}+\boldsymbol{\xi}\cdot \frac{\boldsymbol{\sigma}}{2}\overline{\sigma}^{\mu} -i\boldsymbol{\theta}\cdot \frac{\boldsymbol{\sigma}}{2}\overline{\sigma}^{\mu}  \right)
\left(\mathbf{1}+\boldsymbol{\xi}\cdot \frac{\boldsymbol{\sigma}}{2} +i\boldsymbol{\theta}\cdot \frac{\boldsymbol{\sigma}}{2}  \right)
=&\left[ \mathbf{1}+i\boldsymbol{\xi}\cdot \mathbf{K}+i\boldsymbol{\theta}\cdot \mathbf{L} \right]^{\mu}_{\ \nu}\overline{\sigma}^\nu \,.
\end{align*}
Hasta primer orden en los parametros $\xi^i$ y $\theta^i$,
\begin{align*}
 \overline{\sigma}^{\mu}+\boldsymbol{\xi}\cdot \left(\overline{\sigma}^{\mu} \frac{\boldsymbol{\sigma}}{2} \right)  +i\boldsymbol{\theta}\cdot \left(\overline{\sigma}^{\mu} \frac{\boldsymbol{\sigma}}{2} \right)+\boldsymbol{\xi}\cdot \frac{\boldsymbol{\sigma}}{2}\overline{\sigma}^{\mu} -i\boldsymbol{\theta}\cdot \frac{\boldsymbol{\sigma}}{2}\overline{\sigma}^{\mu}
 =&\delta^{\mu}_{\nu}\overline{\sigma}^\nu+i\boldsymbol{\xi}\cdot {\mathbf{K}^{\mu}}_{\nu}\overline{\sigma}^\nu+i\boldsymbol{\theta}\cdot {\mathbf{L}^{\mu}}_{\nu}\overline{\sigma}^\nu \nonumber\\
   \overline{\sigma}^{\mu}+\boldsymbol{\xi}\cdot \left(\overline{\sigma}^{\mu}\frac{\boldsymbol{\sigma}}{2}+ \frac{\boldsymbol{\sigma}}{2}\overline{\sigma}^{\mu}\right)  +i\boldsymbol{\theta}\cdot \left(\overline{\sigma}^{\mu} \frac{\boldsymbol{\sigma}}{2}-\frac{\boldsymbol{\sigma}}{2}\overline{\sigma}^{\mu} \right)  
 =&\delta^{\mu}_{\nu}\overline{\sigma}^\nu+i\boldsymbol{\xi}\cdot {\mathbf{K}^{\mu}}_{\nu}\overline{\sigma}^\nu+i\boldsymbol{\theta}\cdot {\mathbf{L}^{\mu}}_{\nu}\overline{\sigma}^\nu \nonumber\\
  \boldsymbol{\xi}\cdot \left(\overline{\sigma}^{\mu}\frac{\boldsymbol{\sigma}}{2}+ \frac{\boldsymbol{\sigma}}{2}\overline{\sigma}^{\mu}\right)  +i\boldsymbol{\theta}\cdot \left(\overline{\sigma}^{\mu} \frac{\boldsymbol{\sigma}}{2}-\frac{\boldsymbol{\sigma}}{2}\overline{\sigma}^{\mu} \right)  
 =&i\boldsymbol{\xi}\cdot {\mathbf{K}^{\mu}}_{\nu}\overline{\sigma}^\nu+i\boldsymbol{\theta}\cdot {\mathbf{L}^{\mu}}_{\nu}\overline{\sigma}^\nu\,.
\end{align*}
Igualando coeficientes
\begin{align*}
\overline{\sigma}^{\mu}\frac{\boldsymbol{\sigma}}{2}+ \frac{\boldsymbol{\sigma}}{2}\overline{\sigma}^{\mu}  =&i{\mathbf{K}^{\mu}}_{\nu}\overline{\sigma}^\nu \nonumber\\
\overline{\sigma}^{\mu} \frac{\boldsymbol{\sigma}}{2}-\frac{\boldsymbol{\sigma}}{2}\overline{\sigma}^{\mu}=&
{\mathbf{L}^{\mu}}_{\nu}\overline{\sigma}^\nu
\end{align*}
La primera ecuación es
\begin{align*}
  \overline{\sigma}^{\mu}\frac{\sigma^i}{2} +\frac{\sigma^i}{2}\overline{\sigma}^{\mu}  =&i{\left[ K^i \right]^{\mu}}_{\nu}\overline{\sigma}^\nu \nonumber\\
 =&i{\left[ J^{0i} \right]^{\mu}}_{\nu}\overline{\sigma}^\nu \nonumber\\
  =&i{\left[ J^{0i} \right]^{\mu}}_{\nu}\overline{\sigma}^\nu \nonumber\\
  =&-\left(g^{0\mu}\delta^i_{\nu} -\delta^0_{\nu}g^{i\mu}  \right)\overline{\sigma}^\nu \nonumber\\
  =&-\left(g^{0\mu}\overline{\sigma}^i -g^{i\mu} \overline{\sigma}^0  \right)\,,
\end{align*}
para $\mu=0$
\begin{align*}
  \overline{\sigma}^{0}\frac{\sigma^i}{2} +\frac{\sigma^i}{2}\overline{\sigma}^{0}=&-\overline{\sigma}^i \nonumber\\
  \sigma^i=\sigma^i\,.
\end{align*}
Para $\mu=j$
\begin{align*}
  -\sigma^{j}\frac{\sigma^i}{2} -\frac{\sigma^i}{2}\sigma^j  =& +g^{ij}\sigma^0\nonumber\\
  -\delta^{ij}\mathbf{1}=-\delta^{ij}\mathbf{1}\,.
\end{align*}
La segunda ecuación es
\begin{align*}
\overline{\sigma}^{\mu} \frac{{\sigma^i}}{2}-\frac{{\sigma^i}}{2}\overline{\sigma}^{\mu}=&{\left(L^i  \right)^{\mu}}_{\nu}\overline{\sigma}^\nu \nonumber\\
 =&-{\left(L_i  \right)^{\mu}}_{\nu}\overline{\sigma}^\nu \nonumber\\
=&-\tfrac{1}{2}\epsilon_{ijk}{\left(J^{jk}  \right)^{\mu}}_{\nu}\overline{\sigma}^\nu \nonumber\\
 =&-\tfrac{i}{2}\epsilon_{ijk}\left(g^{j\mu}\delta^{k}_{\nu}-\delta^{j}_{\nu}g^{k\mu}  \right)\overline{\sigma}^\nu \nonumber\\
 =&-\tfrac{i}{2}\epsilon_{ijk}\left(g^{j\mu}\overline{\sigma}^k-g^{k\mu}\overline{\sigma}^j  \right) \nonumber\\
 =&\tfrac{i}{2}\epsilon_{ijk}\left(g^{j\mu}{\sigma}^k-g^{k\mu}{\sigma}^j  \right)\,.
\end{align*}
Para $\mu=0$
\begin{align*}
  \overline{\sigma}^{0} \frac{{\sigma^i}}{2}-\frac{{\sigma^i}}{2}\overline{\sigma}^{0}=& \frac{i}{2}\epsilon_{ijk}\left(g^{j0}{\sigma}^k-g^{k0}{\sigma}^j  \right)\nonumber\\
0=&0 \,.
\end{align*}
Para $\mu=l$
\begin{align*}
  \overline{\sigma}^l \frac{{\sigma^i}}{2}-\frac{{\sigma^i}}{2}\overline{\sigma}^l=&\frac{i}{2}\epsilon_{ijk}\left(g^{jl}{\sigma}^k-g^{kl}{\sigma}^j  \right)\nonumber\\
   \frac{{\sigma^i}}{2}{\sigma}^l -{\sigma}^l \frac{{\sigma^i}}{2}=&\frac{i}{2}\epsilon_{ijk}\left(-\delta^{jl}{\sigma}^k+\delta^{kl}{\sigma}^j  \right)\nonumber\\
  2\frac{\sigma^i}{2}\frac{\sigma^l}{2} -2\frac{\sigma^l}{2}\frac{\sigma^i}{2}=&\frac{i}{2}\left(-\epsilon_{ilk}{\sigma}^k+\epsilon_{ijl}{\sigma}^j  \right)\nonumber\\
 2\left[ \frac{\sigma^i}{2},\frac{\sigma^l}{2} \right]=&\frac{i}{2}\left(\epsilon_{lik}{\sigma}^k+\epsilon_{lik}{\sigma}^k  \right)\nonumber\\
 2i\epsilon_{lik}\frac{\sigma^{k}}{2}=&\frac{i}{2}\left(2\epsilon_{lik}  \sigma^{k}\right)\nonumber\\
 i\epsilon_{lik}{\sigma^{k}}=&i\epsilon_{lik}  \sigma^{k}\,.
\end{align*}







\begin{subappendices}

\section{Dirac's Action}
\label{sec:dirac-equation}
The Scrodinger equation can be written as
\begin{align}
    i\frac{\partial}{\partial t}\psi=\hat{H}_{S} \psi\,,  
\end{align}
where
\begin{align}
  \hat{H}_{S}=
\end{align}


In order to have a well defined probabilty in relativistic quantum mechanics it is necessary that Lagrangian be linear in the time derivative, in order to obtain the general Sccödinger equation:
\begin{align}
  i\frac{\partial}{\partial t}\psi=\hat{H} \psi\,,  
\end{align}
like the Scrödinger Lagrangian. However, this automatically imply that the Lagrangian will be also linear in the spacial derivatives. A pure scalar field cannot involve a Lorentz invariant term of only first derivatives (see eq.~\eqref{eq:nolor}). Therefore the proposed field must have some internal structure associated with some representation of the Lorentz Group. Therefore we build the Lagrangian for a field of several components
\begin{align}
  \psi=  \begin{pmatrix}
\psi_1\\
\psi_2\\
\vdots\\
\psi_n    
  \end{pmatrix}
\end{align}

\subsection{Lorentz transformation}

If the field is to describe the electron. it must have spin and in this way it must transform under some spin representation of the Lorentz Group
\begin{align}
  \psi(x)\to \psi'(x)=S(\Lambda)\psi\left(\Lambda^{-1}x\right)\,.
\end{align}
One possible invariant could be the term $\psi^\dagger(x)\psi(x)$. However, under a Lorentz transformation we should have $\psi^\dagger S^\dagger S\psi$. As we cannot assume that $S(\Lambda)$ is unitary, the solution is to define the \emph{adjoint} spinor
\begin{align}
  \overline{\psi}=\psi^\dagger b\,.
\end{align}
which transforms as
\begin{align}
  \overline{\psi}(x)\to  \overline{\psi}'(x)&=
{\psi'}^\dagger(x)b=
\psi^\dagger\left(\Lambda^{-1}x\right)S^\dagger(\Lambda)b\,,
\end{align}
and,

\begin{align}
  \overline{\psi}(x)\psi(x)\to  \overline{\psi}'(x)\psi'(x)&=
\psi^\dagger\left(\Lambda^{-1}x\right)S^\dagger(\Lambda)b S(\Lambda)\psi\left(\Lambda^{-1}x\right)
\end{align}
The condition that must be fulfilled for Lorentz invariance of the Action is 
\begin{align}
  \label{eq:ltrinscal}
  S^\dagger(\Lambda)bS(\Lambda)=&b\,,
\end{align}
and therefore, 
\begin{align}
  \overline{\psi}(x)\psi(x)\to  \overline{\psi}'(x)\psi'(x)&=
\overline{\psi}\left(\Lambda^{-1}x\right)\psi\left(\Lambda^{-1}x\right)\,,
\end{align}
and:
\begin{align}
  \overline{\psi}(x)\to  \overline{\psi}'(x)&=
\psi^\dagger\left(\Lambda^{-1}x\right)b S^{-1}(\Lambda)\nonumber\\
&=\overline{\psi}\left(\Lambda^{-1}x\right)S^{-1}(\Lambda)\,.
\end{align}


A Action with a Lagrangian term linear in the derivatives, could be Lorentz invariant if, taking into account:
 \begin{align}
   \overline{\psi}(x)\gamma^\mu\partial_\mu\psi(x)\to  \overline{\psi'}(x)\gamma^\mu\partial_\mu\psi'(x)&=
 \overline{\psi}_a\left(\Lambda^{-1}x\right)S^{-1}_{ab}(\Lambda)\gamma^\mu_{bc}{\left(\Lambda^{-1}\right)^\rho}_\mu\partial_\rho S_{cd}(\Lambda)\psi_d\left(\Lambda^{-1}x\right)\nonumber\\
   &=
\overline{\psi} \psi\left(\Lambda^{-1}x\right){\left(\Lambda^{-1}\right)^\rho}_\mu \left(S^{-1}(\Lambda)\gamma^\mu S(\Lambda)\right)\partial_\rho\psi\left(\Lambda^{-1}x\right)\nonumber\\
&=\overline{\psi}(x)\gamma^\mu\partial_\mu\psi(x)\,,
 \end{align}
if the following condition is satisfied:
\begin{align}
\label{eq:ltrincond}
  S^{-1}(\Lambda)\gamma^\mu S(\Lambda)={\Lambda^\mu}_\sigma\gamma^\sigma\,.
\end{align}




the most general Lagrangian for this field is
\begin{align}
   \mathcal{L}&=i \overline{\psi} \gamma^\mu\partial_\mu\psi-m\overline{\psi} \psi\,,
\end{align}
Where the coefficients have been already fixed by convenience. Since the Action is real, it is convenient to rewrite this as
\begin{align}
   \mathcal{L}&=i \overline{\psi} \gamma^\mu\partial_\mu\psi-m\overline{\psi} \psi\nonumber\\
&=-\frac{1}{2}\partial_\mu\left(i \overline{\psi} \gamma^\mu\psi\right)+i \overline{\psi} \gamma^\mu\partial_\mu\psi-m\overline{\psi} \psi\nonumber\\
  &=-\frac{i}{2}(\partial_\mu \overline{\psi}) \gamma^\mu\psi-\frac{i}{2} \overline{\psi} \gamma^\mu\partial_\mu\psi+i \overline{\psi} \gamma^\mu\partial_\mu\psi-m\overline{\psi} \psi\nonumber\\
  &=\frac{i}{2} \overline{\psi} \gamma^\mu\partial_\mu\psi-\frac{i}{2}(\partial_\mu \overline{\psi}) \gamma^\mu\psi-m\overline{\psi} \psi\,.
\end{align}
 
Para que este nuevo Lagrangiano sea real se requiere que,
\begin{align}
  \label{eq:185qft}
  b^\dagger&=b\nonumber\\
  b^2&=I\nonumber\\
  b \gamma_\mu^\dagger b&=\gamma_\mu
\end{align}
ya que
\begin{align*}
  \mathcal{L}^\dagger&=\left(\frac{i}{2}\psi^\dagger \gamma_\mu^\dagger b \partial_\mu\psi-\frac{i}{2}\partial_\mu\psi^\dagger \gamma_\mu^\dagger b\psi\right)-m\psi^\dagger  b \psi\\
  &=\left(\frac{i}{2}\psi^\dagger b^2 \gamma_\mu^\dagger b \partial_\mu\psi-\frac{i}{2}\partial_\mu\psi^\dagger b^2 \gamma_\mu^\dagger b\psi\right)-m\psi^\dagger b \psi\\
  &=\left(\frac{i}{2}\bar{\psi} b \gamma_\mu^\dagger b \partial_\mu\psi-\frac{i}{2}\partial_\mu\bar{\psi}b \gamma_\mu^\dagger b\psi\right)-m\bar{\psi} \psi\\
  &=\left(\frac{i}{2}\bar{\psi} \gamma_\mu \partial_\mu\psi-\frac{i}{2}\partial_\mu\bar{\psi}\gamma_\mu \psi\right)-m\bar{\psi} \psi
\end{align*}

\subsection{Corriente conservada y Lagrangiano de Dirac}
\label{sec:corriente-conservada}
De la ec.~\eqref{eq:197qft}
\begin{align}
  J^0&=\left[\frac{\partial\mathcal{L}}{\partial\left(\partial_0\psi\right)}\right]\delta\psi+\delta\overline{\psi}\left[\frac{\partial\mathcal{L}}{\partial\left(\partial_0\overline{\psi}\right)}\right]\nonumber\\
  &=i\overline{\psi} \gamma^0 \delta\psi
\end{align}
El Lagrangiano es invariante bajo transformaciones de fase globales, $U(1)$
\begin{equation}
  \psi\to\psi'=e^{-i\alpha}\psi\approx\psi-i\alpha\psi,
\end{equation}
de modo que
\begin{equation}
  \delta\psi=-i\alpha\psi.
\end{equation}
Por consiguiente
\begin{equation}
  J^0=\alpha\overline{\psi} \gamma^0 \psi 
\end{equation}
Para que $J^0$ pueda interpretarse como una densidad de probabilidad, se debe cumplir
\begin{equation}
  \label{eq:bgamma0}
  b \gamma^0=I
\end{equation}


La  densidad de corriente es
\begin{align}
  J^0&\propto \psi^\dagger\psi\,.
\end{align}
Que podemos interpretar como una densidad de probabilidad.

De la ec.~\eqref{eq:bgamma0}, ya que la inversa de es única:
\begin{align}
  b=\gamma^0\,.
\end{align}
 
$\overline{\psi}$ se define como la \emph{adjunta} de $\psi$:
 \begin{align}
   \overline{\psi}=\psi^\dagger\gamma^0\,.
 \end{align}

It is convenient at this point to summarize the properties for $\gamma^0$:
\begin{align}
  \label{eq:cft77}
  {\gamma^0}^\dagger=&\gamma^0 & \left(\gamma^0\right)^2=&1 & \gamma^0{\gamma^\mu}^\dagger\gamma^0=&\gamma^\mu\nonumber\\
 &&   S^\dagger(\Lambda)\gamma^0S(\Lambda)=&\gamma^0\,. &&
\end{align}



En general
\begin{align}
   J^\mu&\propto\left[\frac{\partial\mathcal{L}}{\partial\left(\partial_\mu\psi\right)}\right]\delta\psi+\delta\bar{\psi}\left[\frac{\partial\mathcal{L}}{\partial\left(\partial_\mu\bar{\psi}\right)}\right]\nonumber\\
   &\propto i\bar{\psi}\gamma^\mu(-i\alpha\psi)\nonumber\\
   &\propto i\bar{\psi}\gamma^\mu(-i\alpha\psi)\nonumber\\
   &=\bar{\psi}\gamma^\mu\psi
\end{align}
y
\begin{equation}
     J^\mu=\psi^\dagger b \gamma^\mu\psi\,.
\end{equation}

\subsection{Tensor momento-energía}
\label{sec:tens-momento-energi}
\begin{align}
  T^0_0&=\frac{\partial\mathcal{L}}{\partial\left(\partial_0\psi\right)}\partial_0\psi+\partial_0\bar{\psi}\frac{\partial\mathcal{L}}{\partial\left(\partial_0\bar{\psi}\right)}-\mathcal{L}\nonumber\\
  &=i\bar{\psi}\gamma^0\partial_0\psi-\mathcal{L}\nonumber\\
  &=-i\bar{\psi}\gamma^i\partial_i\psi+m\bar{\psi} \psi,\nonumber\\
  &=\bar{\psi}(\boldsymbol{\gamma}\cdot\mathbf{p}+m)\psi,\nonumber\\
  &=\psi^\dagger \gamma^0(\boldsymbol{\gamma}\cdot\mathbf{p}+m)\psi,\nonumber\\
  \label{eq:118qft}
  &=\psi^\dagger\hat{H} \psi,
\end{align}
donde
\begin{equation}
  \label{eq:denshal}
  \hat{H}= \gamma^0(\boldsymbol{\gamma}\cdot\mathbf{p}+m)
\end{equation}
la ecuación de Scröndinger de validez general es entonces:
\begin{equation}
  i\frac{\partial}{\partial t}\psi=\hat{H} \psi
\end{equation}
y, como en mecánica clásica usual
\begin{equation}
  \label{eq:99qft}
  \langle\hat{H}\rangle=\int \psi^\dagger\hat{H} \psi\,d^3x.
\end{equation}


Además
\begin{align}
    T^0_i&=\frac{\partial\mathcal{L}}{\partial\left(\partial_0\psi\right)}\partial_i\psi+\partial_i\bar{\psi}\frac{\partial\mathcal{L}}{\partial\left(\partial_0\bar{\psi}\right)}\nonumber\\
    &=i\bar{\psi}\gamma^0 \partial_i\psi\nonumber\\
    &=-\psi^\dagger(-i\partial_i)\psi
\end{align}
de modo que
\begin{equation}
  \langle\hat{\mathbf{p}}\rangle=\int\psi^\dagger\hat{\mathbf{p}}\psi\,d^3 x
\end{equation}
\subsection{Ecuaciones de Euler-Lagrange}
\label{sec:ecuaciones-de-euler}
Queremos que el Lagrangiano de lugar a la ecuación de Scröndinger de validez general
\begin{equation}
  \label{eq:grlsch}
  i\frac{\partial}{\partial t}\psi=\hat{H} \psi
\end{equation}
con el Hamiltoniano dado en la ec.~(\ref{eq:99qft}), que corresponde a un Lagrangiano de sólo derivadas de primer orden y covariante, en lugar del Hamiltoniano para el caso no relativista. 

De hecho, aplicando las ecuaciones de Euler-Lagrange para el campo $\bar{\psi}$ al Lagrangiano en ec.~(\ref{eq:100qft}) ,tenemos
\begin{align}
  \partial_\mu\left[\frac{\partial\mathcal{L}}{\partial\left(\partial_\mu\bar{\psi}\right)}\right]-\frac{\partial\mathcal{L}}{\partial\bar{\psi}}&=0\nonumber\\
  \frac{\partial\mathcal{L}}{\partial\bar{\psi}}&=0\nonumber\\
  \label{eq:114qftm}
  i\gamma^\mu\partial_\mu\psi-m\psi&=0.
\end{align}
Expandiendo
\begin{align*}
  i\gamma^0\partial_0\psi+i\gamma^i\partial_i\psi-m\psi&=0\\
  i\gamma^0\partial_0\psi-\boldsymbol{\gamma}\cdot(-i\boldsymbol{\nabla})\psi-m\psi&=0,\\
  i\gamma^0\partial_0\psi&=(\boldsymbol{\gamma}\cdot\hat{\mathbf{p}}+m)\psi,
\end{align*}
de donde
\begin{equation}
    i{\gamma^0}^2\frac{\partial}{\partial t}\psi=\gamma^0(\boldsymbol{\gamma}\cdot\mathbf{p}+m)\psi.
\end{equation}
 tenemos que
\begin{align}
  \label{eq:gamma02}
  \left(\gamma^0\right)^2=1.
\end{align}
De la ec.~(\ref{eq:denshal})
\begin{equation}
  \label{eq:186qft}
  \hat{H}= \gamma^0(\boldsymbol{\gamma}\cdot\mathbf{p}+m),
\end{equation}
A este punto, sólo nos queda por determinar los parámetros $\gamma^\mu$. 

La ec.~(\ref{eq:grlsch}) puede escribirse como
\begin{equation}
  \left(i\frac{\partial}{\partial t}-\hat{H}\right)\psi=0.
\end{equation}
El campo $\psi$ también debe satisfacer la ecuación de Klein-Gordon. Podemos derivar dicha ecuación aplicando el operador
\begin{equation*}
  \left(-i\frac{\partial}{\partial t}-\hat{H}\right)
\end{equation*}
De modo que, teniendo en cuenta que $\partial\hat H/\partial t=0$,
\begin{align}
  \label{eq:105qft}
 \left(-i\frac{\partial}{\partial t}-\hat{H}\right)\left(i\frac{\partial}{\partial t}-\hat{H}\right)\psi&=0\nonumber\\
 \left(-i\frac{\partial}{\partial t}-\hat{H}\right)\left(i\frac{\partial\psi}{\partial t}-\hat{H}\psi\right)&=0\nonumber\\
 \frac{\partial^2\psi}{\partial t^2}+i\left(\frac{\partial\hat{H}}{\partial t}\right)\psi
 +i\hat{H}\frac{\partial\psi}{\partial t}-i\hat{H}\frac{\partial\psi}{\partial t}+\hat{H}^2\psi&=0\nonumber\\
 \left(\frac{\partial^2}{\partial t^2}+\hat{H}^2\right)\psi&=0.
\end{align}
% 
De la ec.~(\ref{eq:186qft}), y usando la condición en ec.~(\ref{eq:gamma02}), tenemos
\begin{align}
\label{eq:106qft}
\hat{H}^2&=(\gamma_0\boldsymbol{\gamma}\cdot\mathbf{p}+\gamma_0\,m)(\gamma_0\boldsymbol{\gamma}\cdot\mathbf{p}+\gamma_0\,m)\nonumber\\
&=(\gamma_0\boldsymbol{\gamma}\cdot\mathbf{p})(\gamma_0\boldsymbol{\gamma}\cdot\mathbf{p})+m\gamma_0\boldsymbol{\gamma}\cdot\mathbf{p}\gamma_0+m\gamma_0^2\boldsymbol{\gamma}\cdot\mathbf{p}+m^2
\end{align}
Sea
\begin{align}
  \beta&=\gamma^0\nonumber\\
  \alpha^i&=\beta\gamma^i\nonumber\\
  \gamma^i&=\beta\alpha^i
\end{align}
\begin{align}
  \hat{H}^2&=(\boldsymbol{\alpha}\cdot\mathbf{p})(\boldsymbol{\alpha}\cdot\mathbf{p})
  +m\boldsymbol{\alpha}\cdot\mathbf{p}\beta+m\beta\boldsymbol{\alpha}\cdot\mathbf{p}+m^2\nonumber\\
  &=(\boldsymbol{\alpha}\cdot\mathbf{p})(\boldsymbol{\alpha}\cdot\mathbf{p})
  +m(\boldsymbol{\alpha}\beta+\beta\boldsymbol{\alpha})\cdot\mathbf{p}+m^2
\end{align}
Sea $A$ una matriz y $\theta$ en un escalar. Entonces tenemos la identidad
\begin{align}
  \label{eq:206qft}
  (\mathbf{A}\cdot\boldsymbol{\theta})^2=\sum_i {A^i}^2 {\theta^i}^2+\sum_{i\lt j}\left\{A^i,A^j  \right\}\theta^i \theta^j 
\end{align}
\begin{itemize}
\item \textbf{Demostración}
  \begin{align}
    \left[\left(\mathbf{A}\cdot\boldsymbol{\theta}\right)\right]_{\alpha\beta}
    =&\sum_{i j}\sum_\gamma A^i_{\alpha\gamma}\theta^iA^j_{\gamma\beta}\theta^j\nonumber\\    
    =&\sum_{i j}\theta^i\theta^j\sum_\gamma A^i_{\alpha\gamma}A^j_{\gamma\beta}\nonumber\\    
    =&\sum_\gamma \sum_{i j}\theta^i\theta^jA^i_{\alpha\gamma}A^j_{\gamma\beta}\nonumber\\    
    =&\sum_\gamma \left(\sum_{i}{\theta^i}^2A^i_{\alpha\gamma}A^i_{\gamma\beta}+\sum_{i<j}\theta^i\theta^jA^i_{\alpha\gamma}A^j_{\gamma\beta}+\sum_{i>j}\theta^i\theta^jA^i_{\alpha\gamma}A^j_{\gamma\beta}\right)\nonumber\\    
    =&\sum_\gamma \left(\sum_{i}{\theta^i}^2A^i_{\alpha\gamma}A^i_{\gamma\beta}+\sum_{i<j}\theta^i\theta^jA^i_{\alpha\gamma}A^j_{\gamma\beta}+\sum_{j>i}\theta^j\theta^iA^j_{\alpha\gamma}A^i_{\gamma\beta}\right)\nonumber\\    
    =&\sum_\gamma \left[\sum_{i}{\theta^i}^2A^i_{\alpha\gamma}A^i_{\gamma\beta}+\sum_{i<j}\theta^i\theta^j\left(A^i_{\alpha\gamma}A^j_{\gamma\beta}+A^j_{\alpha\gamma}A^i_{\gamma\beta}\right)\right]\nonumber\\    
    =&\left[\sum_{i}{\theta^i}^2\left(A^iA^i\right)_{\alpha\beta}+\sum_{i<j}\theta^i\theta^j\left\{ A^i,A^j\right\}_{\alpha\beta}\right]\nonumber\\    
    =&\left[\sum_{i}{\theta^i}^2{A^i}^2+\sum_{i<j}\theta^i\theta^j\left\{ A^i,A^j\right\}\right]_{\alpha\beta}\,.
  \end{align}

\end{itemize}
Entonces
\begin{align}
  \hat{H}^2=&\alpha_i^2p_i^2+\sum_{i\lt j}\left\{\alpha_i,\alpha_j\right\}p_i p_j+m(\alpha_i \beta+\beta\alpha_i)p_i+m^2
\end{align}
(suma sobre índices repetidos). Si
\begin{align}
  \label{eq:107qft}
  \alpha_i^2&=1\nonumber\\
  \left\{\alpha_i,\alpha_j\right\}&=0\qquad i\neq j\nonumber\\
  \alpha_i \beta+\beta\alpha_i&=0
\end{align}
\begin{equation}
  \hat{H}^2=-\boldsymbol{\nabla}^2+m^2
\end{equation}
y reemplazando en la ec.~\eqref{eq:105qft} llegamos a la ecuación de Klein-Gordon para $\psi$
\begin{align}
   \left(\frac{\partial^2}{\partial t^2}-\boldsymbol{\nabla}^2+m^2\right)\psi&=0\nonumber\\
   \left(\Box+m^2\right)\psi&=0
\end{align}
En términos de las matrices $\gamma^\mu$ las condiciones en ec.~\eqref{eq:107qft} son
\begin{align}
  \label{eq:108qft}
  \left({\gamma^0}\right)^2&=1\nonumber\\
  \left({\alpha^i}\right)^2=1\to\gamma^0\gamma^i \gamma^0\gamma^i=-\left({\gamma^i}\right)^2=1\to\left({\gamma^i}\right)^2&=-1\nonumber\\
  \gamma^i \gamma^0+\gamma^0\gamma^i=\left\{\gamma^i,\gamma^0\right\}&=0
\end{align}
De modo que
\begin{align}
  \label{eq:198qft}
\left\{\alpha^i,\alpha^j\right\}=\gamma^0\gamma^i \gamma^0\gamma^j+\gamma^0\gamma^j \gamma^0\gamma^i&=0\qquad i\neq j\nonumber\\
-\gamma^0\gamma^0\gamma^i \gamma^j-\gamma^0\gamma^0\gamma^j \gamma^i&=0\qquad i\neq j\nonumber\\
\gamma^i \gamma^j+\gamma^j \gamma^i&=0\qquad i\neq j\nonumber\\
\left\{\gamma^i,\gamma^j\right\}&=0\qquad i\neq j
\end{align}
Las ecuaciones \eqref{eq:108qft}\eqref{eq:198qft} pueden escribirse como
\begin{equation}
  \label{eq:109qft}
  \left\{\gamma^\mu,\gamma^\nu\right\}\equiv\gamma^\mu\gamma^\nu+\gamma^\nu\gamma^\mu=2g^{\mu\nu}\mathbf{1}
\end{equation}
donde
\begin{align}
  \gamma^\mu=(\gamma^0,\gamma^i)
\end{align}
Además, de la ec.~\eqref{eq:cft77}
\begin{equation}
  \label{eq:112qft}
   \gamma^0{\gamma^\mu}^\dagger \gamma^0=\gamma^\mu.
\end{equation}
Cualquier conjunto de matrices que satisfagan el álgebra en ec.~\eqref{eq:109qft} y la condición en ec.~\eqref{eq:112qft}, se conocen como matrices de Dirac. A $\psi$ se le llama espinor de Dirac.

En términos de la matrices $\gamma^\mu$, el Lagrangiano de Dirac y la ecuación de Dirac, son respectivamente de las ecs.~(\ref{eq:100qft}) y (\ref{eq:114qft})
\begin{equation}
  \label{eq:115qft}
  \mathcal{L}=\bar{\psi}\left(i\gamma^\mu\partial_\mu-m\right)\psi,
\end{equation}
\begin{equation}
  \label{eq:116qft}
  i\gamma^\mu\partial_\mu\psi-m\psi=0,
\end{equation}
donde
\begin{equation}
  \bar{\psi}=\psi^\dagger\gamma^0.
\end{equation}





\subsection{Propiedades de las matrices de Dirac}
\label{sec:propiedades-de-las}
De la ec.~(\ref{eq:112qft})
\begin{equation}
  {\gamma^\mu}^\dagger=\gamma^0\gamma^\mu\gamma^0\Rightarrow  
  \begin{cases}
    {\gamma^0}^\dagger=\gamma^0&\mu=0\\
    {\gamma^i}^\dagger=-{\gamma^0}^2\gamma^i=-\gamma^i&\mu=i
  \end{cases}.
\end{equation}
Definiendo
\begin{equation}
\label{eq:117qft}
  \gamma_5=i\gamma_0\gamma_1\gamma_2\gamma_3,
\end{equation}
entonces,
\begin{align}
  \gamma_5^2=&-\gamma_0\gamma_1\gamma_2\gamma_3\gamma_0\gamma_1\gamma_2\gamma_3\nonumber\\
  \gamma_5^2=&+\gamma_0^2\gamma_1\gamma_2\gamma_3\gamma_1\gamma_2\gamma_3\nonumber\\
  \gamma_5^2=&+\gamma_1\gamma_2\gamma_3\gamma_1\gamma_2\gamma_3\nonumber\\
  \gamma_5^2=&-\gamma_2\gamma_3\gamma_2\gamma_3\nonumber\\
  \gamma_5^2=&\gamma_2\gamma_2\gamma_3\gamma_3\nonumber\\
  \gamma_5^2=&\mathbf{1}\,.
\end{align}

\begin{equation}
  \gamma_5^2=\mathbf{1},
\end{equation}
Teniendo en cuenta que $\gamma_\mu^2\propto\mathbf{1}$ y conmuta con las demás matrices, tenemos por ejemplo
\begin{align}
  \gamma_5\gamma_3=&i\gamma_0\gamma_1\gamma_2\gamma_3^2=\gamma_3^2i\gamma_0\gamma_1\gamma_2=-\gamma_3i\gamma_0\gamma_1\gamma_2\gamma_3=-\gamma_3\gamma_5\nonumber\\
  \gamma_5\gamma_2=&-i\gamma_0\gamma_1\gamma_2^2\gamma_3=-\gamma_2^2i\gamma_0\gamma_1\gamma_3=-\gamma_2i\gamma_0\gamma_1\gamma_2\gamma_3=-\gamma_2\gamma_5\nonumber\\
  \gamma_5\gamma_1=&i\gamma_0\gamma_1^2\gamma_2\gamma_3=\gamma_1^2i\gamma_0\gamma_2\gamma_3=-\gamma_1i\gamma_0\gamma_1\gamma_2\gamma_3=-\gamma_1\gamma_5\nonumber\\
  \gamma_5\gamma_0=&i\gamma_0\gamma_1\gamma_2\gamma_3\gamma_0=-\gamma_0^2i\gamma_1\gamma_2\gamma_3=-\gamma_0\gamma_5\,.
\end{align}
De modo que
\begin{equation}
  \label{eq:218qft}
  \left\{\gamma_\mu,\gamma_5\right\}=0. 
\end{equation}
Expandiendo el anticonmutador tenemos
\begin{align}
  \gamma_\mu\gamma_5=-\gamma_5\gamma_\mu\nonumber\\
  \gamma_5\gamma_\mu\gamma_5=-\gamma_\mu\nonumber\\
\operatorname{Tr}\left(\gamma_5\gamma_\mu\gamma_5\right)=-\operatorname{Tr}\gamma_\mu\nonumber\\
\operatorname{Tr}\left(\gamma_5\gamma_5\gamma_\mu\right)=-\operatorname{Tr}\gamma_\mu\nonumber\\
\operatorname{Tr}\gamma_\mu=-\operatorname{Tr}\gamma_\mu,
\end{align}
y por consiguiente
\begin{equation}
  \operatorname{Tr}\gamma_\mu=0.
\end{equation}


De otro lado, si
\begin{equation}
  \tilde{\gamma_\mu}\equiv U\gamma_\mu U^\dagger,
\end{equation}
para alguna matriz unitaria $U$, entonces $\tilde{\gamma_\mu}$ corresponde a otra representación de álgebra de Dirac en ec.~(\ref{eq:109qft}), ya que
\begin{align}
  \left\{\tilde\gamma^\mu,\tilde\gamma^\nu\right\}&=\left\{U\gamma^\mu U^\dagger,U\gamma^\nu U^\dagger\right\}\nonumber\\
  &=U\left\{\gamma^\mu,\gamma^\nu\right\}U^\dagger\nonumber\\
  &=2g^{\mu\nu}UU^\dagger\nonumber\\
  &=2g^{\mu\nu}\mathbf{1}.
\end{align}
Claramente, la condición en ec.~(\ref{eq:112qft}) se mantiene para la nueva representación. Como $\gamma_0$ es hermítica, siempre es posible escoger una representación tal que $\tilde{\gamma_0}\equiv U\gamma_0U^\dagger$ sea diagonal. Como $\gamma_0^2=1$, sus entradas en la diagonal deben ser $\pm1$, y como $\operatorname{Tr}\tilde\gamma_0=0$, debe existir igual número de $+1$ que de $-1$. Por lo tanto la dimensión de $\gamma_0$ (y de $\gamma_\mu$) debe ser par: $2,4,\ldots$. Para un fermion sin masa
\begin{align}
  \mathcal{L}=i\psi^\dagger\gamma^0\gamma^0\partial_0\psi+i\psi^\dagger\gamma^0\gamma^i\partial_i\psi=i\psi^\dagger\partial_0\psi+i\psi^\dagger\alpha^i\partial_i\psi\,,
\end{align}
solo se requieren tres matrices $2\times 2$ que satisfacen
\begin{align}
  \left\{\alpha^i,\alpha^j\right\} =2\delta^{ij}\,,
\end{align}
y por lo tanto pueden identificarse con las tres matrices de Pauli. 
Como en general tenemos 4 matrices independientes, su dimensión mínima debe ser 4.

Como $\tilde\gamma^i=\gamma^0\gamma^i\gamma^0={\gamma^i}^\dagger=-\gamma^i$, podemos definir la \emph{representación de paridad}
\begin{align}
\label{eq:parityrep}
\tilde\gamma^0=&\gamma^0,\qquad\tilde\gamma^i=-\gamma^i\,,&\text{para}\qquad U=&\gamma^0   
\end{align}




\begin{inprogress}
  \subsection{Dirac representation}
The set of $4\times 4$ matrices
\begin{align}
  S^{\mu\nu}=\frac{i}{4}\left[\gamma^\mu,\gamma^\nu\right]\,,
\end{align}
also satisfy the commutations relations \eqref{eq:lrtalg}, and constitute a new matrix representation of the Lorentz Group. The subgroup of rotation group has the generators $S^{ij}$. We define the spin matrices:
\begin{align}
  \Sigma^i=\epsilon^{ijk}S_{jk}\,,
\end{align}
taking into account that
\begin{align}
  \gamma_0\gamma_i\gamma_5=-\epsilon_{0ijk}S^{jk}\,,
\end{align}
Taking th convention $\epsilon_{0ijk}=\epsilon_{ijk}$, we have
\begin{align}
  \Sigma_{i}=-\gamma_0\gamma_i\gamma_5\,.
\end{align}
With this form, it is easy to show that
\begin{align}
  \left\{\Sigma_i,\Sigma_j\right\}=2\delta_{ij}\,, 
\end{align}
so that 
\begin{align}
  \left[\frac{\Sigma^i}{2},\frac{\Sigma^j}{2} \right]=i\,\epsilon_{ijk}\frac{\Sigma^k}{2}
\end{align}
\end{inprogress}



\subsection{Lorentz invariance of the Dirac Action}
We need to satisfy the following conditions
\begin{align}
  S^{-1}(\Lambda) \gamma^\mu S(\Lambda)=&{\Lambda^\mu}_\nu\gamma^\nu\nonumber\\
  S^\dagger(\Lambda) \gamma^0 S(\Lambda)=&\gamma^0\qquad\text{or}\quad S^\dagger(\Lambda) \gamma^0= \gamma^0S^{-1}(\Lambda)\, .
\end{align}
We now set the notation
\begin{align}
\label{eq:lorentzrep}
  \Lambda=1+\xi_ib^i+\frac{1}{2}\theta_i\epsilon_{i j k}r^{jk}\,,
\end{align}
\begin{align}
 b^i=&-i J^{i0} & r^{jk}=-i J^{j k}\,.
\end{align}

In order to find a representation of the Lorentz Group in terms of the Dirac matrices we propose
  \begin{align}
    \label{eq:diraclorentzrep}
  S(\Lambda)=1+\xi_iB^i+\frac{1}{2}\theta_i\epsilon_{i j k}R^{jk}\,.
\end{align}
Instead of show the Lorentz invariance of the Dirac Action, we use the conditions derived from the invariance, to find a representation in terms of the Dirac matrices for $B^i$ and $R^{jk}$. As a consistency check, the resulting representation would satisfy the Lorentz algebra. In this way, by using eq.~\eqref{eq:lorentzrep} and \eqref{eq:diraclorentzrep}, we obtain from 
\begin{align}
  S^{-1}(\Lambda)\gamma^\mu S(\Lambda)={\Lambda^\mu}_\nu\gamma^\nu\,,
\end{align}
that
\begin{align}
  B^i=\frac{1}{2}\gamma^0\gamma^i\nonumber\\
  R^{jk}=&\frac{1}{2}\gamma^j\gamma^k\,,
\end{align}
which can be written in covariant form if we define
\begin{align}
  \mathcal{S}^{\mu\nu}=\frac{i}{4}\left[\gamma^\mu,\gamma^\nu\right]\,.
\end{align}
In fact, the six set of non-zero independently generators are
\begin{align}
  \mathcal{S}^{0i}=&\frac{i}{4}\left(\gamma^0\gamma^i-\gamma^i\gamma^0\right)=\frac{i}{2}\gamma^0\gamma^i= i B^i\nonumber\\
  \mathcal{S}^{i j}=&\frac{i}{4}\left(\gamma^i\gamma^j-\gamma^j\gamma^i\right)=\frac{i}{2}\gamma^i\gamma^j= i R^{i j}\,.
\end{align}
It is worth notices that in fact $\mathcal{S}^{\mu\nu}$ satisfy the Lorentz algebra, and therefore are the generators of the Lorentz group elements:
\begin{align}
  S(\Lambda)=&\exp\left(-i \omega_{\mu\nu}\frac{\mathcal{S}^{\mu\nu}}{2}\right)\nonumber\\
  \approx&1-\frac{i}{2} \omega_{\mu\nu}{\mathcal{S}^{\mu\nu}}\,.
\end{align}
Another consistency check is
\begin{align}
  S^\dagger(\Lambda)\gamma^0S(\Lambda)=&\gamma^0\,,
\end{align}
or equivalently
\begin{align}
S^\dagger(\Lambda)\gamma^0=&\gamma^0S^{-1}(\Lambda)\nonumber\\
\left(1+\frac{i}{2} \omega_{\mu\nu}{\mathcal{S}^{\mu\nu}}^\dagger \right)\gamma^0=&\gamma^0\left(1+\frac{i}{2} \omega_{\mu\nu}{\mathcal{S}^{\mu\nu}}\right)\nonumber\\
{\mathcal{S}^{\mu\nu}}^\dagger \gamma^0=&\gamma^0{\mathcal{S}^{\mu\nu}}\,.
\end{align}
Taking into account that
\begin{align}
  {\gamma^\mu}^\dagger{\gamma^\nu}^\dagger\gamma^0=\left(\gamma^0\right)^2{\gamma^\mu}^\dagger\left(\gamma^0\right)^2{\gamma^\nu}^\dagger\gamma^0=\gamma^0\gamma^\mu\gamma^\nu\,,
\end{align}
we have
\begin{align}
  {\mathcal{S}^{\mu\nu}}^\dagger \gamma^0=&-\frac{i}{4}\left[\gamma^\mu,\gamma^\nu\right]^\dagger\gamma^0\nonumber\\
=&-\frac{i}{4}\left[{\gamma^\nu}^\dagger,{\gamma^\mu}^\dagger\right]\gamma^0\nonumber\\
=&\frac{i}{4}\left[{\gamma^\mu}^\dagger,{\gamma^\nu}^\dagger\right]\gamma^0\nonumber\\
=&\frac{i}{4}\left[{\gamma^\mu},{\gamma^\nu}\right]\gamma^0\nonumber\\
=&\gamma^0\mathcal{S}^{\mu\nu}\nonumber\\
\end{align}


\subsection{Dirac's Lagrangian}
\label{sec:diracs-lagrangian}

Para una matriz de $n$ dimensiones existen $n^2$ matrices hermíticas (o anti--hermíticas) independientes. Si se sustrae la identidad quedan $n^2-1$ matrices hermíticas (o anti--hermíticas) independientes de traza nula. En el caso $n=2$ corresponden a las 3 matrices de Pauli. En el caso de la ecuación de Dirac se requieren 4 matrices independientes, por lo tanto deben ser matrices $4\times 4$. En efecto para $n=4$ existen 15 matrices independientes de traza nula dentro de las cuales podemos acomodar sin problemas las 4 $\gamma^\mu$. 

De \cite{Gross:1993}:
\begin{quote}
  All Dirac matrix elements will now be written in the form
  \begin{align}
    \overline{\psi}(x)\Gamma\psi(x)\,,
  \end{align}
where $\Gamma$ is a $4\times 4$ complex matrix. The most general such matrix can always be expanded in terms of 16 independent $4\times 4$ matrices multiplied by complex coefficients. In short the matrices $\Gamma$ can be regarded as a \emph{16--dimensional complex vector space} spanned by 16 matrices.

It is convenient to choose the 16 matrices, $\Gamma_i$, so that they have well defined transformation properties under the Lorentz Transformations. Since the $\gamma^\mu$'s have such properties, we are lead to choose the following 16 matrices for this basis:
\end{quote}


En la Tabla~\ref{tab:Gamma} se muestran las matrices de traza nula con sus propiedades de transformación bajo el Grupo de Lorentz. En la última se muestra el correspondiente escalar en el espacio de Dirac $\bar\psi\Gamma\psi$.
%instiki:
\begin{table} %noinstiki
  \centering %noinstiki
  \begin{tabular}{l|l|l|l} %noinstiki
Matriz $\Gamma$&Transformación&Número&Escalar en Dirac\\\hline{}
%instiki:
$\mathbf{1}$&Escalar (S)&1&$\bar\psi\psi$\\
%instiki:
$\gamma_5$&Pseudoescalar (P)&1&$\bar\psi\gamma_5\psi$\\
%instiki:
$\gamma_\mu$&Vector (V)&4&$\bar\psi\gamma_\mu\psi$\\
%instiki:
$\gamma_\mu\gamma_5$ &Vector axial (A)&4&$\bar\psi\gamma_\mu\gamma_5\psi$\\
%instiki:
$\sigma_{\mu\nu}=\frac{i}{2}\left[\gamma_\mu,\gamma_\nu\right]$&Tensor antisimétrico (T)&6&$\bar\psi\sigma_{\mu\nu}\psi$\\\hline{}
%instiki:
&&16&\\
  \end{tabular} %noinstiki
  \caption{Matrices $\Gamma_i$.} %noinstiki
\label{tab:Gamma} %noinstiki
\end{table} %noinstiki
%instiki:
Demostración
\begin{align}
J^\mu(x)\equiv  \bar\psi(x)\gamma^\mu\psi(x)\to&\bar\psi(\Lambda^{-1}x)S^{-1}(\Lambda)\gamma^\mu S(\Lambda)\psi(\Lambda^{-1}x) \nonumber\\
=&{\Lambda^\mu}_\nu\bar\psi(\Lambda^{-1}x) \gamma^\nu\psi(\Lambda^{-1}x) \nonumber\\
=&{\Lambda^\mu}_\nu J^\nu(\Lambda^{-1}x)\,.
\end{align}
In \cite{Gross:1993}: Problem 5.4: 
\begin{align}
  \overline{\psi}\gamma_5\psi\to\overline{\psi}S^{-1}(\Lambda)\gamma^5S(\Lambda)\psi =(\det\Lambda)\overline{\psi}\gamma_5\psi
\end{align}
The solution is in Appendix C. of Burgess book, by using
\begin{align}
  \gamma^5=\frac{i}{24}\epsilon_{\mu\nu\alpha\beta}\gamma^\mu\gamma^\nu\gamma^\alpha\gamma^\beta
\end{align}
and
\begin{align}
  \det \Lambda=\epsilon_{\mu\nu\alpha\beta}{\Lambda^\mu}_1{\Lambda^\nu}_2{\Lambda^\alpha}_3{\Lambda^\beta}_4\,.
\end{align}

\end{subappendices}

% \left(\right)
%
%%% Local Variables: 
%%% mode: latex
%%% TeX-master: "fullnotes"
%%% ispell-local-dictionary: "castellano8"
%%% End: 
